%!TeX root = main.tex
\section{Fondamenti logico-matematici della Meccanica Quantistica}
\subsection{Grandezze Classiche e Misurabilità}

Le quantità classiche più complesse di un sistema possono essere descritte mediante funzioni misurabili secondo Borel definite sullo spazio delle fasi $\Gamma$. La scelta della $\sigma$-algebra di Borel è naturale in quanto determinata dalla topologia che si assume esistere su $\Gamma$.

\begin{definizione}[Grandezza Fisica e Valore di Aspettazione]
	Una grandezza fisica è rappresentata da una funzione Borel-misurabile $f: \Gamma \to \mathbb{R}$. La misurabilità è un requisito fondamentale che permette di definire operazioni fisiche, come il calcolo del \textit{valore di aspettazione} quando lo stato è descritto da una misura $\mu$:
	$$
	\langle f \rangle_\mu = \int_\Gamma f \, d\mu \,.
	$$
\end{definizione}

\begin{osservazione}[Proprietà Elementari]
	Le cosiddette ``proprietà elementari'' del sistema sono descritte da funzioni misurabili che assumono solo valori binari, ovvero $g: \Gamma \to \{0,1\}$. Queste funzioni sono identificate fedelmente con le funzioni caratteristiche (o indicatrici) di insiemi di Borel.
	Dato un insieme di Borel $E_g = g^{-1}(\{1\})$, si ha infatti $g = \chi_{E_g}$.
	È importante notare che richiedere la continuità per queste funzioni sarebbe troppo restrittivo: le proprietà elementari non sarebbero rappresentabili se ci si limitasse alle mappe continue.
\end{osservazione}

Una generica quantità fisica $f: \Gamma \to \mathbb{R}$ è completamente determinata dalla classe degli insiemi di Borel $E_B^{(f)}$, definiti come la controimmagine di un insieme $B \in \mathscr{B}(\mathbb{R})$:
\begin{equation}
	E_B^{(f)} := f^{-1}(B) \,.
\end{equation}
Il significato fisico dell'insieme $E_B^{(f)}$ è: ``l'insieme degli stati in cui il valore di $f$ appartiene a $B$''.

\begin{proposizione}[Proprietà della Mappa Inversa]
	La mappa $\mathscr{B}(\mathbb{R}) \ni B \mapsto E_B^{(f)}$ permette di ricostruire la funzione $f$. La classe degli insiemi $E_B^{(f)}$ forma una $\sigma$-algebra e soddisfa le seguenti proprietà elementari al variare di $B$ in $\mathscr{B}(\mathbb{R})$:
	\begin{itemize}
		\item[(Fi)] $E_{\mathbb{R}}^{(f)} = \Gamma$;
		\item[(Fii)] $E_B^{(f)} \cap E_C^{(f)} = E_{B \cap C}^{(f)}$;
		\item[(Fiii)] Se $N \subset \mathbb{N}$ e $\{B_k\}_{k \in N} \subset \mathscr{B}(\mathbb{R})$ è una famiglia di insiemi a due a due disgiunti ($B_j \cap B_k = \varnothing$ per $k \neq j$), allora:
		$$
		\bigcup_{j \in N} E_{B_j}^{(f)} = E_{\bigcup_{j \in N} B_j}^{(f)} \,.
		$$
	\end{itemize}
\end{proposizione}

\begin{osservazione}[Commenti sulle proprietà]
	Si possono fare alcune precisazioni sulle proprietà sopra elencate:
	\begin{enumerate}
		\item[(a)] Utilizzando (Fi) e (Fiii), la proprietà (Fii) può essere sostituita dalla proprietà sul complemento: $E_{\mathbb{R}\setminus E}^{(f)} = \Gamma \setminus E_E^{(f)}$. In particolare, si ha $E_{\varnothing}^{(f)} = \varnothing$.
		\item[(b)] Le proprietà (Fii) e (Fiii) implicano che la condizione di disgiunzione in (Fiii) non è strettamente necessaria; l'uguaglianza vale anche rimuovendo il vincolo $B_j \cap B_k = \varnothing$.
		\item[(c)] Infine, invocando le leggi di De Morgan, la proprietà (Fiii) è valida anche sostituendo l'unione $\cup$ con l'intersezione $\cap$.
	\end{enumerate}
\end{osservazione}

\subsection{La nozione di Reticolo}

Tornando alla struttura matematica delle proprietà elementari classiche, osserviamo che esse possono essere inquadrate nella nozione algebrica di \textit{reticolo}.

Richiamiamo brevemente che in un insieme parzialmente ordinato (o \textit{poset}) $(X, \geq)$, se $Y \subset X$, il simbolo $\sup Y$ denota (se esiste) il più piccolo elemento $x \in X$ tale che $x \geq y$ per ogni $y \in Y$. Analogamente, $\inf Y$ denota (se esiste) il più grande elemento $x \in X$ tale che $y \geq x$ per ogni $y \in Y$.

\begin{definizione}[Reticolo]
	Un insieme parzialmente ordinato $(X, \geq)$ è definito \textbf{reticolo} (in inglese \textit{lattice}) se, per ogni coppia di elementi $a, b \in X$:
	\begin{itemize}
		\item[(a)] Esiste in $X$ $\sup\{a, b\}$, denominato \textbf{join} e indicato con $a \lor b$;
		\item[(b)] Esiste in $X$ $\inf\{a, b\}$, denominato \textbf{meet} e indicato con $a \land b$.
	\end{itemize}
	Si noti che il poset $(X, \geq)$ non deve necessariamente essere totalmente ordinato. La notazione $a \leq b$ è equivalente a $b \geq a$.
\end{definizione}

\begin{osservazione}
	\begin{itemize}
		\item[(a)] Nei casi concreti in cui $X = \mathscr{B}(\mathbb{R})$ o $X = \mathscr{B}(\Gamma)$, la relazione d'ordine $\geq$ coincide con l'inclusione insiemistica $\supseteq$. Di conseguenza, l'operazione di join $\lor$ corrisponde all'unione $\cup$, mentre il meet $\land$ corrisponde all'intersezione $\cap$.
		\item[(b)] In un reticolo generale, le operazioni $\lor$ e $\land$ risultano essere \textit{associative} e \textit{commutative}. Valgono inoltre le \textbf{leggi di assorbimento}:
		$$
		a \lor (a \land b) = a \quad \text{e} \quad a \land (a \lor b) = a \,.
		$$
		\item[(c)] Esiste una stretta relazione tra l'ordine e le operazioni: in un reticolo, $a \geq b$ se e solo se $a \lor b = a$ (o equivalentemente $a \land b = b$).
	\end{itemize}
\end{osservazione}

\begin{definizione}[Algebra di Boole]
	Un reticolo $(X, \geq)$ si dice:
	\begin{itemize}
		\item[(a)] \textbf{Distributivo}: se le operazioni $\lor$ e $\land$ si distribuiscono l'una rispetto all'altra. Per ogni $a, b, c \in X$:
		$$
		a \lor (b \land c) = (a \lor b) \land (a \lor c) \,, \quad a \land (b \lor c) = (a \land b) \lor (a \land c) \,.
		$$
		
		\item[(b)] \textbf{Limitato} (o \textit{bounded}): se ammette un elemento minimo $\mathbf{0}$ (\textit{bottom}) e un elemento massimo $\mathbf{1}$ (\textit{top}).
		
		\item[(c)] \textbf{Ortocomplementato}: se è limitato ed è dotato di una mappa $X \ni a \mapsto \neg a$, detta \textbf{ortocomplemento} di $a$, tale che:
		\begin{enumerate}
			\item[(i)] $a \lor \neg a = \mathbf{1}$ per ogni $a \in X$;
			\item[(ii)] $a \land \neg a = \mathbf{0}$ per ogni $a \in X$;
			\item[(iii)] $\neg(\neg a) = a$ per ogni $a \in X$ (involutività);
			\item[(iv)] $a \geq b \implies \neg b \geq \neg a$ per ogni $a, b \in X$ (inversione dell'ordine).
		\end{enumerate}
		\item[(d)] \textbf{completo (risp. $\sigma$-completo)}, se ogni insieme (numerabile) $\{a_j\} \subset X$ ammette inf: $\displaystyle\bigwedge_{j\in J} a_j$ e sup: $\displaystyle\bigvee_{j\in J} a_j$.
	\end{itemize}
	Un reticolo con proprietà (a), (b) e (c) è chiamato \textbf{Algebra di Boole}. Un'algebra di Boole che soddisfa (d) con $J = \mathbb{N}$ è una $\sigma$-algebra di Boole.
\end{definizione}

\begin{definizione}[Sottoreticolo]
	Un \textbf{sottoreticolo} è un sottoinsieme $X_0 \subset X$ che eredita la struttura di reticolo da $X$ nel seguente senso preciso: l'estremo inferiore e l'estremo superiore di ogni coppia di elementi di $X_0$ devono esistere e coincidere con i corrispondenti estremo inferiore e superiore calcolati in $X$.
	
	Facendo riferimento a sottoreticoli limitati (\textit{bounded}) o ortocomplementati, si richiede per definizione che:
	\begin{itemize}
		\item L'elemento minimo (\textbf{bottom}) e l'elemento massimo (\textbf{top}) della sottostruttura coincidano con quelli della struttura maggiore;
		\item L'ortocomplemento di un elemento nella sottostruttura coincida con quello definito nella struttura maggiore.
	\end{itemize}
\end{definizione}
In modo banale si può definire omomorfismi e isomorfismi su reticoli.
\begin{osservazione}[Rappresentazione delle Algebre di Boole]
	È naturale chiedersi se un'algebra di Boole astratta corrisponda sempre a un'algebra di insiemi.
	\begin{itemize}
		\item Il \textbf{Teorema di Loomis-Sikorski} garantisce che ogni $\sigma$-algebra di Boole astratta è isomorfa al quoziente $\Sigma / \mathcal{N}$ di una $\sigma$-algebra concreta $\Sigma$ su uno spazio di misura, modulo un ideale $\mathcal{N}$ di insiemi di misura nulla.
		\item Nel caso più semplice di algebre di Boole (non $\sigma$), il \textbf{Teorema di rappresentazione di Stone} prova l'isomorfismo con un'algebra concreta di insiemi.
	\end{itemize}
	Pertanto, il reticolo delle proprietà elementari di un sistema classico è a tutti gli effetti una $\sigma$-algebra di Boole concreta: è distributivo, limitato (con $\mathbf{0}=\varnothing$ e $\mathbf{1}=\Gamma$), ortocomplementato (complemento insiemistico) e $\sigma$-completo. La mappa definita in precedenza, $\mathscr{B}(\mathbb{R}) \ni B \mapsto E_B^{(f)} \in \mathscr{B}(\Gamma)$, è un omomorfismo di $\sigma$-algebre di Boole.
\end{osservazione}

\subsection{La logica non-Booleana della Meccanica Quantistica}

Il quadro classico appena descritto diviene insostenibile per i sistemi quantistici. La ragione profonda risiede nell'esistenza di proprietà elementari \textit{incompatibili}.
Le idee fondamentali introdotte da von Neumann per formalizzare questa nuova logica si basano sui seguenti postulati:

\begin{itemize}
	\item[\textbf{(N1)}] Dato un sistema quantistico, esiste uno spazio di Hilbert complesso e separabile $\Hspace$ tale che le osservabili elementari (quelle che assumono solo valori in $\{0,1\}$) sono rappresentate fedelmente dall'insieme $\mathscr{L}(\Hspace)$ dei \textbf{proiettori ortogonali} su $\Hspace$.
	\item[\textbf{(N2)}] Due osservabili elementari $P, Q$ sono \textit{compatibili} se e solo se i corrispondenti proiettori commutano ($PQ = QP$).
\end{itemize}

\begin{osservazione}
	La richiesta di separabilità in (N1) è utile per molte costruzioni tecniche, ma talvolta può essere omessa o dedotta da requisiti fisici. Inoltre, per sistemi che ammettono \textit{regole di superselezione}, il postulato (N1) richiede delle modifiche, ma per ora ci atteniamo alla versione standard.
\end{osservazione}

L'insieme $\mathscr{L}(\Hspace)$ possiede una naturale struttura d'ordine parziale indotta dall'inclusione dei sottospazi immagine. Scriviamo $P \geq Q$ se e solo se $P(\Hspace) \supseteq Q(\Hspace)$. Dotato di questo ordine, $\mathscr{L}(\Hspace)$ diventa un reticolo con proprietà specifiche molto diverse da quelle classiche.

\begin{proposizione}[Struttura del reticolo quantistico $\mathscr{L}(\Hspace)$]
	Sia $\Hspace$ uno spazio di Hilbert complesso (non necessariamente separabile). Per ogni $P \in \mathscr{L}(\Hspace)$, definiamo l'ortocomplemento $\neg P := \Identity - P$ (il proiettore sul complemento ortogonale $P(\Hspace)^\perp$).
	La struttura $$(\mathscr{L}(\Hspace), \geq, \mathbf{0}, \Identity, \neg)$$ è un reticolo limitato, ortocomplementato e completo (e quindi $\sigma$-completo), che tuttavia \textbf{non è distributivo} se $\dim(\Hspace) \geq 2$.
	
	In dettaglio, date $P, Q \in \mathscr{L}(\Hspace)$ e una famiglia $\{P_j\}_{j \in J} \subset \mathscr{L}(\Hspace)$:
	
	\begin{itemize}
		\item[(i)] Il \textbf{Join} $P \lor Q$ è il proiettore ortogonale sulla somma chiusa dei range:
		$$ P \lor Q = \text{proj su } \overline{P(\Hspace) + Q(\Hspace)} \,.$$
		Analogamente, $\bigvee_{j \in J} P_j$ è il proiettore sullo spazio chiuso generato dall'unione dei range: $\text{span}\{P_j(\Hspace)\}_{j \in J}$.
		
		\item[(ii)] Il \textbf{Meet} $P \land Q$ è il proiettore ortogonale sull'intersezione dei range:
		$$ P \land Q = \text{proj su } P(\Hspace) \cap Q(\Hspace) \,.$$
		Analogamente, $\bigwedge_{j \in J} P_j$ è il proiettore sull'intersezione $\bigcap_{j \in J} P_j(\Hspace)$.
		
		\item[(iii)] Gli elementi minimo (\textit{bottom}) e massimo (\textit{top}) sono rispettivamente l'operatore nullo $0$ e l'identità $\Identity$.
		
		\item[(iv)] Nel caso numerabile ($J = \mathbb{N}$), le operazioni di reticolo sono legate ai limiti nella \textit{topologia forte degli operatori} ($s\text{-}\lim$):
		$$
		\bigvee_{n \in \mathbb{N}} P_n = s\text{-}\lim_{k \to +\infty} \bigvee_{n \le k} P_n \quad \text{e} \quad \bigwedge_{n \in \mathbb{N}} P_n = s\text{-}\lim_{k \to +\infty} \bigwedge_{n \le k} P_n \,.
		$$
	\end{itemize}
\end{proposizione}

\begin{proposizione}[Sottoinsiemi Massimali e Algebre di Boole]
	Sia $\Hspace$ uno spazio di Hilbert complesso e separabile e si consideri il reticolo ortocomplementato e $\sigma$-completo dei proiettori ortogonali $(\mathscr{L}(\Hspace), \geq, \mathbf{0}, \Identity, \neg)$.
	
	Assumiamo che $\mathcal{L}_0 \subset \mathscr{L}(\Hspace)$ sia un sottoinsieme \textit{massimale} di elementi che commutano a due a due (ovvero, se $Q \in \mathscr{L}(\Hspace)$ commuta con ogni $P \in \mathcal{L}_0$, allora $Q \in \mathcal{L}_0$). Valgono i seguenti fatti:
	
	\begin{enumerate}
		\item \textbf{Struttura di Sottoreticolo:}
		\begin{itemize}
			\item[(a)] $\mathcal{L}_0$ contiene l'operatore nullo $\mathbf{0}$ e l'identità $\Identity$;
			\item[(b)] $\mathcal{L}_0$ è chiuso rispetto all'ortocomplemento ($\neg$);
			\item[(c)] L'estremo superiore e l'estremo inferiore di successioni di elementi calcolati in $\mathcal{L}_0$ coincidono con i corrispondenti sup e inf calcolati nell'intero $\mathscr{L}(\Hspace)$.
		\end{itemize}
		Di conseguenza, $\mathcal{L}_0$ è un sottoreticolo di $\mathscr{L}(\Hspace)$.
		
		\item \textbf{Struttura Booleana:} $\mathcal{L}_0$ è anche una $\sigma$-algebra di Boole. Inoltre, per ogni $P, Q \in \mathcal{L}_0$, le operazioni di reticolo assumono la seguente forma algebrica:
		\begin{itemize}
			\item[(i)] $P \lor Q = P + Q - PQ$;
			\item[(ii)] $P \land Q = PQ$.
		\end{itemize}
		
		\item \textbf{Generalizzazione:} Le identità algebriche riportate al punto (2) sono valide più in generale per qualsiasi coppia di proiettori $P, Q \in \mathscr{L}(\Hspace)$, sotto la sola ipotesi che essi commutino ($PQ=QP$).
	\end{enumerate}
\end{proposizione}

\begin{teorema}[Caratterizzazione della $\wedge$ logica]
	In uno spazio di Hilbert $\Hspace$ per ogni $P,Q \in \mathscr{L}(\Hspace)$ e $x \in \Hspace$
	\[
	(P \wedge Q)x = \lim_{n \to +\infty} (PQ)^nx
	\]
\end{teorema}
\begin{osservazione}[Misurazioni Alternate e Convergenza]
	La dimostrazione matematica sui proiettori implica un fatto fisico più forte riguardante le misurazioni consecutive. Si ha infatti che le successioni alternate di proiettori convergono al proiettore sull'intersezione dei sottospazi:
	$$
	P x, \, Q P x, \, P Q P x, \dots \longrightarrow (P \land Q)x \quad \forall x \in \Hspace \,.
	$$
	Poiché $P \land Q = Q \land P$, il risultato vale anche scambiando l'ordine.
	Fisicamente, se assumiamo che lo stato post-misurazione sia descritto dalla proiezione, il lato destro della formula
	$$
	\|(P \land Q)x\|^2 = \lim_{n \to +\infty} \|(PQ)^n x\|^2
	$$
	rappresenta la probabilità che una sequenza infinita di misurazioni alternate di $P$ e $Q$ (anche incompatibili) dia esito positivo a ogni passo, partendo dallo stato $x$.
\end{osservazione}

\subsection{Perché le osservabili sono operatori autoaggiunti}

Siamo ora in grado di chiarire perché, in questo contesto, le osservabili corrispondano alle \textbf{PVM} (Projection-Valued Measures) su $\mathscr{B}(\mathbb{R})$ e, di conseguenza, agli operatori autoaggiunti grazie ai teoremi di integrazione spettrale.

Analogamente al caso classico, un'osservabile $A$ può essere vista come una collezione di osservabili elementari SI-NO $\{P_E\}_{E \in \mathscr{B}(\mathbb{R})}$, etichettate sugli insiemi di Borel di $\mathbb{R}$. Il significato di $P_E$ è:
\begin{equation}
	P_E = \text{``il valore dell'osservabile appartiene a } E \text{''} \,.
\end{equation}

Assumendo che queste osservabili elementari siano \textit{a due a due compatibili} (ipotesi cruciale che permette di lavorare in un sottoinsieme massimale $\mathcal{L}_0$ che si comporta come un'algebra di Boole), ci aspettiamo che soddisfino le proprietà analoghe a quelle classiche (Fi)-(Fiii), tradotte nel linguaggio dei proiettori:

\begin{enumerate}
	\item[(i')] $P_{\mathbb{R}} = \Identity$;
	\item[(ii')] $P_E \land P_F = P_{E \cap F}$;
	\item[(iii')] Se $\{E_k\}_{k \in \mathbb{N}}$ sono disgiunti, allora $\bigvee_{j \in \mathbb{N}} P_{E_j} = P_{\bigcup_{j \in \mathbb{N}} E_j}$.
\end{enumerate}

Tenendo conto delle proprietà dei proiettori compatibili e della convergenza nella topologia forte per le serie infinite, queste relazioni diventano la definizione esatta di una PVM su $\mathbb{R}$:

\begin{enumerate}
	\item[(i)] $P_{\mathbb{R}} = \Identity$;
	\item[(ii)] $P_E P_F = P_{E \cap F}$ (poiché per proiettori commutanti il meet è il prodotto);
	\item[(iii)] Per insiemi disgiunti $\{E_k\}$, vale la $\sigma$-additività nella topologia forte degli operatori:
	$$ \sum_{j \in \mathbb{N}} P_{E_j} x = P_{\bigcup_{j \in \mathbb{N}} E_j} x \quad \text{per ogni } x \in \Hspace \,. $$
\end{enumerate}

\begin{teorema}[Corrispondenza con Operatori Autoaggiunti]
	Le PVM su $\mathbb{R}$ sono associate in modo biunivoco agli operatori autoaggiunti. Integrando la funzione identità $\imath: \mathbb{R} \ni r \mapsto r \in \mathbb{R}$ rispetto alla misura $P$, otteniamo l'operatore normale:
	$$
	A_P = \int_{\mathbb{R}} r \, dP(r) \,.
	$$
	Poiché la funzione integranda è a valori reali, $A_P$ è autoaggiunto. Inoltre, $P$ è l'unica PVM associata ad $A_P$ e il supporto di $P$ coincide con lo spettro $\sigma(A_P)$.
\end{teorema}

Concludiamo che, adottando il framework di von Neumann, le osservabili in Meccanica Quantistica sono naturalmente descritte da operatori autoaggiunti, i cui spettri coincidono con l'insieme dei valori assumibili dalle osservabili stesse.

\subsection{Il Teorema di Soler}

Per affrontare il problema della coordinatizzazione, elenchiamo alcune proprietà speciali del reticolo dei proiettori ortogonali. Poiché la distributività non vale in $\mathscr{L}(\Hspace)$, si introduce la nozione più debole di \textbf{ortomodularità}. Un altro concetto centrale è quello di \textit{atomo}.

\begin{definizione}[Atomo]
	Sia $(\mathscr{L}, \geq, \mathbf{0}, \mathbf{1})$ un reticolo limitato. Un elemento $a \in \mathscr{L} \setminus \{\mathbf{0}\}$ è chiamato \textbf{atomo} se $p \leq a$ implica $p = \mathbf{0}$ oppure $p = a$.
\end{definizione}

Il seguente teorema raccoglie le proprietà rilevanti del reticolo $\mathscr{L}(\Hspace)$, definendo contestualmente le proprietà astratte che possono applicarsi a reticoli ortocomplementati generici.

\begin{teorema}[Proprietà di $\mathscr{L}(\Hspace)$] 
	Nel reticolo ortocomplementato $(\mathscr{L}(\Hspace), \geq, 0, I, \neg)$ dei proiettori ortogonali su uno spazio di Hilbert complesso $\Hspace$, gli unici atomi sono i proiettori su sottospazi monodimensionali.
	Inoltre, $\mathscr{L}(\Hspace)$ soddisfa le seguenti proprietà:
	
	\begin{itemize}
		\item[(i)] \textbf{Separabilità} (per $\Hspace$ separabile): se una famiglia $\{P_j\}_{j \in J} \subset \mathscr{L}(\Hspace) \setminus \{0\}$ è ortogonale (cioè $P_i \leq \neg P_j$ per $i \neq j$), allora l'insieme degli indici $J$ è al più numerabile.
		
		\item[(ii)] \textbf{Atomicita e Atomisticità}:
		\begin{enumerate}
			\item Per ogni $P \in \mathscr{L}(\Hspace) \setminus \{0\}$ esiste un atomo $A$ tale che $A \leq P$ (\textit{atomicità}).
			\item Ogni $P \in \mathscr{L}(\Hspace) \setminus \{0\}$ è il join degli atomi che contiene: $P = \bigvee \{A \leq P \mid A \text{ è un atomo}\}$ (\textit{atomisticità}).
		\end{enumerate}
		
		\item[(iii)] \textbf{Ortomodularità}: La relazione d'ordine implica una forma debole di distributività:
		$$ P \leq Q \implies Q = P \lor ((\neg P) \land Q) \,.$$
		
		\item[(iv)] \textbf{Proprietà di copertura}: Se $A$ è un atomo e $P \in \mathscr{L}(\Hspace)$ tali che $A \land P = 0$, allora valgono specifiche relazioni tra $P$ e $A \lor P$ che regolano l'altezza degli elementi nel reticolo.
		
		\item[(v)] \textbf{Irriducibilità}: Solo gli operatori $0$ e $I$ commutano con ogni elemento di $\mathscr{L}(\Hspace)$.
	\end{itemize}
\end{teorema}

\begin{teorema}[Atomicità $\Rightarrow$ atomisticità]
	In un reticolo ortocomplementato $(\mathscr{L}(\Hspace), \geq, \mathbf{0}, \Identity, \neg)$ ortomodularità e atomicità implica atomisticità.
\end{teorema}

È possibile definire le operazioni di negazione logica e compatibilità direttamente sulla struttura astratta del reticolo, senza fare inizialmente riferimento alla natura operatoriale. Tuttavia, queste definizioni astratte si rivelano equivalenti alle nozioni standard sugli operatori.

\begin{definizione}[Ortogonalità e Commutatività in Reticoli]
	Sia $(\mathscr{L}, \geq, \mathbf{0}, \mathbf{1}, \neg)$ un reticolo ortocomplementato e siano $a, b \in \mathscr{L}$.
	\begin{itemize}
		\item[(a)] $a$ e $b$ si dicono \textbf{ortogonali} (scritto $a \perp b$) se $a \leq \neg b$ (o equivalentemente $b \leq \neg a$).
		\item[(b)] $a$ e $b$ si dicono \textbf{commutanti} se esistono tre elementi a due a due ortogonali $c_1, c_2, c_3$ ($c_i \perp c_j$ per $i \neq j$) tali che:
		$$ a = c_1 \lor c_3 \quad \text{e} \quad b = c_2 \lor c_3 \,.$$
	\end{itemize}
\end{definizione}

\begin{osservazione}
	La definizione astratta di commutatività è equivalente a richiedere che il sottoreticolo ortocomplementato generato da $p$ e $q$ sia Booleano. Questo conferma che la commutatività rappresenta la "compatibilità classica" tra proposizioni.
\end{osservazione}

Queste nozioni astratte recuperano esattamente il loro significato fisico usuale quando applicate al caso concreto dei proiettori.

\begin{proposizione}[Equivalenza con le proprietà degli operatori]
	Sia $\Hspace$ uno spazio di Hilbert e consideriamo $\mathscr{L}(\Hspace)$ come reticolo ortocomplementato. Due elementi $P, Q \in \mathscr{L}(\Hspace)$:
	\begin{itemize}
		\item[(i)] Sono \textbf{ortogonali} nel senso della Definizione precedente se e solo se proiettano su sottospazi mutuamente ortogonali, il che equivale alla condizione operatoriale $PQ = QP = 0$.
		\item[(ii)] \textbf{Commutano} nel senso della Definizione precedente se e solo se commutano come operatori, ovvero $PQ = QP$.
	\end{itemize}
\end{proposizione}

Piron ha dimostrato che, assumendo l'esistenza di almeno 4 atomi a due a due ortogonali, il reticolo delle proposizioni può essere identificato canonicamente con i sottospazi chiusi di uno \textit{spazio di Hilbert generalizzato}, dove il campo scalare è sostituito da un \textbf{anello con divisione} $\mathbb{D}$ (dotato di involuzione).

È stato a lungo congetturato che, aggiungendo ipotesi di separabilità e ortomodularità, l'anello $\mathbb{D}$ si restringesse ai soli casi reali, complessi o quaternionici. La conferma formale è arrivata con il seguente teorema fondamentale.

\begin{teorema}[Teorema di Solèr]
	Consideriamo un reticolo separabile, ortomodulare, $\sigma$-completo, atomico e irriducibile, che soddisfi la proprietà di copertura.
	
	Se tale reticolo ammette un insieme \textbf{infinito} di atomi a due a due ortogonali, allora esso è isomorfo al reticolo dei sottospazi chiusi (ordinati per inclusione insiemistica) di uno spazio di Hilbert separabile definito sul corpo $\mathbb{K}$, dove $\mathbb{K}$ può essere scelto esclusivamente tra: $\R$ $\C$ o $\mathbb{H}$.
\end{teorema}

Come di consueto, il reticolo dei sottospazi chiusi può essere interpretato come il reticolo $\mathscr{L}(\Hspace)$ dei proiettori ortogonali corrispondenti.

\begin{osservazione}[La natura del corpo e l'irriducibilità]
	\begin{itemize}
		\item \textbf{Irriducibilità e Superselezione:} La proprietà di irriducibilità non è strettamente necessaria per la validità fisica. Se fallisce, il reticolo si decompone in sottoreticoli irriducibili. Fisicamente, questa situazione è naturale in presenza di \textit{regole di superselezione}.
		
		\item \textbf{La scelta dei numeri complessi:} Il teorema di Solèr lascia aperta la scelta tra $\mathbb{R}, \mathbb{C}$ e $\mathbb{H}$. Perché la fisica standard usa $\mathbb{C}$? Studi recenti suggeriscono che, per sistemi elementari relativistici, la struttura complessa sia imposta dalla simmetria relativistica stessa. Tuttavia, se non diversamente specificato, nel seguito assumeremo l'uso del campo $\mathbb{C}$ per descrivere i sistemi quantistici.
	\end{itemize}
\end{osservazione}

\begin{osservazione}[Necessità delle ipotesi di Solèr e Spazio-Tempo]
	È fondamentale sottolineare che la validità di \textit{tutte} le proprietà elencate nel Teorema 4.16 (Solèr) è cruciale per ottenere la tesi, ovvero la struttura di spazio di Hilbert.
	Esistono infatti strutture di reticolo fisicamente rilevanti che soddisfano molte di quelle proprietà, come l'ortomodularità, ma falliscono in altre.
	
	Un esempio notevole è la famiglia degli \textbf{insiemi causalmente completi} nello spazio-tempo di Minkowski $M$.
	Definiamo il \textit{complemento causale} $\Sigma^\perp$ di un sottoinsieme di Borel $\Sigma \subset M$ come:
	$$
	\Sigma^\perp := \{ x \in M \mid x \notin \Sigma \text{ e } x, y \text{ sono cronologicamente separati per ogni } y \in \Sigma \} \,.
	$$
	(Ricordiamo che $x, y$ sono cronologicamente separati se $x \neq y$ e non esiste alcun segmento di tipo tempo che li congiunge).
	Una regione $\Delta$ si dice \textbf{causalmente completa} se $\Delta = (\Delta^\perp)^\perp$.
	
	Il reticolo costituito da tutte le regioni causalmente completi di $M$, ordinato tramite l'inclusione insiemistica, risulta essere:
	\begin{itemize}
		\item Un reticolo non-Booleano, atomico, atomistico, irriducibile e ortomodulare;
		\item Dotato di ortocomplemento $\perp$, con minimo $\varnothing$ e massimo $M$.
	\end{itemize}
	Tuttavia, questo reticolo \textbf{non soddisfa la proprietà di copertura} (e anche la separabilità è problematica). Il fallimento della proprietà di copertura impedisce di dotare lo spazio-tempo di una naturale struttura di Hilbert generalizzata. Questo risultato distingue strutturalmente la logica quantistica standard dalla logica causale dello spazio-tempo, offrendo forse spunti per una formulazione della gravità quantistica.
\end{osservazione}

\subsection{Argomenti avanzati su Teoremi di Soler e Piron}

Abbiamo finora stabilito che il reticolo dei proiettori $\mathscr{L}(\Hspace)$ soddisfa proprietà specifiche (completezza, ortomodularità, atomicità, proprietà di copertura). Il problema inverso, noto come \textit{problema della coordinatizzazione}, consiste nel dimostrare che un reticolo astratto $\mathscr{L}$ che soddisfi tali assiomi sia necessariamente isomorfo al reticolo dei sottospazi di uno spazio di Hilbert.

Questo programma di ricerca, avviato da Birkhoff e von Neumann e sviluppato da Jauch, Piron e altri, ha trovato il suo culmine nel Teorema di Solèr (1995).

\subsubsection{Dagli assiomi allo Spazio Vettoriale Generalizzato}

Il primo passo consiste nel recuperare una struttura di spazio vettoriale su un corpo generico.

\begin{teorema}[Piron-Maeda-Maeda]
	Sia $\mathscr{L}$ un reticolo ortocomplementato, completo, irriducibile e atomistico che soddisfa la \textbf{proprietà di copertura}. Supponiamo inoltre che $\mathscr{L}$ contenga almeno 4 atomi a due a due ortogonali.
	
	Allora esistono:
	\begin{enumerate}
		\item Un anello con divisione (corpo) $\mathbb{K}$ dotato di un'involuzione antiautomorfica $\lambda \mapsto \bar{\lambda}$;
		\item Uno spazio vettoriale ``generalizzato'' $E$ su $\mathbb{K}$;
		\item Una forma hermitiana $\langle \cdot | \cdot \rangle : E \times E \to \mathbb{K}$ non singolare (tale cioè che $\langle x | x \rangle = 0 \implies x=0$);
	\end{enumerate}
	tali che $\mathscr{L}$ è isomorfo al reticolo dei sottospazi ``chiusi'' $\mathbf{M} \subset E$ (dove la chiusura è definita algebricamente come $\mathbf{M} = \mathbf{M}^{\perp\perp}$ rispetto alla forma hermitiana), ordinati per inclusione.
	
	Inoltre, $\mathscr{L}$ è ortomodulare se e solo se la forma soddisfa la condizione: $\mathbf{M} + \mathbf{M}^\perp = E$ per ogni sottospazio chiuso $\mathbf{M}$.
\end{teorema}

\subsubsection{Il Teorema di Solèr e la natura del campo}

Il risultato precedente lascia aperta la natura dell'anello $\mathbb{K}$. Molti autori congetturarono che l'aggiunta dell'ipotesi di ortomodularità e di infinitezza dimensionale costringesse $\mathbb{K}$ ad essere uno dei campi classici ($\mathbb{R}, \mathbb{C}, \mathbb{H}$). Questa congettura è stata dimostrata da Solèr (e indipendentemente da Holland per condizioni equivalenti).

\begin{teorema}[Solèr-Holland]
	Consideriamo un reticolo $\mathscr{L}$ che soddisfi tutte le ipotesi del teorema precedente (ortocomplementato, completo, irriducibile, atomistico, con proprietà di copertura e almeno 4 atomi ortogonali) e assumiamo inoltre che sia \textbf{ortomodulare}.
	
	Supponiamo che valga una delle seguenti condizioni equivalenti (che impongono l'infinitezza dimensionale):
	\begin{itemize}
		\item[(a)] \textbf{(Solèr)} Esiste una successione ortonormale infinita $\{e_n\}_{n \in \mathbb{N}}$ (ovvero $\langle e_n | e_n \rangle = 1$);
		\item[(b)] \textbf{(Holland)} Esiste una successione ortogonale infinita con norma fissata;
		\item[(c)] Lo spazio non ha dimensione finita e il reticolo ammette sufficienti simmetrie unitarie tra gli atomi.
	\end{itemize}
	
	Allora valgono le seguenti conclusioni fondamentali:
	\begin{enumerate}
		\item Il corpo $\mathbb{K}$ può essere solo uno tra i \textbf{Reali} $\mathbb{R}$, i \textbf{Complessi} $\mathbb{C}$ o i \textbf{Quaternioni} $\mathbb{H}$.
		\item La forma hermitiana è definita positiva.
		\item Lo spazio $E$ è \textbf{completo} rispetto alla norma indotta, ed è quindi uno \textbf{Spazio di Hilbert} (reale, complesso o quaternionico).
		\item Lo spazio di Hilbert è separabile se e solo se il reticolo $\mathscr{L}$ è separabile.
	\end{enumerate}
\end{teorema}

\begin{osservazione}[Significato delle Ipotesi e Fisica]
	\begin{itemize}
		\item \textbf{Irriducibilità e Superselezione:} L'ipotesi di irriducibilità non è essenziale. Senza di essa, il reticolo si decompone in sottoreticoli irriducibili. In fisica, questo corrisponde alla presenza di \textit{regole di superselezione}, dove lo spazio di Hilbert totale è somma diretta di settori coerenti tra i quali non sono possibili sovrapposizioni.
		
		\item \textbf{Proprietà di Copertura e Gravità Quantistica:} La \textit{covering property} è invece cruciale. Esistono strutture fisicamente rilevanti che la violano. Ad esempio, il reticolo delle regioni \textit{causalmente complete} nello spaziotempo di Minkowski soddisfa quasi tutte le proprietà (inclusa l'ortomodularità) ma non la proprietà di copertura. Questo impedisce di associare allo spaziotempo una struttura di Hilbert naturale, suggerendo che in una teoria di Gravità Quantistica la struttura geometrica dello spaziotempo e quella lineare della meccanica quantistica potrebbero divergere.
		
		\item \textbf{La scelta di $\mathbb{C}$:} Il teorema di Solèr ammette ancora $\mathbb{R}$ e $\mathbb{H}$. Risultati recenti (Oppio et al.) mostrano che se si richiede che il sistema supporti un'azione del gruppo di Poincaré (necessaria per le particelle elementari relativistiche), i casi reale e quaternionico possono essere esclusi, lasciando $\mathbb{C}$ come unica scelta consistente.
	\end{itemize}
\end{osservazione}

\subsection{Il Teorema di Gleason}

Sulla base della struttura di reticolo dei proiettori, possiamo formulare una definizione generale di probabilità in ambito quantistico.

\begin{definizione}[Misura di Probabilità Quantistica]
	Sia $\Hspace$ uno spazio di Hilbert. Una \textbf{misura di probabilità quantistica} su $\Hspace$ è una mappa $\rho: \mathscr{L}(\Hspace) \to [0,1]$ che soddisfa i seguenti requisiti:
	\begin{enumerate}
		\item $\rho(\Identity) = 1$;
		\item $\sigma$-additività su proiettori ortogonali: se $\{P_n\}_{n \in \mathbb{N}} \subset \mathscr{L}(\Hspace)$ è una famiglia tale che $P_h P_k = 0$ per $h \neq k$, allora:
		$$
		\rho\left( \text{s-}\!\sum_{n \in \mathbb{N}} P_n \right) = \sum_{n \in \mathbb{N}} \rho(P_n) \,.
		$$
	\end{enumerate}
	L'insieme convesso delle misure di probabilità quantistiche su $\Hspace$ viene denotato con $\mathscr{M}(\Hspace)$.
\end{definizione}

\begin{osservazione}[Stati e Operatori]
	È fondamentale distinguere tra la nozione matematica astratta di misura di probabilità e quella di \textit{operatore di stato quantistico}.
	Come caso particolare, ogni vettore unitario $\psi \in \Hspace$ definisce una misura $\rho_\psi(P) := \langle \psi | P \psi \rangle$. Poiché $\mathscr{M}(\Hspace)$ è convesso, anche le combinazioni convesse finite $\rho = \sum p_k \rho_{\psi_k}$ sono misure.
	Associando a tale somma l'operatore $T = \sum p_k \ket{\psi_k}\bra{\psi_k}$, si osserva che il valore della probabilità può essere calcolato tramite la traccia:
	$$
	\rho(P) = \text{tr}(TP) \,.
	$$
	Questo suggerisce la definizione della seguente classe di operatori.
\end{osservazione}

\begin{definizione}[Operatori di Stato Quantistico]
	Si definisce l'insieme chiuso e convesso degli \textbf{operatori di stato quantistico} (o matrici densità) $\mathscr{S}(\Hspace)$ come il sottoinsieme degli operatori di classe traccia $\mathfrak{B}_1(\Hspace)$ positivi e con traccia unitaria:
	$$
	\mathscr{S}(\Hspace) := \{ T \in \mathfrak{B}_1(\Hspace) \mid T \geq 0, \, \text{tr}(T) = 1 \} \,.
	$$
\end{definizione}

La relazione tra le due definizioni è formalizzata dalla seguente proposizione.

\begin{proposizione}
	Sia $\Hspace$ uno spazio di Hilbert. Se $T \in \mathscr{S}(\Hspace)$, la mappa
	$$
	\rho_T: \mathscr{L}(\Hspace) \ni P \mapsto \text{tr}(TP)
	$$
	è ben definita e rappresenta una misura di probabilità quantistica, ovvero $\rho_T \in \mathscr{M}(\Hspace)$.
\end{proposizione}

Il risultato più profondo, che stabilisce una corrispondenza biunivoca tra le misure astratte sul reticolo e gli operatori di stato, è il celebre teorema di Gleason.

\begin{teorema}[Teorema di Gleason]
	Sia $\Hspace$ uno spazio di Hilbert di dimensione finita $\neq 2$, oppure di dimensione infinita e separabile.
	L'insieme delle misure di probabilità quantistica $\rho \in \mathscr{M}(\Hspace)$ è in corrispondenza biunivoca con l'insieme degli operatori di stato quantistico $T \in \mathscr{S}(\Hspace)$.
	La biiezione è data dalla formula:
	$$
	\text{tr}(TP) = \rho(P) \quad \text{per ogni } P \in \mathscr{L}(\Hspace) \,.
	$$
	Inoltre, tale corrispondenza preserva la struttura convessa dei due insiemi.
\end{teorema}

\begin{osservazione}[Confronto con la notazione fisica standard]
	È utile notare una differenza di convenzione rispetto ai testi standard di Meccanica Quantistica.
	\begin{itemize}
		\item In questi appunti, $T$ denota l'\textit{operatore di stato} (o matrice densità), mentre $\rho$ indica la misura di probabilità astratta.
		\item Nella letteratura fisica, è l'operatore stesso ad essere solitamente indicato con il simbolo $\rho$ (o $\hat{\rho}$).
	\end{itemize}
	Pertanto, l'uguaglianza $\rho(P) = \text{tr}(TP)$ espressa nel Teorema di Gleason equivale alla formula standard per il calcolo del valore di aspettazione (o probabilità) di un proiettore $P$ su uno stato misto $\hat{\rho}$:
	$$
	\langle P \rangle_{\hat{\rho}} = \text{tr}(\hat{\rho} P) \,.
	$$
	Nel caso particolare in cui il sistema si trovi in uno stato puro descritto dal vettore unitario $\ket{\psi}$, l'operatore di stato diventa il proiettore $T = \ket{\psi}\bra{\psi}$ e la formula restituisce la celebre \textbf{regola di Born}:
	$$
	\text{tr}\big( (\ket{\psi}\bra{\psi}) P \big) = \bra{\psi} P \ket{\psi} \,.
	$$
\end{osservazione}

\begin{osservazione}[Il caso dim $\Hspace = 2$ e altre note]
	\begin{itemize}
		\item[(a)] L'ipotesi $\dim(\Hspace) \neq 2$ nel Teorema di Gleason è fondamentale. Esiste infatti un noto controesempio per $\Hspace = \mathbb{C}^2$.
		
		Identificando i proiettori unidimensionali $P \in \mathscr{L}(\mathbb{C}^2)$ con i vettori unitari $\mathbf{n} = (n_1, n_2, n_3)^t \in \mathbb{R}^3$ sulla sfera di Bloch, possiamo scrivere ogni proiettore come:
		$$
		P_{\mathbf{n}} = \frac{1}{2} \left( \Identity + \sum_{j=1}^3 n_j \sigma_j \right) \,,
		$$
		dove $\sigma_j$ sono le matrici di Pauli standard. Si osserva che due proiettori sono ortogonali, $P_{\mathbf{n}} \perp P_{\mathbf{n}'}$, se e solo se i vettori corrispondenti sono antipodi, ovvero $\mathbf{n} = -\mathbf{n}'$.
		
		Fissiamo ora un vettore unitario $\mathbf{m} \in \mathbb{R}^3$ e definiamo la mappa $\rho: \mathscr{L}(\mathbb{C}^2) \to [0,1]$ come segue:
		$$
		\rho(P_{\mathbf{n}}) := \frac{1}{2} \left( 1 + \sum_{j=1}^3 (n_j m_j)^3 \right) \,.
		$$
		Questa mappa si estende univocamente a una misura di probabilità quantistica. Infatti, verifichiamo l'additività per proiettori ortogonali ($\mathbf{n}$ e $-\mathbf{n}$):
		$$
		\rho(P_{\mathbf{n}}) + \rho(P_{-\mathbf{n}}) = \frac{1}{2} (1 + \Sigma) + \frac{1}{2} (1 - \Sigma) = 1 \,,
		$$
		dato che il termine cubico è dispari, $(-n_j)^3 = -n_j^3$.
		
		Tuttavia, \textbf{non esiste alcun operatore} $T$ (matrice densità) tale che $\text{tr}(T P_{\mathbf{n}}) = \rho(P_{\mathbf{n}})$ per ogni $\mathbf{n}$.
		Se esistesse, imponendo l'uguaglianza avremmo:
		$$
		\text{tr}\left[ T \frac{1}{2} (\Identity + \mathbf{n} \cdot \boldsymbol{\sigma}) \right] = \frac{1}{2} + \frac{1}{2} \sum_{j=1}^3 n_j \text{tr}(T \sigma_j) = \frac{1}{2} + \frac{1}{2} \sum_{j=1}^3 (n_j m_j)^3 \,.
		$$
		Semplificando, si otterrebbe una relazione del tipo:
		$$
		\sum_{j=1}^3 n_j T_j = \sum_{j=1}^3 n_j^3 m_j^3 \,,
		$$
		dove $T_j = \text{tr}(T \sigma_j)$ sono costanti. Questo è impossibile da soddisfare per ogni $\mathbf{n}$, poiché il membro di sinistra dipende \textbf{linearmente} dalle componenti $n_j$, mentre il membro di destra dipende da esse in modo cubico.
		\item[(b)] Il teorema può essere esteso a spazi di Hilbert reali o quaternionici (in accordo con il teorema di Solèr), sebbene con complicazioni tecniche legate alla definizione di traccia.
		\item[(c)] Il teorema di Gleason ha conseguenze cruciali per l'interpretazione della Meccanica Quantistica, in particolare escludendo la possibilità di stati "sharp" (senza dispersione) per tutte le osservabili, anticipando concetti legati ai teoremi di \textit{no-go} per variabili nascoste (Bell, Kochen-Specker).
	\end{itemize}
\end{osservazione}

\subsection{Il Teorema di Gleason Generalizzato}

Il celebre risultato di Gleason, che associa le misure di probabilità quantistica agli operatori di classe traccia (matrici densità), può essere esteso dal caso standard $\mathscr{B}(\Hspace)$ al caso di una generica algebra di von Neumann $\mathfrak{R}$. Tuttavia, è necessario escludere casi patologici (il fattore $I_2$) e considerare forme di additività più forti.

\begin{teorema}[Gleason Generalizzato]
	Sia $\mathfrak{R}$ un'algebra di von Neumann su uno spazio di Hilbert complesso $\Hspace \neq \{\mathbf{0}\}$, la cui decomposizione in tipi non includa alcuna componente di Tipo $I_2$.
	Sia $\mu: \mathscr{L}_{\mathfrak{R}}(\Hspace) \to [0, +\infty]$ una misura sul reticolo dei proiettori tale che $0 < \mu(\Identity) < +\infty$ e che sia \textbf{$\sigma$-additiva} (additiva su successioni numerabili di proiettori ortogonali).
	
	Le seguenti tre condizioni sono equivalenti all'esistenza di un operatore positivo di classe traccia $T \in \mathfrak{B}_1(\Hspace)$ tale che $\mu(P) = \text{tr}(TP)$ per ogni $P \in \mathscr{L}_{\mathfrak{R}}(\Hspace)$:
	
	\begin{itemize}
		\item[(i)] $\mu$ è \textbf{completamente additiva}:
		Per ogni famiglia arbitraria (non necessariamente numerabile) $\{P_j\}_{j \in J} \subset \mathscr{L}_{\mathfrak{R}}(\Hspace)$ di elementi a due a due ortogonali:
		\begin{equation}
			\mu \left( \bigvee_{j \in J} P_j \right) = \sum_{j \in J} \mu(P_j) \,.
		\end{equation}
		
		\item[(ii)] $\mu$ ammette un \textbf{supporto}:
		Esiste un elemento $P \in \mathscr{L}_{\mathfrak{R}}(\Hspace)$ tale che $\mu(Q) = 0$ se e solo se $Q \perp P$ (per ogni $Q \in \mathscr{L}_{\mathfrak{R}}(\Hspace)$).
		
		\item[(iii)] $\mu$ è normale:
		La funzione normalizzata $\frac{1}{\mu(\Identity)}\mu$ è la restrizione al reticolo dei proiettori di uno \textit{stato algebrico normale} sulla $C^*$-algebra $\mathfrak{R}$.
	\end{itemize}
\end{teorema}

\begin{osservazione}[Implicazioni Fisiche e Matematiche]
	Analizziamo le conseguenze di questo teorema:
	
	\begin{enumerate}
		\item \textbf{Il caso Separabile:} Se lo spazio di Hilbert $\Hspace$ è \textbf{separabile}, ogni famiglia di proiettori a due a due ortogonali è al più numerabile. Di conseguenza, la $\sigma$-additività implica automaticamente la completa additività.
		Pertanto, in uno spazio separabile (e senza componenti $I_2$), ogni misura $\sigma$-additiva è automaticamente generata da un operatore di classe traccia.
		\textbf{Attenzione:} A differenza del caso $\mathfrak{R} = \mathscr{B}(\Hspace)$, se l'algebra è un sottoinsieme proprio, l'operatore $T$ che rappresenta $\mu$ non è in generale unico.
		
		\item \textbf{Stati Normali:} Una misura di probabilità $\mu_T$ su un'algebra di von Neumann $\mathfrak{R}$ indotta da un operatore $T$ di traccia unitaria ($\text{tr}(T)=1$) è chiamata \textbf{stato normale} di $\mathfrak{R}$.
		Tale misura è la restrizione al reticolo dei proiettori del funzionale lineare $\omega_T: \mathfrak{R} \to \mathbb{C}$ definito da:
		$$ \omega_T(A) := \text{tr}(TA) \,. $$
		Il concetto di stato normale (che corrisponde alla continuità nella topologia ultra-debole) è centrale nella trattazione algebrica della Meccanica Quantistica e sarà approfondito nel Capitolo 14.
	\end{enumerate}
\end{osservazione}


\subsection{Teorema di Bell debole}

In Meccanica Classica esistono misure di probabilità che assegnano valore 1 a certe proprietà elementari e 0 alle rimanenti (stati deterministici). In Meccanica Quantistica, invece, la natura è intrinsecamente probabilistica: non esistono misure ``sharp'' (senza dispersione). Questo fatto è sancito matematicamente dal seguente teorema, spesso citato come versione matematica del Teorema di Bell (o legato al Teorema di Kochen-Specker).

\begin{teorema}[Teorema di Bell debole]
	Sia $\Hspace$ uno spazio di Hilbert con $\dim(\Hspace) > 2$ (finito o infinito separabile).
	Non esiste alcuna misura di probabilità quantistica $\rho: \mathscr{L}(\Hspace) \to \{0,1\}$ che assuma esclusivamente i valori 0 e 1.
\end{teorema}
\begin{proof}(Sintesi) La dimostrazione sfrutta il Teorema di Gleason. Se tale misura esistesse, sarebbe rappresentata da un operatore $T$. La funzione $f(\psi) = \langle \psi | T \psi \rangle$ mapperebbe la sfera unitaria (che è connessa) nell'insieme discreto $\{0,1\}$. Per continuità, l'immagine dovrebbe essere connessa, quindi costante (tutta 0 o tutta 1). Ma questo contraddice il fatto che $\text{tr}(T)=1$.
\end{proof}

\begin{osservazione}
	Questo risultato implica che non è possibile spiegare la MQ classica in termini di ``variabili nascoste'' non contestuali che assegnino valori definiti a tutte le osservabili simultaneamente.
	Alla luce delle precedenti discussioni, identificheremo d'ora in poi l'insieme delle misure fisicamente rilevanti con l'insieme convesso $\mathscr{S}(\Hspace)$ degli operatori di stato (matrici densità), anche nel caso in cui $\Hspace$ non fosse separabile.
\end{osservazione}

\subsubsection{Stati Puri e Misti: Struttura convessa}

Siamo ora in grado di analizzare la geometria dell'insieme degli stati $\mathscr{S}(\Hspace)$, introducendo la distinzione fondamentale per i fisici tra stati puri e misti.
Ricordiamo che, dato un insieme convesso $C$ in uno spazio vettoriale, un punto $e \in C$ si dice \textbf{estremale} se non può essere scritto come combinazione convessa non banale di altri punti (cioè $e = \lambda x + (1-\lambda)y$ implica $x=y=e$).

\begin{proposizione}[Struttura di $\mathscr{S}(\Hspace)$]
	Sia $\Hspace$ uno spazio di Hilbert.
	
	\begin{itemize}
		\item[(a)] \textbf{Stati Puri (Punti Estremali):} I punti estremali dell'insieme convesso $\mathscr{S}(\Hspace)$ sono esattamente i proiettori di rango 1, ovvero gli operatori della forma:
		$$ P_\psi = \ket{\psi}\bra{\psi} \quad \text{con } \|\psi\|=1 \,. $$
		Esiste quindi una biiezione tra i punti estremali e i raggi dello spazio proiettivo complesso $\mathbb{P}\Hspace$. In fisica, questi corrispondono agli \textbf{stati puri}. Gli indicheremo con $\mathscr{S}(\Hspace)_p$
		
		\item[(b)] \textbf{Stati Misti (Decomposizione):} Ogni operatore di stato $T \in \mathscr{S}(\Hspace)$ è una combinazione lineare convessa (eventualmente infinita) di stati puri. In particolare, esiste sempre una decomposizione spettrale:
		$$
		T = \sum_{u \in M} p_u \ket{u}\bra{u} \,,
		$$
		dove $M$ è una base hilbertiana di autovettori di $T$, i coefficienti $p_u \in [0,1]$ soddisfano $\sum p_u = 1$, e la serie converge nella norma della classe traccia $\|\cdot\|_1$ (e quindi anche uniformemente). Tali stati non estremali sono detti \textbf{stati misti}.
	\end{itemize}
\end{proposizione}

\begin{osservazione}[Sulla dimostrazione]
	La parte (b) è una conseguenza diretta del fatto che gli operatori di classe traccia sono compatti e autoaggiunti (essendo positivi), ammettendo quindi una base di autovettori. La condizione $\text{tr}(T)=1$ assicura che la somma degli autovalori sia unitaria, definendo una distribuzione di probabilità classica $\{p_u\}$ sugli stati puri $\ket{u}$.
\end{osservazione}
\subsection{Tipologie di Sovrapposizione e Stati Misti}

In questa sezione approfondiamo la natura degli stati non puri e le diverse modalità con cui gli stati quantistici possono combinarsi.

\begin{definizione}[Stati Misti e Matrici Densità]
	Gli operatori di stato che non sono estremali (cioè non sono proiettori di rango 1 della forma $\ket{\psi}\bra{\psi}$) sono chiamati \textbf{operatori statistici} o \textbf{matrici densità}. In fisica, si dice che essi descrivono \textbf{stati misti}, miscele o stati non puri.
\end{definizione}

È fondamentale distinguere due modi in cui gli stati possono essere "sommati":

\begin{enumerate}
	\item \textbf{Sovrapposizione Coerente (Somma vettoriale):}
	Dati dei vettori $\phi_i \in \Hspace$, consideriamo il vettore somma:
	$$ \psi = \sum_{i \in I} a_i \phi_i $$
	con la serie convergente nella topologia di $\Hspace$ se l'insieme $I$ è infinito. Lo stato puro $T_\psi = \ket{\psi}\bra{\psi}$ generato da questo vettore è detto \textit{sovrapposizione coerente} degli stati puri associati ai $\phi_i$. Questo è il contenuto del \textbf{Principio di Sovrapposizione} degli stati puri.
	
	\item \textbf{Sovrapposizione Incoerente (Somma convessa di operatori):}
	Se consideriamo una famiglia di operatori di stato $\{T_i\}_{i \in I}$ e dei pesi $p_i \in [0,1]$ tali che $\sum p_i = 1$, definiamo:
	$$ T = \sum_{i \in I} p_i T_i \,.$$
	In questo caso, l'operatore $T$ descrive una \textit{sovrapposizione incoerente} (o miscela statistica). Qui non si sommano le ampiezze di probabilità, ma le probabilità classiche di trovare il sistema in uno degli stati $T_i$.
\end{enumerate}

\begin{definizione}[Ampiezza e Probabilità di Transizione]
	Siano $\psi, \phi \in \Hspace$ vettori unitari.
	\begin{itemize}
		\item Il numero complesso $\braket{\psi}{\phi}$ è detto \textbf{ampiezza di transizione} (o ampiezza di probabilità).
		\item Il numero reale non negativo $|\braket{\psi}{\phi}|^2$ è la \textbf{probabilità di transizione} dello stato $\ket{\phi}\bra{\phi}$ sullo stato $\ket{\psi}\bra{\psi}$.
	\end{itemize}
	Interpretiamo $|\braket{\psi}{\phi}|^2 = \text{tr}(T_\psi P_\phi)$ come la probabilità che il sistema, preparato nello stato $\psi$, venga trovato nello stato $\phi$ a seguito di una misurazione (al tempo $t$). Si noti la simmetria $|\braket{\psi}{\phi}|^2 = |\braket{\phi}{\psi}|^2$.
\end{definizione}

\subsubsection{L'origine dell'Interferenza Quantistica}

La differenza fisica tra sovrapposizione coerente e incoerente diventa evidente calcolando le probabilità dei risultati di una misura.
Consideriamo due vettori unitari $\psi, \phi \in \Hspace$ e costruiamo due stati diversi basati su di essi.

\begin{osservazione}[Analisi comparata delle probabilità]
	Siano $a, b \in \mathbb{C}$ coefficienti tali che $|a|^2 + |b|^2 = 1$.
	
	1. \textbf{Caso Coerente (Stato Puro):} Definiamo il vettore $\chi := a\psi + b\phi$. Lo stato è $T_\chi = \ket{\chi}\bra{\chi}$.
	La probabilità che una proposizione elementare $Q \in \mathscr{L}(\Hspace)$ sia vera in questo stato è data da:
	\begin{align*}
		\text{tr}(T_\chi Q) = \bra{\chi} Q \ket{\chi} &= \bra{a\psi + b\phi} Q \ket{a\psi + b\phi} \\
		&= |a|^2 \bra{\psi}Q\ket{\psi} + |b|^2 \bra{\phi}Q\ket{\phi} + \underbrace{\bar{a}b \bra{\psi}Q\ket{\phi} + \bar{b}a \bra{\phi}Q\ket{\psi}}_{\text{Termini misti}} \,.
	\end{align*}
	
	2. \textbf{Caso Incoerente (Stato Misto):} Usiamo gli stessi pesi $|a|^2$ e $|b|^2$ per costruire una miscela statistica:
	$$ T := |a|^2 T_\psi + |b|^2 T_\phi = |a|^2 \ket{\psi}\bra{\psi} + |b|^2 \ket{\phi}\bra{\phi} \,.$$
	La probabilità che $Q$ sia vera in questo stato è semplicemente la media pesata delle probabilità:
	$$ \text{tr}(T Q) = |a|^2 \bra{\psi}Q\ket{\psi} + |b|^2 \bra{\phi}Q\ket{\phi} \,.$$
\end{osservazione}

La differenza tra le due probabilità (4.23 e 4.24 nel testo) è esattamente il \textbf{Termine di Interferenza Quantistica}:
$$
r(T_\chi Q) - \text{tr}(TQ) = \bar{a}b \braket{\psi}{Q\phi} + \bar{b}a \braket{\phi}{Q\psi} \,.
$$
Questo termine (che può essere positivo o negativo) è assente nella miscela statistica classica. È proprio questo termine a spiegare fenomeni sperimentali come l'interferenza degli elettroni nell'esperimento della doppia fenditura: la sovrapposizione coerente permette alle "traiettorie" (o meglio, alle ampiezze) di interferire costruttivamente o distruttivamente, mentre la miscela incoerente somma semplicemente le intensità.

\subsection{Stati post-misurazione: una critica al postulata del collasso}

Dopo aver ridefinito lo stato in termini di operatori di classe traccia $\mathscr{S}(\Hspace)$, è necessario aggiornare l'assioma relativo al collasso della funzione d'onda. La formulazione standard (introdotta da von Neumann e generalizzata da Lüders) descrive cosa accade al sistema fisico, in uno stato $T \in \mathscr{S}(\Hspace)$, quando viene sottoposto alla misurazione di un'osservabile elementare $P \in \mathscr{L}(\Hspace)$ (assumendo che il risultato sia positivo, $\text{tr}(TP)>0$).

Ci riferiamo qui a \textit{misurazioni non distruttive} (o di prima specie), dove il sistema esaminato (tipicamente una particella) non viene assorbito o annichilito dallo strumento.

\begin{definizione}[Postulato di proiezione generale]
	Se un sistema quantistico si trova nello stato descritto dall'operatore statistico $T \in \mathscr{S}(\Hspace)$ al tempo $t_0$, e la proposizione $P \in \mathscr{L}(\Hspace)$ risulta vera dopo una misurazione, lo stato del sistema immediatamente dopo la misura è descritto dall'operatore:
	\begin{equation}
		T_P := \frac{PTP}{\text{tr}(TP)} \,.
	\end{equation}
	Il termine al denominatore è semplicemente la probabilità che l'evento $P$ si verifichi nello stato $T$, necessaria per rinormalizzare la traccia a 1.
\end{definizione}

\begin{osservazione}[Caso dello stato puro]
	Se lo stato iniziale è puro, ovvero $T = \ket{\psi}\bra{\psi}$ per un vettore unitario $\psi$, la formula generale si riduce alla nota proiezione del vettore di stato. Lo stato post-misura è ancora puro ed è determinato dal vettore:
	$$
	\psi_P = \frac{P\psi}{\|P\psi\|} \,.
	$$
\end{osservazione}

\begin{osservazione}[Misurazione di Osservabili e Spettro]
	Il postulato si applica concretamente alla misurazione di un'osservabile $A$ con misura spettrale $P^{(A)}$. Negli esperimenti reali, si testa una famiglia di proposizioni elementari mutuamente esclusive $\{P_{E_j}^{(A)}\}$, costruite su una partizione dello spettro $\sigma(A)$. L'ampiezza di questa partizione corrisponde alla sensibilità (risoluzione) dello strumento di misura. Misurare l'osservabile equivale a testare simultaneamente quale di queste proposizioni è vera.
\end{osservazione}

Il postulato di proiezione sopra enunciato non è arbitrario, ma possiede un'importante caratterizzazione legata al concetto di probabilità condizionata.

Supponiamo di misurare $P$ (ottenendo successo) e di volerci chiedere, subito dopo, qual è la probabilità di misurare una proposizione $Q$.
In generale, se $P$ e $Q$ non commutano, la logica quantistica rende il problema della probabilità condizionata molto complesso e diverso dal caso classico.

Tuttavia, se ci restringiamo al caso di proposizioni compatibili, in particolare quando $Q \leq P$ (ovvero il sottospazio di $Q$ è contenuto in quello di $P$, il che implica $Q$ implica $P$), ci aspettiamo che valgano le intuizioni classiche. In questo caso $P \land Q = Q$, quindi la regola classica di Bayes suggerisce:
$$
\mathbb{P}_T(Q|P) = \frac{\mathbb{P}_T(Q \cap P)}{\mathbb{P}_T(P)} = \frac{\text{tr}(TQ)}{\text{tr}(TP)} \,.
$$
Se imponiamo che questa relazione debba valere per lo stato post-misurazione $T'$, otteniamo una caratterizzazione univoca del postulato di collasso.

\begin{proposizione}[Caratterizzazione di Lüders]
	Sia $T \in \mathscr{S}(\Hspace)$ uno stato quantistico e supponiamo che per $P \in \mathscr{L}(\Hspace)$ si abbia $\text{tr}(TP) > 0$.
	Esiste \textbf{esattamente un} altro operatore di stato $T' \in \mathscr{S}(\Hspace)$ tale che:
	\begin{equation}
		\text{tr}(T'Q) = \frac{\text{tr}(TQ)}{\text{tr}(TP)} \quad \text{per ogni } Q \in \mathscr{L}(\Hspace) \text{ con } Q \leq P \,.
	\end{equation}
	Tale operatore è esattamente quello fornito dal postulato di Lüders-von Neumann:
	$$
	T' = \frac{PTP}{\text{tr}(TP)} \,.
	$$
\end{proposizione}

\begin{osservazione}[Critica Fisica]
	Contrariamente a quanto potrebbe sembrare, l'ipotesi che giustifica la proposizione precedente (ovvero la validità della formula classica per la probabilità condizionata quando $Q \leq P$) non è fisicamente banale.
	Essa si basa sull'idea che una misurazione consecutiva di $P$ e poi di $Q$ sia equivalente a una singola misurazione simultanea. Questa assunzione idealizza il processo, trascurando il fatto che due misurazioni in sequenza potrebbero coinvolgere strumenti e azioni fisiche differenti rispetto a una misura congiunta.
\end{osservazione}

\subsubsection{Confronto tra i postulati di von Neumann e Lüders}

Sebbene il postulato di proiezione sembri univoco, storicamente esistono due formulazioni che differiscono nel modo di trattare gli autospazi degeneri (con dimensione $>1$).

\begin{osservazione}[Differenza tra von Neumann e Lüders]
	Consideriamo un'osservabile $A$ con spettro discreto e decomposizione spettrale $A = \sum_\lambda \lambda P^{(\lambda)}$, dove $P^{(\lambda)}$ è il proiettore sull'autospazio relativo all'autovalore $\lambda$, avente dimensione $d_\lambda$.
	
	\begin{enumerate}
		\item \textbf{Approccio di von Neumann:} L'idea originale prevedeva che la misura di $A$ proiettasse lo stato sugli autovettori di una base fissata $\{ \psi_j^{(\lambda)} \}_{j=1}^{d_\lambda}$ dell'autospazio. Se il risultato è $\lambda$, lo stato post-misura diventa una miscela statistica (incoerente) delle proiezioni sugli assi della base:
		\begin{equation}
			T'_{vN} := \sum_{j=1}^{d_\lambda} P_j^{(\lambda)} T P_j^{(\lambda)} \,,
		\end{equation}
		dove $P_j^{(\lambda)} = \ket{\psi_j^{(\lambda)}}\bra{\psi_j^{(\lambda)}}$. Questo processo distrugge eventuali coerenze (sovrapposizioni) interne all'autospazio degenere.
		
		\item \textbf{Approccio di Lüders:} Il postulato moderno (che abbiamo adottato nella formula 4.25) prevede invece una proiezione sull'intero autospazio $P^{(\lambda)}$. Lo stato post-misura è:
		\begin{equation}
			T'_{L} := \frac{P^{(\lambda)} T P^{(\lambda)}}{\text{tr}(P^{(\lambda)} T P^{(\lambda)})} \,.
		\end{equation}
		Questo approccio preserva la coerenza quantistica all'interno dell'autospazio degenere (principio di minima perturbazione).
	\end{enumerate}
	
	Le due prescrizioni coincidono se e solo se lo spettro è non degenere ($d_\lambda = 1$). Tuttavia, notiamo che le probabilità dei risultati calcolate con i due metodi sono identiche: $\text{tr}(T'_{vN}) = \text{tr}(T'_{L}) = 1$.
\end{osservazione}

\subsubsection{Dipendenza dello stato dalla procedura di misura}

Un aspetto sottile ma fondamentale è che lo stato post-misura dipende strettamente da \textit{cosa} è stato effettivamente misurato e da \textit{come} l'informazione è stata raccolta. Non basta sapere "quale valore" ha assunto una grandezza, ma bisogna sapere se lo strumento era in grado di distinguere valori più fini.

\begin{esempio}[Misura di funzioni di osservabili]
	Consideriamo un sistema preparato in uno stato $\rho$ (usiamo $\rho$ per brevità al posto di $T$) e un'osservabile $A$ con spettro puntuale $\sigma(A) = \{\lambda_1, \lambda_2, \lambda_3, \lambda_4\}$. Supponiamo che i proiettori spettrali $P_i := P_{\lambda_i}$ siano unidimensionali.
	
	Definiamo una funzione $f(A)$ tale che:
	$$ f(\lambda_1) = f(\lambda_2) = 1 \quad \text{e} \quad f(\lambda_3) = f(\lambda_4) = -1 \,.$$
	Lo spettro dell'osservabile $f(A)$ è quindi $\sigma(f(A)) = \{1, -1\}$.
	
	Analizziamo due procedure diverse per ottenere il risultato "1":
	
	\begin{enumerate}
		\item \textbf{Misura diretta di $f(A)$:} Usiamo uno strumento che distingue solo tra il valore $1$ e $-1$. Se otteniamo $1$, secondo Lüders proiettiamo sul proiettore somma $P_{1,2} = P_1 + P_2$. Lo stato collassa in:
		$$ \rho' = \frac{(P_1 + P_2) \rho (P_1 + P_2)}{\text{tr}((P_1 + P_2)\rho)} \,. $$
		In questo stato, la \textit{coerenza} (i termini di interferenza) tra i sottospazi associati a $\lambda_1$ e $\lambda_2$ viene preservata. Se $\rho$ era uno stato puro, $\rho'$ rimane puro (ma confinato nel sottospazio $1,2$).
		
		\item \textbf{Misura di $A$ con "dimenticanza":} Misuriamo l'osservabile $A$ completa, distinguendo $\lambda_1$ e $\lambda_2$, ma successivamente raggruppiamo i dati (o ignoriamo la distinzione) dicendo solo "è uscito 1".
		Fisicamente, lo strumento ha interagito distinguendo i singoli autostati. Lo stato risultante è una miscela statistica dei risultati possibili pesati con le loro probabilità condizionate:
		$$ \rho'' = \frac{P_1 \rho P_1 + P_2 \rho P_2}{\text{tr}((P_1 + P_2)\rho)} \,. $$
		Qui i termini incrociati (tipo $P_1 \rho P_2$) sono spariti. $\rho''$ è una miscela incoerente (stato misto), anche se $\rho$ era puro. Questo è un caso di probabilità \textbf{epistemica} ossia derivante da un'ignoranza di informazioni sul sistema. L'osservatore che ignora la distinzione dei risultati ha un'informazione sullo stato del sistema che non è totale, anche se non può saperlo. Ciò non è sintomo di un'interpretazione soggettivistica della meccanica quantistica, quanto suggerisce che lo stato è "oggettivo" tra osservatori che hanno la stessa conoscenza sul sistema.
	\end{enumerate}
	In conclusione, $\rho' \neq \rho''$. La probabilità di ottenere il risultato $1$ è la stessa, ma lo stato fisico finale è diverso. Questo dimostra che misurare $f(A)$ non è equivalente a misurare $A$ e poi calcolare la funzione, se ci interessa lo stato successivo del sistema.
\end{esempio}

\begin{esempio}[Esistono stati rappresentati da matrici densità che sono "intrinsecamente" probabilistiche?]
	Sì, si prenda una coppia entangled. Se ho accesso solo ad un sottosistema e non all'altro, lo stato di esso è rappresentato da una matrice densità intrinsecamente probabilistica.
\end{esempio}

\subsubsection{Misurazioni Non Selettive (Ideali)}

Il caso (2) dell'esempio precedente introduce il concetto di \textbf{misura non selettiva}.
Una misura selettiva filtra il sistema in base al risultato (come nell'equazione 4.25). Una misura non selettiva avviene quando l'apparato interagisce con il sistema misurando un'osservabile $A$, ma l'osservatore non legge il risultato o raccoglie tutti i sistemi uscenti in un unico ensemble.

Se $\sigma(A) = \bigcup_{j \in \mathbb{N}} E_j$ è una decomposizione in insiemi disgiunti di risultati e misuriamo $A$ senza selezionare l'uscita, lo stato finale $T'$ è la sovrapposizione incoerente di tutti i possibili stati post-misura pesati con la loro probabilità.

\begin{definizione}[Misura Non Selettiva]
	Dato uno stato iniziale $T \in \mathscr{S}(\Hspace)$ e una misura spettrale $\{P_{E_j}^{(A)}\}$, lo stato dopo una misura non selettiva è:
	\begin{equation}
		T' = \sum_{j \in \mathbb{N}} P_{E_j}^{(A)} T P_{E_j}^{(A)} \,. \label{eq:non_selective}
	\end{equation}
	Ponendo $p_j = \text{tr}(T P_{E_j}^{(A)})$ e definendo gli stati condizionati normalizzati $T_j$, possiamo scrivere:
	$$ T' = \sum_{j \in \mathbb{N}} p_j T_j \,. $$
\end{definizione}

\begin{osservazione}
	È importante notare due aspetti:
	\begin{itemize}
		\item La mappa $T \mapsto T'$ descritta dalla (\ref{eq:non_selective}) è \textbf{lineare}. Al contrario, la mappa della misura selettiva (il collasso standard) non è lineare a causa del fattore di normalizzazione al denominatore che dipende da $T$.
		\item Questo formalismo giustifica la creazione di stati misti a partire da stati puri mediante l'interazione con un apparato di misura (decoerenza). Una volta creato uno stato misto in questo modo, la meccanica quantistica non offre alcun modo per distinguere tramite esperimenti diretti se tale miscela provenga da una misura non selettiva su un autovettore di $A$ o da un'altra preparazione che ha generato la stessa matrice densità (ambiguità della decomposizione dell'ensemble).
	\end{itemize}
\end{osservazione}

\subsubsection{Struttura dei proiettori in un algebra di Von Neumann}

Per concludere la disamina delle proprietà elementari, è fondamentale analizzare la struttura formata dai proiettori ortogonali appartenenti a un'algebra di von Neumann. In meccanica quantistica, questi proiettori corrispondono alle "proposizioni" (domande sì/no) sul sistema fisico.

Sia $\mathfrak{R}$ un'algebra di von Neumann su uno spazio di Hilbert $\Hspace$. L'intersezione tra l'algebra e l'insieme di tutti i proiettori ortogonali $\mathscr{L}(\Hspace)$ eredita una struttura di reticolo:
$$ \mathscr{L}_{\mathfrak{R}}(\Hspace) := \mathfrak{R} \cap \mathscr{L}(\Hspace) \,. $$

Questo insieme eredita le operazioni di \textit{meet} ($\wedge$) e \textit{join} ($\vee$) e l'ortocomplemento ($\neg$) da $\mathscr{L}(\Hspace)$, ma è necessario garantire che il risultato di queste operazioni rimanga dentro l'algebra $\mathfrak{R}$.

\begin{proposizione}[Operazioni nel Reticolo]
	Siano $P, Q \in \mathscr{L}_{\mathfrak{R}}(\Hspace)$. Poiché $\mathfrak{R}$ è chiusa nella topologia forte e chiusa rispetto al prodotto, valgono le seguenti proprietà costruttive:
	\begin{enumerate}
		\item \textbf{Intersezione (Meet):} L'operatore $P \wedge Q$ (proiezione sul sottospazio $P\Hspace \cap Q\Hspace$) appartiene a $\mathfrak{R}$ ed è dato dal limite forte:
		$$ P \wedge Q = \operatorname{s-lim}_{n \to +\infty} (PQ)^n \,. $$
		\item \textbf{Unione (Join):} L'operatore $P \vee Q$ (proiezione sulla chiusura di $P\Hspace + Q\Hspace$) appartiene a $\mathfrak{R}$ e si ottiene tramite le leggi di De Morgan:
		$$ P \vee Q = \neg((\neg P) \wedge (\neg Q)) = I - \left( \operatorname{s-lim}_{n \to +\infty} [(I-P)(I-Q)]^n \right) \,. $$
		\item \textbf{Struttura:} La tripla $(\mathscr{L}_{\mathfrak{R}}(\Hspace), \ge, 0, I, \neg)$ è un reticolo limitato, ortocomplementato e \textbf{$\sigma$-completo} (poiché la $\sigma$-completezza coinvolge solo la topologia forte e l'algebra è fortemente chiusa).
		\item \textbf{Ortomodularità:} Il reticolo è ortomodulare, una proprietà cruciale per l'interpretazione fisica, sebbene proprietà più sottili come l'irriducibilità o l'atomicità non siano garantite a priori.
	\end{enumerate}
\end{proposizione}

Una delle caratteristiche più sorprendenti delle algebre di von Neumann è che la struttura "statica" delle proposizioni (il reticolo) contiene tutta l'informazione dinamica dell'algebra.

\begin{proposizione}[Generazione dell'Algebra, Prop 6.14]
	Sia $\mathfrak{R}$ un'algebra di von Neumann. L'algebra generata dal suo reticolo dei proiettori coincide con l'algebra stessa:
	$$ (\mathscr{L}_{\mathfrak{R}}(\Hspace))'' = \mathfrak{R} \,. $$
	In altre parole, ogni operatore in $\mathfrak{R}$ può essere ricostruito (tramite combinazioni lineari e limiti forti) dai suoi proiettori. Inoltre, il centro del reticolo genera il centro dell'algebra.
\end{proposizione}

\subsubsection{Fattori e Classificazione di Murray-von Neumann}

La nozione di \textbf{Fattore} gioca un ruolo distinto nella teoria. Come introdotto nella Definizione 6.15, un fattore è un'algebra di von Neumann con centro banale ($\mathfrak{R} \cap \mathfrak{R}' = \mathbb{C}I$).

Esiste un legame diretto tra la banalità del centro e la struttura del reticolo delle proposizioni:

\begin{proposizione}[Caratterizzazione dei Fattori, Prop 6.16]
	Un'algebra di von Neumann $\mathfrak{R}$ è un fattore se e solo se il reticolo associato $\mathscr{L}_{\mathfrak{R}}(\Hspace)$ è \textbf{irriducibile}.
\end{proposizione}

Sugli spazi di Hilbert separabili, ogni algebra di von Neumann può essere decomposta in una somma diretta (o integrale diretto) di fattori. Pertanto, la classificazione delle algebre si riduce alla classificazione dei fattori. Murray e von Neumann hanno sviluppato una classificazione basata sul confronto delle "dimensioni" relative dei proiettori.

\begin{definizione}[Equivalenza di Proiettori]
	Siano $P, Q \in \mathfrak{R}$ due proiettori. Si dicono \textbf{equivalenti} ($P \sim Q$) se esiste un'isometria parziale $U \in \mathfrak{R}$ tale che:
	$$ U^* U = P \quad \text{e} \quad U U^* = Q \,. $$
	Geometricamente, questo significa che i sottospazi corrispondenti sono isometrici tramite un'operazione interna all'algebra.
\end{definizione}

\begin{definizione}[Finitezza]
	Un proiettore $P \in \mathscr{L}_{\mathfrak{R}}(\Hspace)$ si dice \textbf{finito} se non è equivalente a nessun suo sottoproiettore proprio (ovvero, se $Q \le P$, $Q \ne P$ implica $P \not\sim Q$).
	L'algebra (o il fattore) si dice finita se l'operatore identità $I$ è un proiettore finito; altrimenti si dice \textbf{infinita}.
\end{definizione}

Basandosi sulla presenza di proiettori minimali (atomi) e sulla proprietà di finitezza, i fattori si classificano in tre Tipi:

\begin{enumerate}
	\item \textbf{Tipo I (Meccanica Quantistica Standard):}
	Il fattore contiene proiettori minimali (atomi).
	\begin{itemize}
		\item Un fattore è di Tipo I se e solo se è isomorfo a $\mathscr{B}(\Hspace_1)$ per qualche spazio di Hilbert $\Hspace_1$.
		\item Il reticolo è atomico e soddisfa la proprietà di ricoprimento.
	\end{itemize}
	
	\item \textbf{Tipo II (Geometria Continua):}
	Il fattore \textbf{non} contiene proiettori minimali (non ci sono atomi), ma esistono proiettori finiti non nulli.
	\begin{itemize}
		\item \textbf{Tipo II$_1$:} L'operatore identità $I$ è finito. Questo definisce una "geometria continua" dove le dimensioni sono numeri reali nell'intervallo $[0,1]$.
		\item \textbf{Tipo II$_\infty$:} L'identità è infinita (simile al prodotto tensoriale di un Tipo II$_1$ con $\mathscr{B}(\Hspace)$ infinito-dimensionale).
	\end{itemize}
	
	\item \textbf{Tipo III (Teoria Quantistica dei Campi):}
	Il fattore non contiene proiettori minimali e \textbf{non} contiene nessun proiettore finito non nullo (eccetto lo 0).
	\begin{itemize}
		\item Tutti i proiettori non nulli sono infiniti.
		\item Questa classe, inizialmente considerata patologica, è fondamentale nella descrizione di sistemi termodinamici estesi e nella \textbf{Teoria Quantistica dei Campi Algebrica (AQFT)}.
		\item Un risultato celebre stabilisce che, sotto ipotesi standard, l'algebra delle osservabili localizzate in una regione limitata dello spaziotempo di Minkowski è isomorfa all'unico \textit{fattore iperfinito di Tipo III$_1$}.
	\end{itemize}
\end{enumerate}

\subsection{Irriducibilità e Lemma di Schur nelle Algebre di von Neumann}

Un potente strumento tecnico, essenziale sia per la teoria matematica che per le applicazioni fisiche (come le regole di superselezione), riguarda il concetto di irriducibilità di un insieme di operatori.

\begin{definizione}[Invarianza e Irriducibilità Topologica]
	Sia $\Hspace \neq \{0\}$ uno spazio di Hilbert e $\mathfrak{M} \subset \mathscr{B}(\Hspace)$ una collezione di operatori.
	\begin{itemize}
		\item[(a)] Un sottospazio chiuso $\Hspace_0 \subset \Hspace$ si dice \textbf{invariante} per $\mathfrak{M}$ (o $\mathfrak{M}$-invariante) se $A(\Hspace_0) \subset \Hspace_0$ per ogni $A \in \mathfrak{M}$.
		\item[(b)] L'insieme $\mathfrak{M}$ si dice \textbf{topologicamente irriducibile} (o semplicemente irriducibile) se gli unici sottospazi chiusi $\mathfrak{M}$-invarianti sono quelli banali: $\{0\}$ e $\Hspace$.
	\end{itemize}
\end{definizione}

Il legame tra irriducibilità e la struttura del commutante è sancito dalla versione per spazi di Hilbert del celebre Lemma di Schur.

\begin{teorema}[Lemma di Schur]
	Sia $\Hspace \neq \{0\}$ uno spazio di Hilbert e $\mathfrak{M} \subset \mathscr{B}(\Hspace)$ un insieme *-chiuso. I seguenti fatti sono equivalenti:
	\begin{itemize}
		\item[(a)] $\mathfrak{M}$ è irriducibile.
		\item[(b)] Il commutante di $\mathfrak{M}$ è banale (multipli dell'identità): $\mathfrak{M}' = \mathbb{C}I$.
		\item[(c)] L'algebra di von Neumann generata coincide con l'intero insieme degli operatori limitati: $\mathfrak{M}'' = \mathscr{B}(\Hspace)$.
	\end{itemize}
\end{teorema}

Questo risultato ha un'applicazione immediata nella teoria delle rappresentazioni dei gruppi, fondamentale in fisica quantistica.

\begin{corollario}[Rappresentazioni Unitarie]
	Sia $\pi: G \to \mathscr{B}(\Hspace)$ una rappresentazione unitaria di un gruppo $G$ (oppure di una *-algebra $\mathfrak{A}$) su $\Hspace \neq \{0\}$. Se $G$ (o $\mathfrak{A}$) è abeliano, l'immagine della rappresentazione è irriducibile se e solo se $\dim(\Hspace) = 1$.
\end{corollario}

\subsubsection{Sottospazi Riducenti per Operatori Illimitati}

Quando si trattano osservabili fisiche non limitate, la nozione di invarianza va maneggiata con cura a causa dei domini. Si introduce quindi il concetto più forte di \textit{riduzione}.

\begin{definizione}[Sottospazio Riducente]
	Sia $T: D(T) \to \Hspace$ un operatore (anche illimitato) e $\Hspace_0 \subset \Hspace$ un sottospazio chiuso. Siano $P_0$ il proiettore ortogonale su $\Hspace_0$ e $\Hspace_0^\perp$ il complemento ortogonale.
	Si dice che $\Hspace_0$ \textbf{riduce} $T$ se valgono entrambe le condizioni:
	\begin{itemize}
		\item[(i)] $T(D(T) \cap \Hspace_0) \subset \Hspace_0$ e $T(D(T) \cap \Hspace_0^\perp) \subset \Hspace_0^\perp$;
		\item[(ii)] $P_0(D(T)) \subset D(T)$.
	\end{itemize}
	In questo caso, l'operatore si decompone nella somma diretta delle sue parti su $\Hspace_0$ e $\Hspace_0^\perp$.
\end{definizione}

\begin{proposizione}[Caratterizzazione tramite Proiettori]
	Un sottospazio chiuso $\Hspace_0$ riduce un operatore $T$ se e solo se il relativo proiettore $P_0$ commuta con $T$ nel senso forte: $P_0 T \subset T P_0$.
	Inoltre, $T$ è irriducibile se e solo se gli unici proiettori ortogonali che commutano con esso sono $0$ e $I$.
\end{proposizione}

Un risultato tecnico ma fondamentale lega le misure spettrali (PVM) alle algebre di von Neumann:
\begin{proposizione}
	Sia $P$ una misura a valori di proiettore (PVM). L'algebra di von Neumann generata dall'immagine del PVM coincide con l'insieme degli operatori definiti da integrali di funzioni misurabili limitate rispetto a $P$:
	$$ \{ P_E \mid E \in \Sigma(X) \}'' = \left\{ \int_X f dP \mid f \in \mathcal{M}_b(X) \right\} \,. $$
\end{proposizione}

\subsection{L'Algebra di von Neumann delle Osservabili}

In fisica quantistica, si assume che le osservabili siano rappresentate dagli elementi autoaggiunti di un'algebra di von Neumann $\mathfrak{R}$. Questa costruzione è motivata dal fatto che $\mathfrak{R}$ è l'insieme massimale di operatori costruibile a partire dal reticolo delle proposizioni elementari $\mathscr{L}_{\mathfrak{R}}(\Hspace)$.

\subsubsection{Osservabili Illimitate e Struttura di Jordan}

Sebbene un'algebra di von Neumann contenga per definizione solo operatori limitati, la fisica richiede l'uso di osservabili illimitate (come posizione e momento).

\begin{osservazione}[Gestione dell'illimitatezza]
	Limitarsi agli operatori limitati non è una restrizione fisica reale. Se $A$ è un'osservabile illimitata autoaggiunta, essa è univocamente determinata dalla famiglia dei suoi troncamenti spettrali limitati $\{A_n\}_{n \in \mathbb{N}}$, dove:
	$$ A_n := \int_{[-n, n]} \lambda dP^{(A)}(\lambda) \,. $$
	Ogni $A_n$ appartiene all'algebra $\mathfrak{R}$ (se $A$ è affiliato ad essa) e $A$ viene recuperato come limite forte di questa successione.
\end{osservazione}

Un aspetto delicato riguarda la struttura algebrica delle osservabili. Lo spazio vettoriale reale degli operatori autoaggiunti non è chiuso rispetto al prodotto di operatori (poiché $AB$ non è necessariamente autoaggiunto se $A$ e $B$ non commutano).
Si può dotare tale spazio di una struttura di \textbf{Algebra di Jordan} tramite il prodotto simmetrizzato:
$$ A \circ B := \frac{1}{2}(AB + BA) \,. $$
Sebbene fisicamente attraente, il prodotto di Jordan non è associativo, rendendo la teoria matematica complicata. Tuttavia, formule integrali complesse (come la 6.7) permettono di definire funzioni di combinazioni lineari $aA + bB$ anche nel caso non commutativo, stabilendo un ponte tra la struttura lineare e quella spettrale.

\subsection{Insiemi Completi di Osservabili Compatibili (CSCO)}

Un concetto cardine per la definizione degli stati quantistici è quello di compatibilità e completezza delle misure.

\begin{definizione}[CSCO]
	Sia $\mathfrak{R}$ un'algebra di von Neumann su $\Hspace$ e $\mathfrak{A} = \{A_1, \dots, A_n\}$ un insieme finito di osservabili (limitati o illimitati affiliati a $\mathfrak{R}$) le cui misure spettrali commutano a due a due.
	$\mathfrak{A}$ si dice un \textbf{Insieme Completo di Osservabili Compatibili (CSCO)} se ogni operatore autoaggiunto limitato $B \in \mathscr{B}(\Hspace)$ che commuta con tutti i PVM di $\mathfrak{A}$ è una funzione di essi:
	$$ B = f(A_1, \dots, A_n) := \int_{\mathbb{R}^n} f(x_1, \dots, x_n) dP^{(\mathfrak{A})} $$
	per una qualche funzione misurabile $f$.
\end{definizione}

\begin{proposizione}[Algebra Generata]
	Se $\mathfrak{A} = \{A_1, \dots, A_n\}$ è un insieme di operatori con misure spettrali commutanti, l'algebra di von Neumann generata da $\mathfrak{A}$ coincide con l'insieme delle funzioni misurabili limitate degli operatori stessi:
	$$ \mathfrak{A}'' = \{ f(A_1, \dots, A_n) \mid f \in \mathcal{M}_b(\mathbb{R}^n) \} \,. $$
\end{proposizione}

\subsubsection{Conseguenze Fisiche: Stati Puri e Commutante Abeliano}

La nozione di CSCO ha due implicazioni fisiche fondamentali.

\textbf{1. Preparazione degli Stati Puri:}
Se gli osservabili $A_1, \dots, A_n$ hanno spettro puramente puntuale, la misurazione simultanea di questi osservabili "prepara" il sistema in uno stato univocamente determinato.
Se $\Hspace_{\alpha_1, \dots, \alpha_n}$ è l'autospazio comune associato agli autovalori misurati, la condizione di completezza implica che:
$$ \dim(\Hspace_{\alpha_1, \dots, \alpha_n}) = 1 \,. $$
Pertanto, dopo la misura, il sistema collassa su un unico vettore (a meno di fase), identificando uno stato puro senza ambiguità (degenerazione rimossa).

\textbf{2. Struttura del Commutante:}
Esiste un legame profondo tra l'esistenza di un CSCO e la commutatività dell'algebra "duale".

\begin{proposizione}
	Se un sistema fisico ammette un insieme completo di osservabili compatibili $\mathfrak{A}$, allora il commutante dell'algebra delle osservabili $\mathfrak{R}'$ è \textbf{Abeliano}.
	In questo caso, il commutante coincide con il centro dell'algebra: $\mathfrak{R}' = \mathfrak{Z}(\mathfrak{R})$.
\end{proposizione}

\begin{osservazione}[Implicazioni]
	\begin{itemize}
		\item Se $\mathfrak{R}'$ è Abeliano, allora $\mathfrak{R}' \subset \mathfrak{R}'' = \mathfrak{R}$, confermando che $\mathfrak{R}' = \mathfrak{R} \cap \mathfrak{R}'$.
		\item Se $\mathfrak{R}'$ \textbf{non} è Abeliano (come accade in alcune teorie di gauge non abeliane), non possono esistere CSCO. In tali teorie, è impossibile preparare stati vettoriali puri misurando osservabili compatibili, poiché tali stati semplicemente non esistono.
	\end{itemize}
\end{osservazione}

\begin{esempio}[Esempi Fisici Notevoli]
	\begin{enumerate}
		\item \textbf{Oscillatore Armonico:} In $L^2(\mathbb{R})$, l'operatore Hamiltoniano $H$ da solo costituisce un CSCO, poiché il suo spettro è non degenere.
		\item \textbf{Particella Libera:} In $L^2(\mathbb{R}^3)$, gli operatori di posizione $\mathfrak{A}_1 = \{X_1, X_2, X_3\}$ formano un CSCO. Analogamente, gli operatori di momento $\mathfrak{A}_2 = \{P_1, P_2, P_3\}$ formano un altro CSCO. L'algebra generata è massimale, ovvero $\mathfrak{R} = \mathscr{B}(L^2(\mathbb{R}^3))$.
		\item \textbf{Elettrone con Spin:} Per una particella con spin $1/2$, lo spazio è $L^2(\mathbb{R}^3) \otimes \mathbb{C}^2$. Un CSCO è dato da $\{X_1, X_2, X_3, S_z\}$ oppure da $\{P_1, P_2, P_3, S_z\}$. Anche in questo caso il commutante è banale e $\mathfrak{R} = \mathscr{B}(\Hspace)$.
		\item \textbf{Atomo di Idrogeno:} Un tipico CSCO con spettro puntuale è la quadrupla $\{H, L^2, L_z, S_z\}$ (Energia, Momento angolare totale, Componente Z, Spin Z), limitatamente al sottospazio delle energie negative (stati legati).
	\end{enumerate}
\end{esempio}


\subsection{Caratterizzazione Algebrica degli Stati: Il Teorema di Riesz Non Commutativo}

Questa sezione esplora una caratterizzazione puramente matematica dello spazio degli stati $\mathscr{S}(\Hspace)$, fondamentale per la formulazione algebrica delle teorie quantistiche (che vedremo in seguito). L'idea centrale è mostrare che la relazione tra stati e operatori è l'analogo "non commutativo" del celebre Teorema di Rappresentazione di Riesz per le misure classiche.

\subsubsection{Stati come Funzionali Lineari}

Ricordiamo che un operatore di classe traccia positivo $T \in \mathfrak{B}_1(\Hspace)$ definisce un funzionale lineare sull'algebra degli operatori compatti $\mathfrak{B}_\infty(\Hspace)$ tramite la traccia:
$$ \omega_T: \mathfrak{B}_\infty(\Hspace) \to \mathbb{C}, \quad \omega_T(A) := \text{tr}(TA) \,. $$
Questo funzionale è positivo ($\omega_T(A^*A) \ge 0$) e la sua norma coincide con la traccia di $T$: $\|\omega_T\| = \text{tr}(T)$.

Possiamo quindi ribaltare la prospettiva e definire gli stati in modo astratto.

\begin{definizione}[Stato Algebrico]
	Sia $\Hspace$ uno spazio di Hilbert complesso. Chiamiamo \textbf{stato algebrico} sulla $C^*$-algebra $\mathfrak{B}_\infty(\Hspace)$ ogni funzionale lineare $\omega: \mathfrak{B}_\infty(\Hspace) \to \mathbb{C}$ che sia:
	\begin{enumerate}
		\item \textbf{Positivo:} $\omega(A^*A) \ge 0$ per ogni $A \in \mathfrak{B}_\infty(\Hspace)$;
		\item \textbf{Normalizzato:} $\|\omega\| = 1$.
	\end{enumerate}
	L'insieme di tali stati è denotato con $\mathfrak{E}(\mathfrak{B}_\infty(\Hspace))$.
\end{definizione}

Il seguente teorema stabilisce che questa definizione astratta cattura esattamente gli stati fisici (matrici densità) che conosciamo.

\begin{teorema}[Caratterizzazione degli Stati]
	La mappa $T \mapsto \omega_T$ definita da $\omega_T(A) = \text{tr}(TA)$ è una biiezione tra l'insieme degli operatori di stato $\mathscr{S}(\Hspace)$ e l'insieme degli stati algebrici $\mathfrak{E}(\mathfrak{B}_\infty(\Hspace))$.
	In altre parole, ogni funzionale positivo e normalizzato sugli operatori compatti proviene da un unico operatore densità.
\end{teorema}

\subsubsection{Analisi del Ragionamento: L'analogia con Riesz}

Per comprendere appieno il significato "non commutativo" di questo risultato, è utile confrontarlo con il Teorema di Riesz classico per l'integrazione.

\textbf{Teorema di Riesz Classico (Analisi):}
Consideriamo uno spazio di Hausdorff localmente compatto $X$.
Il teorema afferma che ogni funzionale lineare positivo limitato $\Lambda$ sullo spazio delle funzioni continue che svaniscono all'infinito, $C_0(X)$, è rappresentato da un'unica misura di Borel regolare $\mu$:
$$ \Lambda(f) = \int_X f \, d\mu \,. $$

\textbf{Versione Quantistica (Non Commutativa):}
In Meccanica Quantistica, lo spazio delle fasi viene sostituito dallo spazio di Hilbert e le funzioni commutano diventano operatori.
Possiamo istituire il seguente parallelismo:


\begin{table}[h]
	\centering
	\begin{tabular}{|c|c|}
		\hline
		\textbf{Caso Classico (Commutativo)} & \textbf{Caso Quantistico (Non Commutativo)} \\
		\hline
		Spazio $X$ (localmente compatto) & Spazio di Hilbert $\Hspace$ ($\dim \Hspace = \infty$) \\
		\hline
		Funzioni $C_0(X)$ (svaniscono all'$\infty$) & Operatori Compatti $\mathfrak{B}_\infty(\Hspace)$ (spettro tende a 0) \\
		\hline
		Funzionale Positivo $\Lambda$ & Funzionale Positivo $\omega$ \\
		\hline
		Misura di Probabilità $\mu$ & Operatore di Stato $T$ (Matrice Densità) \\
		\hline
		Integrale $\Lambda(f) = \int f d\mu$ & Traccia $\omega(A) = \text{tr}(TA)$ \\
		\hline
	\end{tabular}
	\caption{Dizionario tra Teoria della Misura Classica e Meccanica Quantistica.}
\end{table}

\begin{osservazione}[Interpretazione Fisica]
	La proposizione implica che possiamo pensare alla traccia $\text{tr}(TA)$ come a un \textbf{integrale non commutativo} dell'osservabile $A$ rispetto alla "misura" definita dallo stato $T$.
	Formalmente, se $T_\rho$ è l'operatore associato a una misura di probabilità $\rho$ sul reticolo $\mathscr{L}(\Hspace)$ (grazie al teorema di Gleason), scriveremo simbolicamente:
	\begin{equation}
		\int_{\mathscr{L}(\Hspace)} A \, d\rho := \text{tr}(T_\rho A) \,.
	\end{equation}
	Questa notazione unifica il concetto di valore di aspettazione quantistico con quello classico di media integrale.
\end{osservazione}

Concludiamo con la formulazione formale del teorema di rappresentazione in questo contesto.

\begin{teorema}[Teorema di Riesz Non Commutativo]
	Sia $\Hspace$ uno spazio di Hilbert complesso (separabile o di dimensione finita $\neq 2$).
	Sia $\omega: \mathfrak{B}_\infty(\Hspace) \to \mathbb{C}$ un funzionale lineare positivo limitato con norma unitaria.
	Allora esiste un'unica misura di probabilità $\rho_\omega: \mathscr{L}(\Hspace) \to [0,1]$ (soddisfacente gli assiomi della misura quantistica) tale che:
	\begin{equation}
		\omega(A) = \int_{\mathscr{L}(\Hspace)} A \, d\rho_\omega \quad \forall A \in \mathfrak{B}_\infty(\Hspace) \,.
	\end{equation}
	In concreto, questo integrale è calcolato come $\text{tr}(T A)$, dove $T$ è l'operatore di stato associato a $\omega$.
\end{teorema}