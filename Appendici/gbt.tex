\section{Teorie Bayesiane Generali (GBT)}

In questa sezione presentiamo il framework delle Teorie Bayesiane Generali (General Bayesian Theories - GBT), un approccio ricostruttivo della teoria quantistica dal punto di vista di un agente che effettua scommesse sugli esiti di esperimenti possibili. Questo framework organizza le credenze dell'agente e le aggiorna quando diventano disponibili nuove informazioni, includendo come casi speciali la teoria della probabilità classica e quella quantistica.

\subsection{Credenze e Probabilità}
Consideriamo un agente che scommette sugli esiti di esperimenti effettuati su un dato sistema fisico. Il punto di partenza è una classe basilare di esperimenti, denominati \textbf{esperimenti principali}.

\begin{definizione}[Esperimenti Principali]
	Gli esiti degli esperimenti principali formano un unico spazio campionario $X$, dotato di una $\sigma$-algebra di eventi $\Sigma$. $\Sigma$ è una collezione di sottoinsiemi di $X$ che soddisfa le seguenti proprietà:
	\begin{enumerate}
		\item $X \in \Sigma$;
		\item Se $E \in \Sigma$, allora il complemento $(X \setminus E) \in \Sigma$;
		\item Se una collezione $(E_i)_i \subset \Sigma$, allora la loro unione $\bigcup_i E_i \in \Sigma$.
	\end{enumerate}
	Un esperimento principale corrisponde a una partizione $\mathbb{E} = (E_i)_i$ dello spazio campionario $X$ in eventi disgiunti. Per brevità, identifichiamo l'esperimento con la partizione corrispondente.
\end{definizione}

L'agente si affida alle proprie credenze (beliefs) per effettuare scommesse. Queste possono includere convinzioni sulle leggi della fisica o sulla storia precedente del sistema.

\begin{definizione}[Belief]
	Denotiamo con $B$ l'insieme di tutte le possibili credenze. Per una data Belief $\beta \in B$, l'agente assegna una distribuzione di probabilità $p: \Sigma \to [0,1]$, indicata come $p(E|\beta)$, che soddisfa le condizioni standard:
	\begin{enumerate}
		\item $p(E|\beta) \ge 0$ per tutti gli eventi $E$;
		\item $p(\bigcup_i E_i | \beta) = \sum_i p(E_i | \beta)$ se gli eventi $E_i$ sono mutuamente disgiunti;
		\item $p(X|\beta) = 1$.
	\end{enumerate}
	Si assume che la probabilità di un evento $E$ sia indipendente dallo specifico esperimento $\mathbb{E}$ nel quale $E$ si manifesta.
\end{definizione}

È fondamentale notare che la Belief $\beta$ determina l'assegnazione di probabilità $p(E|\beta)$, ma non viceversa: in generale, una Belief contiene più informazioni delle sole probabilità degli esiti degli esperimenti principali.

\subsection{Aggiornamenti Bayesiani e Assiomi di Coerenza}
Quando l'agente riceve la garanzia che un evento $E$ si è verificato, aggiorna la sua vecchia Belief $\beta$ in una nuova Belief, denotata come $\beta' = E\beta$. L'obiettivo dell'aggiornamento è calcolare le probabilità condizionate. La mappa di aggiornamento $\beta \mapsto E\beta$ non rappresenta un processo fisico sul sistema, ma un'operazione interna all'agente.

Il framework si fonda su tre assiomi di coerenza che vincolano il modo in cui l'agente aggiorna le proprie credenze.

\begin{criterio}[Assioma 1: Regola delle probabilità condizionate]
	Per ogni Belief iniziale $\beta \in B$ e per ogni coppia di eventi $E$ e $F$ con $p(E|\beta) \neq 0$, la Belief aggiornata $E\beta$ soddisfa:
	\begin{equation}
		p(F|E\beta) = \frac{p(E \cap F|\beta)}{p(E|\beta)}.
	\end{equation}
	Questo implica che, una volta aggiornata la Belief basandosi su $E$, l'agente diventa certo dell'evento $E$, ovvero $p(E|E\beta) = 1$.
\end{criterio}

Il secondo assioma riguarda la coerenza dell'agente rispetto a informazioni che già possiede.

\begin{criterio}[Assioma 2: Coerenza in avanti - Forward Consistency]
	Se l'agente è certo dell'evento $E$, allora la Belief dell'agente non cambia sotto l'aggiornamento per l'evento $E$. Matematicamente:
	\begin{equation}
		\forall \beta \in B, \forall E \in \Sigma, \quad p(E|\beta) = 1 \implies E\beta = \beta.
	\end{equation}
	Una conseguenza immediata è che l'evento totale $X$ non porta ad alcun aggiornamento: $X\beta = \beta$.
\end{criterio}

\subsection{Azioni e Coerenza all'Indietro}
Le condizioni di una scommessa possono essere alterate da un'azione. L'insieme delle azioni, denotato con $\text{Act}$, è assunto essere un monoide: le azioni possono essere composte (associatività) ed esiste un'azione identità $\mathcal{I}$.
Quando viene eseguita un'azione $\mathcal{A}$, l'agente cambia la sua Belief in $\beta' = \mathcal{A}\beta$. Questo cambio soddisfa:
\begin{itemize}
	\item $(\mathcal{AB})\beta = \mathcal{A}(\mathcal{B}\beta)$;
	\item $\mathcal{I}\beta = \beta$.
\end{itemize}

L'introduzione delle azioni permette di definire \textbf{esperimenti sequenziali}, costituiti da una sequenza di azioni intervallate da esperimenti principali (es. $\mathcal{A}, \mathbb{E}, \mathcal{B}, \mathbb{F}$). La Belief fornisce una distribuzione di probabilità congiunta per la sequenza.
Si dice che l'evento $F$ implica l'evento $E$ in un esperimento sequenziale se la probabilità condizionata di aver ottenuto $E$ dato $F$ è 1 ($P(E|F)=1$) e se $F$ è "inalterato" dall'esperimento $\mathbb{E}$ (la probabilità marginale di $F$ è la stessa sia che $\mathbb{E}$ venga eseguito o meno).

\begin{criterio}[Assioma 3: Coerenza all'indietro - Backward Consistency]
	Se l'evento $F \in \mathbb{F}$ implica l'evento $E \in \mathbb{E}$ in un esperimento sequenziale $(\mathcal{A}, \mathbb{E}, \mathcal{B}, \mathbb{F})$, allora l'aggiornamento per l'evento $E$ può essere omesso nella Belief finale aggiornata per l'evento $F$. Matematicamente, se valgono le condizioni di implicazione e non-disturbo:
	\begin{equation}
		F\mathcal{B}E\mathcal{A}\beta = F\mathcal{B}\mathcal{A}\beta.
	\end{equation}
\end{criterio}

Le teorie che soddisfano gli Assiomi 1, 2 e 3 sono definite \textbf{Teorie Bayesiane Generali}.

\subsection{Esperimenti Ideali ed Emergenza del Principio di Esclusività}
Ogni GBT contiene una classe speciale di esperimenti, detti \textbf{esperimenti ideali}.


\begin{definizione}[Esperimento sequenzialmente raffinabile]
	Un esperimento $(\mathcal{A}, \mathbb{E})$ è detto \textbf{sequenzialmente raffinabile} se esiste un'azione $\mathcal{A}'$ tale che, per ogni raffinamento $(\mathcal{B}, \mathbb{F})$ (un esperimento più fine le cui probabilità marginali coincidono con quelle di $\mathbb{E}$), l'esperimento grossolano non altera la distribuzione di probabilità dell'esperimento fine, a patto che l'agente esegua l'azione $\mathcal{A}'$ tra i due.
\end{definizione}

\begin{definizione}[Esperimento Ideale]
	Un esperimento $(\mathcal{A}, \mathbb{E})$ è ideale se esiste un'azione $\mathcal{A}'$ che garantisce la proprietà di non disturbo per \textit{tutti} i possibili raffinamenti.
\end{definizione}

\begin{teorema}[Esistenza]
	Gli esperimenti ideali esistono in ogni GBT. In particolare, ogni esperimento principale $\mathbb{E}$ è un esperimento ideale.
\end{teorema}

Una conseguenza centrale del framework GBT è la derivazione del \textbf{Principio di Esclusività}.
Consideriamo un insieme di esiti $\mathcal{O}$ appartenenti a esperimenti (generalmente) diversi. Due esiti $(\mathcal{A}, E)$ e $(\mathcal{A}', E')$ sono detti \textit{mutuamente esclusivi} se sono equivalenti a due esiti distinti di un singolo esperimento $\mathcal{F}$. Un insieme di esiti è \textit{esclusivo a coppie} se ogni coppia nell'insieme è mutuamente esclusiva.

Il Principio di Esclusività afferma che per ogni insieme di esiti esclusivi a coppie $\{(\mathcal{A}_n, E_n)\}$, la somma delle loro probabilità non può superare 1:
\begin{equation}
	\sum_n p(E_n | \mathcal{A}_n \beta) \le 1, \quad \forall \beta \in B.
\end{equation}

\begin{teorema}[Validità del Principio di Esclusività]
	In ogni GBT, gli esiti degli esperimenti ideali soddisfano il principio di esclusività.
\end{teorema}

Questo risultato implica che l'insieme delle correlazioni quantistiche per ogni scenario di contestualità di Bell o Kochen-Specker può essere caratterizzato completamente in termini di condizioni di coerenza bayesiana, assumendo che due esperimenti statisticamente indipendenti possano essere eseguiti in parallelo.

\subsection{Esempio: La Teoria Quantistica come GBT}
La teoria quantistica rientra nel framework GBT con le seguenti identificazioni:
\begin{itemize}
	\item \textbf{Spazio Campionario:} Per uno spazio di Hilbert $\Hspace = \C^d$, $X = \{1, \dots, d\}$.
	\item \textbf{Esperimenti Principali:} Misure proiettive diagonali in una base ortonormale fissata $\{\ket{j}\}$. I proiettori sono $P_{E_i} = \sum_{j \in E_i} \ket{j}\bra{j}$.
	\item \textbf{Credenze:} Operatori densità $\rho \in L(\Hspace)$, $\rho \ge 0$, $\text{tr}(\rho)=1$. La probabilità è data dalla regola di Born: $p(E_i|\rho) = \text{tr}(P_{E_i}\rho)$.
	\item \textbf{Aggiornamento:} Regola di Lüders. $E_i \rho = \frac{P_{E_i}\rho P_{E_i}}{\text{tr}(P_{E_i}\rho)}$.
	\item \textbf{Azioni:} Canali quantistici (mappe completamente positive che preservano la traccia).
\end{itemize}
In questo contesto, l'insieme degli esperimenti ideali coincide esattamente con l'insieme delle misure proiettive.