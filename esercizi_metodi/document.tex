\documentclass{article}
\usepackage[utf8]{inputenc}
\usepackage[T1]{fontenc}
\usepackage[italian]{babel}
\usepackage{amsmath, amssymb, amsthm, mathtools}
\usepackage{geometry}
\usepackage{physics} 


\geometry{a4paper, top=2.5cm, bottom=2.5cm, left=2.5cm, right=2.5cm}

% Ambienti per teoremi e definizioni
\theoremstyle{definition}
\newtheorem{ex}{Esercizio}
\newtheorem*{sol}{Soluzione}
\newtheorem*{oss}{Osservazione}
\newtheorem{lemma}{Lemma}

% Macro utili
\newcommand{\HH}{\mathcal{H}}
\newcommand{\BB}{\mathcal{B}}
\newcommand{\K}{\mathbb{K}} % Campo scalare (R o C)
\newcommand{\inner}[2]{\left\langle #1, #2 \right\rangle}
\DeclareMathOperator{\Ran}{Ran}
\DeclareMathOperator{\Ker}{Ker}
\newcommand{\C}{\mathbb{C}}
\newcommand{\Z}{\mathbb{Z}}
\newcommand{\N}{\mathbb{N}}
\newcommand{\R}{\mathbb{R}}
\newcommand{\einx}{e^{inx}}
\newcommand{\Hone}{H^1(I)}
\newcommand{\Honez}{H^1_0(I)}
\newcommand{\Ltwo}{L^2(I)}
\newcommand{\Dom}{\mathcal{D}}
\newcommand{\Op}[1]{\hat{#1}}
\newcommand{\vecOp}[1]{\hat{\mathbf{#1}}}
\newcommand{\normprime}[1]{\left\lVert#1\right\rVert'}
\newcommand{\closure}[1]{\overline{#1}}
\newcommand{\KK}{\mathcal{K}} % Usato per H'
\newcommand{\Id}{\mathbb{I}}
\newcommand{\innerH}[2]{\left\langle #1, #2 \right\rangle_{\HH}}
\newcommand{\innerK}[2]{\left\langle #1, #2 \right\rangle_{\HH'}}

\title{\textbf{Esercizi Operatori}}
\author{Tommaso Pedroni}
\date{}

\begin{document}
	
	\maketitle
	
	\section*{Traccia dell'Esercizio 1-F}
	Dati due spazi di Hilbert $\HH$ e $\HH'$ e denotando con $\Id$ e $\Id'$ i rispettivi operatori identità, si mostri che $T \in \mathcal{L}(\HH; \HH')$ è unitario se e solo se è limitato e $T^*T = \Id$ mentre $TT^* = \Id'$.
	
	\hrule
	\vspace{0.5cm}
	
	\begin{sol}
		Ricordiamo preliminariarmente la definizione di operatore unitario. Un operatore $U: \HH \to \HH'$ si dice \textit{unitario} se è un isomorfismo isometrico suriettivo tra i due spazi di Hilbert. Ovvero, se conserva il prodotto scalare (e quindi è un'isometria) ed è suriettivo.
		
	\paragraph{Implicazione diretta ($\Rightarrow$):}
	Sia $T$ unitario.
	\begin{itemize}
		\item \textbf{Limitatezza:} Poiché $T$ è unitario, preserva la norma, ossia $\norm{T\psi}_{\HH'} = \norm{\psi}_{\HH}$ per ogni $\psi \in \HH$. Di conseguenza, la norma operatoriale è $\norm{T} = \sup_{\norm{\psi}=1} \norm{T\psi} = 1$, il che implica che $T$ è limitato.
		
		\item \textbf{Relazioni con l'aggiunto:} Poiché $T$ conserva il prodotto scalare, per la definizione di aggiunto vale:
		\[
		\innerH{\phi}{T^*T\psi} = \innerK{T\phi}{T\psi} = \innerH{\phi}{\psi} \quad \forall \phi, \psi \in \HH.
		\]
		Dall'arbitrarietà dei vettori segue l'identità operatoriale:
		\[
		T^*T = \Id.
		\]
		Per mostrare che $TT^* = \Id'$, osserviamo che $T$, essendo unitario, è per definizione una biiezione suriettiva. Pertanto ammette un unico operatore inverso $T^{-1}$.
		Dall'uguaglianza $T^*T = \Id$, componendo a destra con $T^{-1}$ otteniamo:
		\[
		T^* T T^{-1} = \Id T^{-1} \implies T^* = T^{-1}.
		\]
		Sostituendo $T^*$ a $T^{-1}$ nella relazione fondamentale dell'inverso ($TT^{-1} = \Id'$), concludiamo che:
		\[
		TT^* = \Id'.
		\]
	\end{itemize}
		
		\paragraph{Implicazione inversa ($\Leftarrow$):}
		Sia $T \in \mathcal{L}(\HH; \HH')$ limitato tale che $T^*T = \Id$ e $TT^* = \Id'$.
		\begin{itemize}
			\item \textbf{Isometria:} Dalla condizione $T^*T = \Id$, per ogni $\psi \in \HH$ abbiamo:
			\[
			\norm{T\psi}^2_{\HH'} = \innerK{T\psi}{T\psi} = \innerH{\psi}{T^*T\psi} = \innerH{\psi}{\Id\psi} = \norm{\psi}^2_{\HH}.
			\]
			Quindi $T$ è un'isometria.
			\item \textbf{Suriettività:} Dobbiamo mostrare che $\text{Ran}(T) = \HH'$. Sia $\eta \in \HH'$ un vettore arbitrario. Consideriamo il vettore $\xi = T^*\eta \in \HH$. Applicando $T$ otteniamo:
			\[
			T\xi = T(T^*\eta) = (TT^*)\eta = \Id'\eta = \eta.
			\]
			Dunque, per ogni $\eta \in \HH'$ esiste una controimmagine $\xi \in \HH$, il che prova che $T$ è suriettivo.
		\end{itemize}
		Essendo $T$ un'isometria suriettiva limitata, $T$ è unitario.
	\end{sol}
	
	\vspace{1cm}
	
	\section*{Traccia dell'Esercizio 2-F}
	Sia dato uno spazio di Hilbert infinito dimensionale, mostrare che l'operatore identità non può mai essere compatto.
	
	\hrule
	\vspace{0.5cm}
	
	\begin{sol}
		Sia $\HH$ uno spazio di Hilbert separabile con $\dim(\HH) = \infty$ e sia $\Id: \HH \to \HH$ l'operatore identità.
		
		Un operatore $K$ si dice \textit{compatto} se mappa insiemi limitati in insiemi relativamente compatti. Equivalentemente, $K$ è compatto se, per ogni successione limitata $\{x_n\}_{n \in \mathbb{N}} \subset \HH$, la successione trasformata $\{Kx_n\}_{n \in \mathbb{N}}$ ammette una sottosuccessione convergente in $\HH$.
		
		Poiché $\HH$ è infinito dimensionale, esiste in esso un sistema ortonormale infinito $\{e_n\}_{n=1}^\infty$ (si pensi al risultato della procedura di Gram-Schmidt applicata a un insieme numerabile linearmente indipendente).
		
		Consideriamo la successione $\{e_n\}_{n \in \mathbb{N}}$. Essa è limitata poiché $\norm{e_n} = 1$ per ogni $n$.
		Valutiamo l'azione dell'identità su tale successione: $\Id e_n = e_n$.
		
		Affinché $\Id$ sia compatto, dalla successione $\{e_n\}$ si dovrebbe poter estrarre una sottosuccessione convergente (e quindi di Cauchy). Tuttavia, per ogni $n \neq m$, calcoliamo la distanza tra due elementi della base ortonormale:
		\[
		\norm{e_n - e_m}^2 = \inner{e_n - e_m}{e_n - e_m} = \norm{e_n}^2 + \norm{e_m}^2 - 2\text{Re}\inner{e_n}{e_m} = 1 + 1 - 0 = 2.
		\]
		Dunque $\norm{e_n - e_m} = \sqrt{2}$ per ogni $n \neq m$.
		
		Questo implica che non è possibile estrarre alcuna sottosuccessione di Cauchy da $\{e_n\}$, poiché gli elementi mantengono una distanza costante e non nulla l'uno dall'altro. Di conseguenza, la successione $\{ \Id e_n \}$ non ammette sottosuccessioni convergenti.
		
		Concludiamo che l'operatore identità in uno spazio di Hilbert infinito dimensionale non è compatto.
	\end{sol}
	
	\section*{Traccia dell'Esercizio 3-F}
	Sia $\HH$ uno spazio di Hilbert e siano $T$ e $T'$ due operatori densamente definiti. Si mostri che:
	\begin{enumerate}
		\item[(a)] Se $T \subset T'$, allora $(T')^* \subset T^*$.
		\item[(b)] $(T')^*T^* \subseteq (TT')^*$ e l'uguaglianza vale se $T \in \BB(\HH)$.
	\end{enumerate}
	
	\hrule
	\vspace{0.5cm}
	
	\begin{sol}
		Dividiamo l'esercizio in due parti.
		\subsection*{Punto (a)}
		L'ipotesi $T \subset T'$ significa che:
		\[
		\Dom{(T)} \subset \Dom{(T')} \quad \text{e} \quad Tx = T'x \quad \forall x \in \Dom{(T)}.
		\]
		Sia $\phi \in \Dom{(T')^*}$. Per definizione di operatore aggiunto, ciò significa che esiste un vettore $\eta \in \HH$ (denotato con $(T')^*\phi$) tale che:
		\[
		\inner{\phi}{T'x} = \inner{\eta}{x} \quad \forall x \in \Dom{(T')}.
		\]
		Poiché $\Dom{(T)} \subset \Dom{(T')}$, la relazione sopra vale in particolare per ogni $x \in \Dom{(T)}$. Inoltre, per tali $x$, abbiamo $T'x = Tx$. Possiamo quindi scrivere:
		\[
		\inner{\phi}{Tx} = \inner{\eta}{x} \quad \forall x \in \Dom{(T)}.
		\]
		Questa è esattamente la definizione che assicura che $\phi \in \Dom{(T^*)}$ e che $T^*\phi = \eta$.
		
		Abbiamo mostrato che $\phi \in \Dom{((T')^*)} \implies \phi \in \Dom{(T^*)}$ e che l'azione degli operatori coincide. Pertanto:
		\[
		(T')^* \subset T^*.
		\]
		
		\subsection*{Punto (b)}
		\paragraph{Inclusione $(T')^*T^* \subseteq (TT')^*$.}
		Sia $\phi \in \Dom{(T'^*T^*)}$. Per definizione di dominio del prodotto di operatori, questo implica che:
		\[
		\phi \in \Dom{(T^*)} \quad \text{e} \quad T^*\phi \in \Dom{(T'^*)}.
		\]
		Vogliamo mostrare che $\phi \in \Dom{((TT')^*)}$. Consideriamo un generico $x \in \Dom{(TT')}$.
		Ricordiamo che $x \in \Dom{(TT')} \iff x \in \Dom{(T')} \land T'x \in \Dom{(T)}$.
		
		Valutiamo il prodotto scalare $\inner{\phi}{TT'x}$:
		\begin{enumerate}
			\item Poiché $T'x \in \Dom{(T)}$ e $\phi \in \Dom{(T^*)}$, possiamo scaricare $T$:
			\[
			\inner{\phi}{T(T'x)} = \inner{T^*\phi}{T'x}.
			\]
			\item Ora, poniamo $\psi \coloneqq T^*\phi$. Sappiamo per ipotesi che $\psi \in \Dom{((T')^*)}$. Inoltre $x \in \Dom{(T')}$. Possiamo scaricare $T'$:
			\[
			\inner{\psi}{T'x} = \inner{(T')^*\psi}{x} = \inner{(T')^*T^*\phi}{x}.
			\]
		\end{enumerate}
		Mettendo insieme i passaggi, abbiamo trovato che per ogni $x \in \Dom{(TT')}$:
		\[
		\inner{\phi}{TT'x} = \inner{(T')^*T^*\phi}{x}.
		\]
		Questo prova che $\phi \in \Dom{((TT')^*)}$ e che $(TT')^*\phi = (T')^*T^*\phi$.
		
		\paragraph{Uguaglianza nel caso limitato.}
		Sia ora $T \in \BB(\HH)$. Questo implica che $T$ è limitato e definito su tutto lo spazio, ovvero $\Dom{(T)} = \HH$ e $\Dom{(T^*)} = \HH$.
		Dobbiamo mostrare l'inclusione inversa: $(TT')^* \subseteq (T')^*T^*$.
		
		Osserviamo preliminarmente che, essendo $\Dom{(T)}=\HH$, il dominio del prodotto $TT'$ si semplifica:
		\[
		\Dom{(TT')} = \{ x \in \Dom{(T')} : T'x \in \Dom{(T)} \} = \{ x \in \Dom{(T')} : T'x \in \HH \} = \Dom{(T')}.
		\]
		Sia ora $\phi \in \Dom{((TT')^*)}$. Questo significa che esiste un $\eta$ tale che:
		\[
		\inner{\phi}{TT'x} = \inner{\eta}{x} \quad \forall x \in \Dom{(TT')} = \Dom{(T')}.
		\]
		Essendo $T$ limitato e definito ovunque, il suo aggiunto $T^*$ è anch'esso definito ovunque ($T^* \in \BB(\HH)$). Possiamo usare la proprietà dell'aggiunto per operatori limitati $\inner{\phi}{Ty} = \inner{T^*\phi}{y}$ per qualsiasi $y \in \HH$. Ponendo $y = T'x$ (che è un vettore lecito in $\HH$), otteniamo:
		\[
		\inner{\phi}{T(T'x)} = \inner{T^*\phi}{(T'x)}.
		\]
		Confrontando le due espressioni, abbiamo:
		\[
		\inner{T^*\phi}{T'x} = \inner{\eta}{x} \quad \forall x \in \Dom{(T')}.
		\]
		Questa uguaglianza ci dice esattamente che il funzionale lineare $x \mapsto \inner{T^*\phi}{T'x}$ è limitato (rappresentabile da $\eta$). Per definizione di aggiunto di $T'$, ciò implica che il vettore $T^*\phi$ appartiene al dominio di $(T')^*$, ovvero:
		\[
		T^*\phi \in \Dom{((T')^*)}.
		\]
		Poiché $\phi \in \HH = \Dom{(T^*)}$ è sempre vero, la condizione $T^*\phi \in \Dom{((T')^*)}$ è sufficiente per affermare che:
		\[
		\phi \in \Dom{((T')^*T^*)}.
		\]
		Ciò conclude la dimostrazione dell'uguaglianza.
	\end{sol}
	
	\section*{Traccia dell'Esercizio 4-F}
	Sia $\HH$ uno spazio di Hilbert e sia dato $T : \Dom(T) \subset \HH \to \HH$. Si mostri che $T$ è essenzialmente autoaggiunto se e solo se $T$ è denso e chiudibile in $\HH$ e $T^* = \closure{T}$.
	\vspace{0.5cm}
	\hrule
	\vspace{0.5cm}
	
	\begin{sol}
		
	\textbf{Implicazione diretta ($\Rightarrow$):}
		Sia $T$ essenzialmente autoaggiunto.
		Per la \textbf{Definizione 84}, questo implica tre fatti:
		
		\begin{enumerate}
			\item $\Dom(T)$ è denso in $\HH$
			\item $\Dom(T^*)$ è denso in $\HH$
			\item $T^* = (T^*)^*$
		\end{enumerate}
		Poiché $\Dom(T^*)$ è denso, per il \textbf{Teorema 82 (punto 2)}, possiamo affermare che $T$ è chiudibile e che vale l'identità fondamentale:
		\[
		\closure{T} = (T^*)^*.
		\]
		Sostituendo questa identità nella condizione 3 (essenziale autoaggiunzione), otteniamo:
		\[
		T^* = \closure{T}.
		\]
		Abbiamo quindi mostrato che $T$ è denso (cond. 1), chiudibile (dal Teorema 82) e che $T^* = \closure{T}$.
		\newline
		\textbf{Implicazione inversa ($\Leftarrow$):}
		Supponiamo che:
		\begin{enumerate}
			\item[H1.] $\Dom(T)$ sia denso.
			\item[H2.] $T$ sia chiudibile.
			\item[H3.] $T^* = \closure{T}$.
		\end{enumerate}
		Dobbiamo verificare che $T$ soddisfi la \textbf{Definizione 84}.
		La prima condizione della definizione ($\Dom(T)$ denso) è garantita da H1.
		
		Poiché $T$ è chiudibile (H2), il \textbf{Teorema 82 (punto 2)} assicura che $\Dom(T^*)$ è denso (soddisfacendo così la seconda condizione della Def. 84) e che vale:
		\[
		(T^*)^* = \closure{T}.
		\]
		Utilizzando l'ipotesi H3 ($T^* = \closure{T}$), possiamo sostituire $\closure{T}$ nell'equazione precedente:
		\[
		(T^*)^* = T^*.
		\]
		Questo soddisfa la terza condizione della \textbf{Definizione 84}. Dunque $T$ è essenzialmente autoaggiunto.
	\end{sol}
	
	\vspace{1cm}
	
	\section*{Traccia dell'Esercizio 5-F}
	Dati due spazi di Hilbert $\HH$ e $\KK$ e un operatore unitario $U : \KK \to \HH$, si mostri che se $T : \Dom(T) \subset \HH \to \HH$ è essenzialmente autoaggiunto, allora lo è anche l'operatore $T' \doteq U^{-1} T U$ definito su $\Dom(T') = U^{-1}\Dom(T)$.
	
	\hrule
	\vspace{0.5cm}
	
	\begin{sol}
		Sia $T$ essenzialmente autoaggiunto. Per la \textbf{Definizione 84}, valgono:
		\begin{itemize}
			\item $\Dom(T)$ e $\Dom(T^*)$ sono densi in $\HH$.
			\item $T^* = (T^*)^*$.
		\end{itemize}
		Dobbiamo verificare le stesse condizioni per $T'$.
		\newline PS: Credo ci sia un'esercitazione di Beatrice dove dimostra letteralmente le stesse cose. In particolare dimostra che T simmetrico $\Rightarrow$ T' simmetrico, quindi sappiamo già la densità dei domini. 
		Essendo $U$ limitato con inverso limitato ($U^*=U^{-1}$), vale la regola dell'aggiunto del prodotto (esercizio 3-F punto (b)):
		\[
		(T')^* = (U^{-1}TU)^* = U^* T^* (U^{-1})^* = U^{-1} T^* U.
		\]
		Il dominio di $(T')^*$ è $\Dom((T')^*) = U^{-1}\Dom(T^*)$.
		\newline
		Dobbiamo verificare che $(T')^* = ((T')^*)^*$.
		Calcoliamo l'aggiunto dell'aggiunto di $T'$:
		\[
		((T')^*)^* = \left( U^{-1} T^* U \right)^* = U^* (T^*)^* (U^{-1})^* = U^{-1} (T^*)^* U.
		\]
		Poiché $T$ è essenzialmente autoaggiunto, per definizione sappiamo che $T^* = (T^*)^*$. Sostituendo questa uguaglianza nell'equazione sopra:
		\[
		((T')^*)^* = U^{-1} T^* U.
		\]
		Osserviamo che il membro di destra è esattamente l'espressione di $(T')^*$ trovata in precedenza. Dunque:
		\[
		((T')^*)^* = (T')^*.
		\]
		
		Tutte le condizioni della \textbf{Definizione 84} sono soddisfatte per $T'$, che è quindi essenzialmente autoaggiunto.
	\end{sol}
	
	\section*{Traccia dell'Esercizio 1}
	Sia $\HH$ uno spazio di Hilbert e sia $T = T^* \in \BB_\infty(\HH)$ un operatore compatto autoaggiunto. Dato $\psi_0 \in \HH$, si considerino le equazioni:
	\begin{enumerate}
		\item $T\psi = \psi$
		\item $T\psi = \psi + \psi_0$
	\end{enumerate}
	Si mostri che:
	\begin{enumerate}
		\item[(a)] Se l'unica soluzione di (1) è $\psi=0$, allora l'equazione (2) ammette un'unica soluzione.
		\item[(b)] Se l'equazione (1) ammette soluzioni $\psi \neq 0$, allora l'equazione (2) è risolubile se e solo se $\psi_0$ è ortogonale ad ogni soluzione di (1).
	\end{enumerate}
	
	\hrule
	\vspace{0.5cm}
	
	\begin{sol}
		Definiamo l'operatore $S: \HH \to \HH$ come:
		\[
		S \coloneqq T - I.
		\]
		Poiché $T$ è limitato (in quanto compatto) e $I$ è limitato, $S$ è un operatore limitato. Inoltre, poiché $T$ è autoaggiunto ($T=T^*$) e l'identità è autoaggiunta, $S$ è autoaggiunto:
		\[
		S^* = (T-I)^* = T^* - I^* = T - I = S.
		\]
		Le equazioni date possono essere riscritte come:
		\begin{enumerate}
			\item $S\psi = 0$ (Equazione omogenea)
			\item $S\psi = \psi_0$ (Equazione non omogenea, a meno di un segno ininfluente su $\psi_0$)
		\end{enumerate}
		
		\subsection*{Parte (a)}
		\textbf{Ipotesi:} L'unica soluzione della (1) è $\psi=0$. Questo implica che $\Ker(S) = \{0\}$.
		
		\subsubsection*{1. Analisi del Nucleo $\Ker(S)$}
		Dobbiamo formalizzare che $\Ker(S)$ è un sottospazio vettoriale (banale ma necessario per rigore).
		Il nucleo è definito come $\Ker(S) = \{ \psi \in \HH \mid (T-I)\psi = 0 \}$.
		
		\begin{itemize}
			\item \textbf{Contiene lo zero:} $S(0) = T(0) - 0 = 0$, quindi $0 \in \Ker(S)$.
			\item \textbf{Chiusura rispetto alle operazioni lineari:} Siano $\phi_1, \phi_2 \in \Ker(S)$ e $\alpha, \beta \in \mathbb{C}$.
			\[
			S(\alpha \phi_1 + \beta \phi_2) = \alpha S(\phi_1) + \beta S(\phi_2) = \alpha \cdot 0 + \beta \cdot 0 = 0.
			\]
			Pertanto $\alpha \phi_1 + \beta \phi_2 \in \Ker(S)$.
		\end{itemize}
		Dunque $\Ker(S)$ è un sottospazio lineare. Per l'ipotesi di questa sezione, $\Ker(S) = \{0\}$, il che implica che l'operatore $S$ è \textbf{iniettivo}. L'unicità della soluzione, qualora esista, è garantita. Resta da dimostrare l'esistenza (suriettività).
		
		\subsubsection*{2. Dimostrazione della Chiusura del Rango e Suriettività}
		Per dimostrare che l'equazione $S\psi = \psi_0$ ammette soluzione per ogni $\psi_0$, dobbiamo mostrare che $\Ran(S) = \HH$.
		Essendo $S$ autoaggiunto, vale la decomposizione ortogonale:
		\[
		\HH = \overline{\Ran(S)} \oplus \Ker(S).
		\]
		Dato che $\Ker(S) = \{0\}$, segue che $\overline{\Ran(S)} = \HH$. Ovvero, l'immagine è densa.
		Per concludere che $\Ran(S) = \HH$, è necessario e sufficiente dimostrare che $\Ran(S)$ è un insieme \textbf{chiuso}.
		
		\paragraph{Dimostrazione della chiusura di $\Ran(S)$ (Caso Generale).}
		Sia $y_n \in \text{Ran}(S)$, $n \in \mathbb{N}$, e supponiamo che $y_n \to y$ per $n \to +\infty$. Dobbiamo dimostrare che $y \in \text{Ran}(S)$. Per ipotesi:
		\begin{equation}
			y_n = Sx_n = Tx_n - x_n \tag{*}
		\end{equation}
		per una certa successione $\{x_n\}_{n \in \mathbb{N}} \in \HH$. Senza perdita di generalità possiamo assumere $x_n \in \text{Ker}(S)^\perp$, eventualmente eliminando dalla successione la componente che si proietta su $\text{Ker}(S)$.
		L'affermazione è dimostrata se riusciamo a mostrare che la successione $\{x_n\}$ è limitata: infatti, essendo $T$ compatto, esisterà una sottosuccessione $x_{n_k}$ tale che $Ax_{n_k} \to y' \in \HH$ per $k \to \infty$. Sostituendo in (*) concludiamo che $x_{n_k} \to x$ per un certo $x \in \HH$ per $k \to +\infty$. Per la continuità di $T$, $Sx = Tx - x = y$, dunque $y \in \text{Ran}(S)$.
		
		Procederemo per assurdo, assumendo che $\{x_n\}_{n \in \mathbb{N}} \subset \text{Ker}(S)^\perp$ sia illimitata. Quindi esisterebbe una sottosuccessione $x_{n_m}$ con $0 < \|x_{n_m}\| \to +\infty$ per $m \to +\infty$. Poiché i $y_n$ formano una successione convergente, e quindi limitata, dividendo per $\|x_{n_m}\|$ in (*) si ottiene:
		\begin{equation}
			S \frac{x_{n_m}}{\|x_{n_m}\|} = T \frac{x_{n_m}}{\|x_{n_m}\|} - \frac{x_{n_m}}{\|x_{n_m}\|} = \frac{y_{n_m}}{\|x_{n_m}\|} \to \mathbf{0}. \tag{**}
		\end{equation}
		Ma $T$ è compatto e i vettori $\frac{x_{n_m}}{\|x_{n_m}\|}$ sono limitati, quindi possiamo estrarre un'ulteriore sottosuccessione $x_{n_{m_k}}/\|x_{n_{m_k}}\|$ tale che:
		$$
		\frac{x_{n_{m_k}}}{\|x_{n_{m_k}}\|} \to x' \in \HH \quad \text{e} \quad S \frac{x_{n_{m_k}}}{\|x_{n_{m_k}}\|} \to Sx' \quad \text{per } k \to +\infty.
		$$
		Da (**) deduciamo che $x' \in \text{Ker}(S)$. Per ipotesi $\frac{x_{n_{m_k}}}{\|x_{n_{m_k}}\|} \in \text{Ker}(S)^\perp$, e poiché $\text{Ker}(S)^\perp$ è chiuso, allora $x' \in \text{Ker}(S)^\perp$. Di conseguenza $x' \in \text{Ker}(S) \cap \text{Ker}(S)^\perp = \{\mathbf{0}\}$, in contraddizione con
		$$
		\|x'\| = \lim_{k \to +\infty} \frac{\|x_{n_{m_k}}\|}{\|x_{n_{m_k}}\|} = 1.
		$$
		
		\textbf{Conclusione Parte (a):}
		Poiché $\Ran(S)$ è chiuso e $\Ker(S)=\{0\}$, abbiamo:
		\[
		\Ran(S) = (\Ker(S^*))^\perp = (\Ker(S))^\perp = \{0\}^\perp = \HH.
		\]
		L'operatore $S$ è quindi una biiezione su $\HH$. L'equazione (2) ammette una ed una sola soluzione.
		
		\subsection*{Parte (b)}
		\textbf{Ipotesi:} L'equazione (1) ammette soluzioni $\psi \neq 0$.
		Questo significa che $\Ker(S) \neq \{0\}$. $\Ker(S)$ è l'autospazio di $T$ relativo all'autovalore $1$. Poiché $T$ è compatto, questo autospazio ha dimensione finita, ma qui ci basta sapere che non è banale.
		
		Dalla teoria generale (e ricalcando la dimostrazione di chiusura fatta al punto (a), che rimane valida anche se il nucleo non è nullo), sappiamo che $\Ran(S)$ è un sottospazio chiuso di $\HH$ .
		
		Essendo $\Ran(S)$ chiuso, vale esattamente:
		\[
		\Ran(S) = (\Ker(S^*))^\perp.
		\]
		Poiché $S$ è autoaggiunto ($S=S^*$), abbiamo:
		\[
		\Ran(S) = (\Ker(S))^\perp.
		\]
		L'equazione (2), $S\psi = \psi_0$, ha soluzione se e solo se $\psi_0 \in \Ran(S)$.
		In virtù dell'uguaglianza sopra, questo equivale a:
		\[
		\psi_0 \in (\Ker(S))^\perp.
		\]
		Ovvero, $\psi_0$ deve essere ortogonale a ogni vettore del nucleo di $S$.
		Poiché i vettori di $\Ker(S)$ sono esattamente le soluzioni dell'equazione omogenea (1), l'equazione (2) è risolubile se e solo se $\psi_0$ è ortogonale ad ogni soluzione della (1).
		
		\qed
		
	\end{sol}
	
	\section*{Traccia dell'Esercizio 2}
	Sia $g \in L^2([0, 2\pi])$. Definito l'operatore $T: L^2([0, 2\pi]) \to L^2([0, 2\pi])$ come:
	\[
	f \mapsto T(f) \doteq g * f = \int_0^{2\pi} g(x-y)f(y) \, dy
	\]
	si mostri che $T \in \BB_\infty(L^2([0, 2\pi]))$ (ovvero $T$ è compatto) e che le funzioni $e^{inx}$ sono autovettori di $T$.
	
	\hrule
	\vspace{0.5cm}
	
	\begin{sol}
		Per procedere con il dovuto rigore, assumiamo che la funzione $g$, data in $L^2([0, 2\pi])$, sia estesa per periodicità a tutto $\R$. Questo rende ben definita l'espressione $g(x-y)$ per ogni coppia $(x,y)$.
		
		\subsection*{1. Spettro Puntuale: Gli autovettori}
		Siano $\phi_n(x) \doteq e^{inx}$ con $n \in \Z$. Verifichiamo direttamente che questi sono autovettori applicando l'operatore integrale.
		\[
		(T\phi_n)(x) = \int_0^{2\pi} g(x-y) e^{iny} \, dy.
		\]
		Operiamo il cambio di variabile $z = x - y$. Di conseguenza $y = x - z$ e $dy = -dz$.
		Gli estremi di integrazione si trasformano come segue: $y=0 \to z=x$ e $y=2\pi \to z=x-2\pi$.
		\[
		(T\phi_n)(x) = \int_{x}^{x-2\pi} g(z) e^{in(x-z)} (-dz) = e^{inx} \int_{x-2\pi}^{x} g(z) e^{-inz} \, dz.
		\]
		L'integranda $h(z) = g(z)e^{-inz}$ è il prodotto di funzioni $2\pi$-periodiche, ed è quindi essa stessa $2\pi$-periodica. È fatto noto dell'analisi reale che l'integrale di una funzione periodica su un intervallo pari al periodo è invariante per traslazione degli estremi:
		\[
		\int_{x-2\pi}^{x} h(z)\, dz = \int_{0}^{2\pi} h(z)\, dz.
		\]
		Sostituendo, otteniamo:
		\[
		(T\phi_n)(x) = e^{inx} \left( \int_{0}^{2\pi} g(z) e^{-inz} \, dz \right).
		\]
		Definendo lo scalare $\lambda_n \doteq \int_{0}^{2\pi} g(z) e^{-inz} \, dz$, abbiamo dimostrato che:
		\[
		T\phi_n = \lambda_n \phi_n.
		\]
		Dunque, le funzioni $\phi_n$ sono autovettori di $T$ relativi agli autovalori $\lambda_n$.
		
		\subsection*{2. Completezza del sistema ortonormale}
		Per poter analizzare le proprietà spettrali globali dell'operatore, dobbiamo assicurarci che gli autovettori trovati generino l'intero spazio.
		Definiamo il sistema normalizzato:
		\[
		u_n(x) \doteq \frac{1}{\sqrt{2\pi}} e^{inx}, \quad n \in \Z.
		\]
		Il sistema $\{u_n\}_{n \in \Z}$ è chiaramente ortonormale rispetto al prodotto scalare standard di $L^2$: $\inner{u_n}{u_m} = \delta_{nm}$.
		
		Per dimostrarne la completezza, mostriamo che il complemento ortogonale del sottospazio generato da $\{u_n\}$ è il solo vettore nullo.
		Sia $f \in L^2([0, 2\pi])$ tale che $\inner{f}{u_n} = 0$ per ogni $n \in \Z$.
		\[
		\inner{f}{u_n} = \int_0^{2\pi} f(x) \overline{u_n(x)} \, dx = \frac{1}{\sqrt{2\pi}} \int_0^{2\pi} f(x) e^{-inx} \, dx = 0.
		\]
		Gli integrali sopra corrispondono (a meno del fattore di normalizzazione) ai coefficienti di Fourier di $f$, denotati $\hat{f}_n$.
		L'ipotesi $\inner{f}{u_n} = 0, \forall n$ implica $\hat{f}_n = 0, \forall n$.
		Questo può essere detto per l'iniettività di Fourier su $L^2$ enunciata in classe oppure 
		per il \textbf{Teorema di Unicità} della serie di Fourier (conseguenza della densità dei polinomi trigonometrici e della completezza di $L^2$), una funzione $L^2$ con tutti i coefficienti di Fourier nulli è nulla quasi ovunque.
		\[
		f = 0 \quad \text{q.o.}
		\]
		Pertanto, $\{u_n\}_{n \in \Z}$ è una base ortonormale completa (Base di Hilbert) per $\HH$.
		
		\subsection*{3. Compattezza e Classe di Hilbert-Schmidt}
		Per dimostrare che $T$ è compatto ($T \in \BB_\infty(\HH)$), dimostreremo la condizione più forte che $T$ è un operatore di Hilbert-Schmidt.
		Un operatore $T$ è di Hilbert-Schmidt se, data una base ortonormale $\{e_k\}$, la quantità $\|T\|_{2}^2 \doteq \sum_k \norm{T e_k}^2$ è finita.
		
		Scegliamo come base proprio gli autovettori normalizzati $\{u_n\}_{n \in \Z}$.
		Poiché $T u_n = \lambda_n u_n$, abbiamo:
		\[
		\norm{T u_n}^2 = \norm{\lambda_n u_n}^2 = |\lambda_n|^2 \norm{u_n}^2 = |\lambda_n|^2.
		\]
		La norma Hilbert-Schmidt è dunque data dalla serie degli autovalori al quadrato:
		\[
		\|T\|_{2}^2 = \sum_{n \in \Z} |\lambda_n|^2.
		\]
		Analizziamo i termini $\lambda_n$. Dalla definizione data nel punto 1:
		\[
		\lambda_n = \int_0^{2\pi} g(z) e^{-inz} \, dz.
		\]
		Possiamo riscrivere $\lambda_n$ in funzione dei coefficienti di Fourier della funzione $g$ rispetto alla base ortonormale $\{u_n\}$. I coefficienti di Fourier sono $\hat{g}_n = \inner{g}{u_n} = \frac{1}{\sqrt{2\pi}} \int_0^{2\pi} g(z) e^{-inz} \, dz$.
		Risulta evidente che:
		\[
		\lambda_n = \sqrt{2\pi} \, \hat{g}_n.
		\]
		Sostituendo nella somma:
		\[
		\sum_{n \in \Z} |\lambda_n|^2 = \sum_{n \in \Z} \left| \sqrt{2\pi} \, \hat{g}_n \right|^2 = 2\pi \sum_{n \in \Z} |\hat{g}_n|^2.
		\]
		Poiché per ipotesi $g \in L^2([0, 2\pi])$, l'\textbf{Identità di Parseval} garantisce che la somma dei quadrati dei suoi coefficienti di Fourier converga al quadrato della norma della funzione:
		\[
		\sum_{n \in \Z} |\hat{g}_n|^2 = \norm{g}_{L^2}^2 < \infty.
		\]
		Di conseguenza:
		\[
		\|T\|_{2}^2 = 2\pi \norm{g}_{L^2}^2 < \infty.
		\]
		L'operatore $T$ è dunque di Hilbert-Schmidt. Poiché la classe degli operatori di Hilbert-Schmidt è contenuta in quella degli operatori compatti ($\mathcal{B}_{2} \subset \mathcal{B}_\infty$), concludiamo che $T$ è un operatore compatto.
		
		\qed
		
	\end{sol}
	
	\section*{Soluzione Esercizio 2  - Metodo Spettrale}
	
	Sia $g \in L^2([0, 2\pi])$ estesa per periodicità. Analizziamo l'operatore integrale $T: L^2 \to L^2$ definito da $Tf = g * f$.
	
	\subsection*{1. Spettro Puntuale e Diagonalizzazione}
	Consideriamo il sistema ortonormale completo in $L^2([0, 2\pi])$ dato da:
	\[
	u_n(x) = \frac{1}{\sqrt{2\pi}} e^{inx}, \quad n \in \Z.
	\]
	Verifichiamo che questi siano autovettori di $T$. Applicando la definizione:
	\[
	(T u_n)(x) = \int_0^{2\pi} g(x-y) \frac{e^{iny}}{\sqrt{2\pi}} \, dy.
	\]
	Ponendo $z = x-y$ e sfruttando la periodicità delle funzioni in gioco:
	\[
	(T u_n)(x) = \frac{1}{\sqrt{2\pi}} \int_0^{2\pi} g(z) e^{in(x-z)} \, dz = \left( \int_0^{2\pi} g(z) e^{-inz} \, dz \right) \frac{e^{inx}}{\sqrt{2\pi}}.
	\]
	Definiamo la successione degli scalari $\lambda_n$ come:
	\[
	\lambda_n \doteq \int_0^{2\pi} g(z) e^{-inz} \, dz.
	\]
	Abbiamo quindi ottenuto la relazione agli autovalori:
	\[
	T u_n = \lambda_n u_n, \quad \forall n \in \Z.
	\]
	Notiamo che $\lambda_n$ è strettamente legato ai coefficienti di Fourier di $g$. Infatti, denotando $\hat{g}_n = \langle g, u_n \rangle$:
	\[
	\lambda_n = \sqrt{2\pi} \left( \frac{1}{\sqrt{2\pi}} \int_0^{2\pi} g(z) e^{-inz} \, dz \right) = \sqrt{2\pi} \, \hat{g}_n.
	\]
	
	\subsection*{2. Compattezza via Fourier-Plancherel}
	Per analizzare la compattezza di $T$, passiamo allo spazio delle frequenze.
	Sia $\mathcal{F}: L^2([0, 2\pi]) \to \ell^2(\Z)$ l'isomorfismo isometrico di Fourier-Plancherel, che associa a ogni funzione $f$ la successione dei suoi coefficienti di Fourier $\{\hat{f}_n\}_{n \in \Z}$.
	
	Grazie alla diagonalizzazione effettuata al punto precedente, l'operatore $T$ è unitariamente equivalente a un **operatore di moltiplicazione diagonale** $\hat{T}$ su $\ell^2(\Z)$:
	\[
	\hat{T}(\{\hat{f}_n\}_n) = \{ \lambda_n \hat{f}_n \}_n.
	\]
	In altre parole, l'azione dell'operatore sulle componenti di Fourier è una semplice moltiplicazione scalare elemento per elemento.
	
	\textbf{Analisi della successione dei moltiplicatori $\{\lambda_n\}$:}
	Dall'ipotesi $g \in L^2([0, 2\pi])$, per il teorema di Plancherel (o Identità di Parseval), sappiamo che la successione dei coefficienti di Fourier di $g$ è a quadrato sommabile:
	\[
	\sum_{n \in \Z} |\hat{g}_n|^2 = \|g\|_{L^2}^2 < \infty.
	\]
	Poiché $\lambda_n = \sqrt{2\pi} \, \hat{g}_n$, ne consegue immediatamente che anche la successione degli autovalori appartiene a $\ell^2(\Z)$:
	\[
	\sum_{n \in \Z} |\lambda_n|^2 = 2\pi \sum_{n \in \Z} |\hat{g}_n|^2 = 2\pi \|g\|_{L^2}^2 < \infty.
	\]
	
	\subsection*{3. Conclusione}
	Un operatore diagonale su $\ell^2$ definito da una successione $\{\lambda_n\}$ è un operatore di Hilbert-Schmidt se e solo se la successione è in $\ell^2(\Z)$.
	Avendo dimostrato che $\{\lambda_n\} \in \ell^2(\Z)$, concludiamo che:
	\begin{enumerate}
		\item $T$ è un operatore di Hilbert-Schmidt, e quindi
		\item $T$ è un operatore compatto (poiché $\mathcal{B}_2 \subset \mathcal{B}_\infty$).
	\end{enumerate}
	
	\textit{Nota a margine:} Anche senza invocare la classe Hilbert-Schmidt, la condizione necessaria e sufficiente affinché un operatore diagonale sia compatto è che $\lim_{|n|\to\infty} \lambda_n = 0$. Questo è garantito dal fatto che $\lambda_n$ sono coefficienti di Fourier di una funzione $L^2$ (Lemma di Riemann-Lebesgue, o semplice conseguenza della convergenza della serie dei quadrati).
	
	\qed
	
	\section*{Traccia dell'Esercizio 3}
	Sia data una matrice $\sigma \in \mathcal{M}(n; \mathbb{R})$ positiva e sia $\psi \in L^2(\mathbb{R}^n)$. Si mostri che l'operatore
	\[
	L \doteq \text{div}(\sigma \nabla) = \nabla \cdot (\sigma \nabla)
	\]
	è essenzialmente autoaggiunto sullo spazio di Hilbert $\HH$, dove $\nabla$ è l'operatore gradiente.
	
	\hrule
	\vspace{0.5cm}
	
	\begin{sol}
		Per affrontare il problema con il dovuto rigore, è necessario specificare il dominio iniziale dell'operatore. Poiché stiamo trattando operatori differenziali su $\R^n$ non limitati, assumiamo come dominio di definizione lo spazio delle funzioni test a supporto compatto:
		\[
		\Dom(L) = C_0^\infty(\R^n) \subset L^2(\R^n).
		\]
		L'operatore $L$ è simmetrico su questo dominio. Infatti, per ogni $\phi, \psi \in C_0^\infty(\R^n)$, integrando per parti due volte e assumendo che i termini di bordo svaniscano (garantito dal supporto compatto), e sfruttando la simmetria di $\sigma$ (implicita nella definizione di positività per matrici reali in questo contesto, $\sigma_{ij} = \sigma_{ji}$):
		\[
		\inner{L\phi}{\psi} = \int_{\R^n} \nabla \cdot (\sigma \nabla \phi) \bar{\psi} \, dx = - \int_{\R^n} (\sigma \nabla \phi) \cdot \nabla \bar{\psi} \, dx = \int_{\R^n} \phi \overline{\nabla \cdot (\sigma \nabla \psi)} \, dx = \inner{\phi}{L\psi}.
		\]
		Per dimostrare che $L$ è \textbf{essenzialmente autoaggiunto} (ovvero che la sua chiusura $\bar{L}$ è autoaggiunta, $\bar{L} = L^*$), utilizziamo il \textbf{Criterio di Von Neumann}. Dobbiamo mostrare che gli indici di difetto sono nulli, ovvero:
		\[
		\ker(L^* - iI) = \{0\} \quad \text{e} \quad \ker(L^* + iI) = \{0\}.
		\]
		Poiché $L$ è a coefficienti reali, è sufficiente studiare una delle due equazioni, in quanto la coniugazione complessa mappa le soluzioni di una in quelle dell'altra. Cerchiamo quindi le soluzioni distribuzionali $\psi \in L^2(\R^n)$ dell'equazione:
		\[
		L\psi = i\psi \implies \nabla \cdot (\sigma \nabla \psi) - i\psi = 0.
		\]
		
		\subsection*{1. Diagonalizzazione e Cambio di Variabili}
		La matrice $\sigma$ è definita positiva. Per il Teorema Spettrale reale, esiste una matrice ortogonale $O \in O(n)$ tale che:
		\[
		O^T \sigma O = D = \text{diag}(\lambda_1, \dots, \lambda_n), \quad \text{con } \lambda_k > 0.
		\]
		Operiamo un cambio di variabili nello spazio $\R^n$ definito dalla rotazione $y = O^T x$. Poiché $O$ è ortogonale, la trasformazione è un'isometria unitaria su $L^2(\R^n)$ (preserva il prodotto scalare e la norma), quindi non altera le proprietà spettrali (come l'appartenenza a $L^2$).
		Esprimiamo l'operatore differenziale nelle nuove coordinate $y$. Notiamo che $\nabla_x = O \nabla_y$. L'operatore diventa:
		\[
		\nabla_x \cdot (\sigma \nabla_x) = (O \nabla_y)^T \sigma (O \nabla_y) = \nabla_y^T (O^T \sigma O) \nabla_y = \nabla_y^T D \nabla_y = \sum_{k=1}^n \lambda_k \frac{\partial^2}{\partial y_k^2}.
		\]
		L'equazione agli autovalori nelle coordinate ruotate diventa:
		\[
		\sum_{k=1}^n \lambda_k \frac{\partial^2 \psi}{\partial y_k^2} - i \psi(y) = 0.
		\]
		
		\subsection*{2. Analisi delle Soluzioni in $L^2$ tramite Trasformata di Fourier}
		Vogliamo dimostrare che l'unica soluzione $\psi \in L^2(\mathbb{R}^n)$ all'equazione $L\psi = i\psi$ è la soluzione nulla $\psi \equiv 0$.
		Utilizziamo la trasformata di Fourier $\mathcal{F}$, che è un automorfismo unitario su $L^2(\mathbb{R}^n)$ (Teorema di Plancherel).
		
		Riprendiamo l'equazione nelle coordinate diagonali $y$ ottenuta al punto precedente:
		\[
		\sum_{k=1}^n \lambda_k \frac{\partial^2 \psi}{\partial y_k^2} - i \psi(y) = 0.
		\]
		Passando allo spazio delle frequenze $k \in \mathbb{R}^n$, denotiamo $\hat{\psi}(k) = \mathcal{F}[\psi](k)$. Ricordando che l'azione della derivata in Fourier diventa una moltiplicazione (i.e., $\mathcal{F}[\partial_{y_j}^2 \psi] = -k_j^2 \hat{\psi}$), l'equazione differenziale si trasforma nella seguente equazione algebrica:
		\[
		\sum_{k=1}^n \lambda_k (-k_k^2) \hat{\psi}(k) - i \hat{\psi}(k) = 0.
		\]
		Raccogliendo $\hat{\psi}(k)$:
		\[
		-\left( \sum_{k=1}^n \lambda_k k_k^2 + i \right) \hat{\psi}(k) = 0.
		\]
		Il coefficiente moltiplicativo $P(k) = -(\sum \lambda_k k_k^2 + i)$ non si annulla mai per nessun $k \in \mathbb{R}^n$. Infatti, la sua parte immaginaria è costantemente $-i \neq 0$ (inoltre, essendo $\lambda_k > 0$, la parte reale è sempre non positiva).
		Affinché il prodotto sia nullo, deve necessariamente valere:
		\[
		\hat{\psi}(k) = 0 \quad \text{quasi ovunque.}
		\]
		Poiché la trasformata di Fourier è iniettiva su $L^2$, $\hat{\psi} = 0$ implica $\psi = 0$ quasi ovunque in $\mathbb{R}^n$.
		Dunque, $\ker(L^* - iI) = \{0\}$.
		Il procedimento è identico per l'equazione $L\psi = -i\psi$ (il termine immaginario cambia segno ma rimane non nullo).
		
		\subsection*{Conclusione}
		Abbiamo dimostrato che $\ker(L^* \pm iI) = \{0\}$. Poiché gli indici di difetto sono $(0,0)$, per il Criterio di autoaggiunzione essenziale (Criterio di Von Neumann), l'operatore $L$ definito su $C_0^\infty(\mathbb{R}^n)$ è essenzialmente autoaggiunto.
		
		\qed
	\end{sol}
	
\section*{Traccia dell'Esercizio 4}
Sia dato il dominio $I = [0, 1]$ e si consideri il problema di Dirichlet per $\psi : I \to \mathbb{R}$
\[
\begin{cases}
	-\frac{d^2\psi}{dx^2} + \psi = f \\
	\psi|_{\partial I} = 0
\end{cases},
\]
dove $f \in C^\infty(I)$. Sia $H^1(I) \doteq \{ \psi \in L^2(I) \mid \frac{d\psi}{dx} \in L^2(I) \}$ e sia $H^1_0(I) \subset H^1(I)$ il sottospazio di Hilbert delle funzioni $\psi \in L^2(I)$ tali che $\psi|_{\partial I} = 0$.
Si mostri che esiste ed è unica una soluzione \textit{debole} $\psi_0$ del problema di Dirichlet ossia $\psi_0 \in H^1_0(I)$ e
\[
\int_I dx \, \psi_0 h + \int_I dx \, \frac{d\psi_0}{dx} \frac{dh}{dx} = \int_I dx \, f h, \quad \forall h \in H^1_0(I).
\]
[Hint: Si ricorda che $\|\psi\|_{H^1(I)}^2 = \|\psi\|_{L^2(I)}^2 + \|\frac{d\psi}{dx}\|_{L^2(I)}^2$.]
\hrule
\vspace{0.5cm}

\begin{sol}
	La dimostrazione dell'esistenza e unicità della soluzione debole si basa sull'applicazione del Teorema di Rappresentazione di Riesz nello spazio di Hilbert appropriato. Procediamo identificando la struttura geometrica indotta dall'equazione differenziale.
	
	\subsection*{1. Struttura dello Spazio di Hilbert}
	Dal testo, consideriamo lo spazio $H^1_0(I)$, definito come il sottospazio delle funzioni in $H^1(I)$ che si annullano al bordo $\partial I$.
	L'Hint ci fornisce la norma su $H^1(I)$:
	\[
	\norm{\psi}_{H^1}^2 = \norm{\psi}_{L^2}^2 + \norm{\frac{d\psi}{dx}}_{L^2}^2 = \int_I |\psi|^2 \, dx + \int_I \left|\frac{d\psi}{dx}\right|^2 \, dx.
	\]
	Questa norma è indotta dal seguente \textbf{prodotto scalare}:
	\[
	\inner{u}{v}_{H^1} \doteq \int_I u(x)v(x) \, dx + \int_I \frac{du}{dx}(x)\frac{dv}{dx}(x) \, dx.
	\]
	Poiché $H^1(I)$ è uno spazio di Hilbert e $H^1_0(I)$ è un suo sottospazio chiuso (definito da condizioni puntuali al bordo, che sono preservate dal limite in norma $H^1$ per i teoremi di traccia o per definizione di chiusura), concludiamo che $(H^1_0(I), \inner{\cdot}{\cdot}_{H^1})$ è esso stesso uno spazio di Hilbert.
	
	\subsection*{2. Derivazione della Formulazione Debole}
	Data l'equazione differenziale classica:
	\[
	-\frac{d^2\psi}{dx^2} + \psi = f,
	\]
	moltiplichiamo ambo i membri per una generica funzione test $h \in H^1_0(I)$ e integriamo sul dominio $I$:
	\[
	\int_I \left( -\frac{d^2\psi}{dx^2} + \psi \right) h \, dx = \int_I f h \, dx.
	\]
	Sfruttando la linearità dell'integrale e applicando l'integrazione per parti al termine con la derivata seconda:
	\[
	\int_I -\frac{d^2\psi}{dx^2} h \, dx = \underbrace{\left[ -\frac{d\psi}{dx} h \right]_{\partial I}}_{=0} + \int_I \frac{d\psi}{dx} \frac{dh}{dx} \, dx.
	\]
	Il termine di bordo si annulla rigorosamente perché $h \in H^1_0(I)$, e dunque per definizione $h|_{\partial I} = 0$.
	L'equazione integrale risultante è esattamente quella richiesta dalla traccia:
	\[
	\int_I \psi h \, dx + \int_I \frac{d\psi}{dx} \frac{dh}{dx} \, dx = \int_I f h \, dx.
	\]
	
	\subsection*{3. Applicazione del Teorema di Riesz}
	Rileggiamo l'equazione debole alla luce della definizione di prodotto scalare data al punto 1.
	L'equazione si può riscrivere come:
	\[
	\inner{\psi}{h}_{H^1} = \int_I f h \, dx, \quad \forall h \in H^1_0(I).
	\]
	Definiamo il funzionale lineare $F: H^1_0(I) \to \mathbb{R}$ come $F(h) = \int_I f h \, dx$.
	
	Per applicare il Teorema di Riesz, dobbiamo dimostrare che $F$ è un funzionale \textbf{continuo} (limitato) rispetto alla norma $H^1$.
	\begin{enumerate}
		\item Poiché $f \in C^\infty(I)$ e il dominio $I$ è compatto, $f \in L^2(I)$, quindi $\norm{f}_{L^2} < \infty$.
		\item Applichiamo la disuguaglianza di Cauchy-Schwarz in $L^2$:
		\[
		|F(h)| = \left| \int_I f h \, dx \right| \le \norm{f}_{L^2} \norm{h}_{L^2}.
		\]
		\item Osserviamo, dalla definizione della norma data nell'Hint, che:
		\[
		\norm{h}_{H^1}^2 = \norm{h}_{L^2}^2 + \norm{h'}_{L^2}^2 \implies \norm{h}_{L^2}^2 \le \norm{h}_{H^1}^2 \implies \norm{h}_{L^2} \le \norm{h}_{H^1}.
		\]
		\item Combinando le disuguaglianze:
		\[
		|F(h)| \le \norm{f}_{L^2} \norm{h}_{H^1}.
		\]
	\end{enumerate}
	Essendo $\norm{f}_{L^2}$ una costante, $F$ è limitato.
	Per il \textbf{Teorema di Rappresentazione di Riesz}, in uno spazio di Hilbert ogni funzionale lineare continuo può essere rappresentato univocamente da un elemento dello spazio tramite il prodotto scalare.
	Esiste quindi un unico $\psi_0 \in H^1_0(I)$ tale che:
	\[
	\inner{\psi_0}{h}_{H^1} = F(h) \quad \forall h \in H^1_0(I).
	\]
	Questa uguaglianza coincide esattamente con la tesi da dimostrare.
	
	\qed
\end{sol}
	
	\section*{Traccia dell'Esercizio 5 $\psi(x_0)=0$}
	Sia $I = [0,1]$. Si consideri l'equazione differenziale:
	\[
	-\frac{d\psi}{dx} + \psi = f, \quad \text{con } f \in L^2(I).
	\]
	Si definisca la mappa $K: L^2(I) \to L^2(I)$ tale che $\psi = K(f)$ è soluzione dell'equazione. Si discuta se $K$ è limitato, compatto e/o positivo.
	
	\hrule
	\vspace{0.5cm}
	
	\begin{sol}
		\subsection*{1. Condizione per la Linearità}
	Per stabilire quale condizione al contorno rende la mappa $K: f \mapsto \psi$ un operatore lineare, analizziamo la struttura della soluzione generale dell'equazione differenziale $-\psi' + \psi = f$.
	La soluzione può essere scritta come somma di una soluzione particolare lineare rispetto a $f$ (diciamo $K_0 f$, ad esempio l'integrale con estremo fisso) e della soluzione generale dell'omogenea:
	\[
	\psi(x; f) = (K_0 f)(x) + C e^x.
	\]
	La costante $C$ è determinata dalla condizione al contorno. Affinché $K$ sia un operatore lineare, deve soddisfare la proprietà di omogeneità $K(\lambda f) = \lambda K(f)$ per ogni $\lambda \in \mathbb{C}$.
	Sostituendo:
	\begin{align*}
		K(\lambda f) &= \lambda (K_0 f) + C e^x \\
		\lambda K(f) &= \lambda \left( (K_0 f) + C e^x \right) = \lambda (K_0 f) + \lambda C e^x.
	\end{align*}
	Uguagliando le due espressioni, otteniamo la condizione necessaria:
	\[
	C e^x = \lambda C e^x \implies C(1 - \lambda) = 0 \quad \forall \lambda.
	\]
	Questo implica necessariamente $C=0$. Pertanto, la condizione al contorno fissata a priori deve essere tale da annullare identicamente la componente omogenea. Se fissiamo la condizione in un punto $x_0$, dobbiamo avere:
	\[
	\psi(x_0) = 0 \implies C e^{x_0} = 0 \implies C=0.
	\]
	Concludiamo che solo una condizione al contorno \textbf{omogenea} garantisce la linearità dell'operatore $K$.
		
		\subsection*{2. Costruzione dell'Operatore e Kernel}
		Risolviamo l'equazione con il metodo della variazione delle costanti o fattore integrante, imponendo $\psi(x_0)=0$.
		L'equazione $\psi' - \psi = -f$ moltiplicata per $e^{-x}$ diventa $\frac{d}{dx}(e^{-x}\psi) = -e^{-x}f$.
		Integrando tra $x_0$ e $x$:
		\[
		e^{-x}\psi(x) - e^{-x_0}\underbrace{\psi(x_0)}_{=0} = -\int_{x_0}^x e^{-y}f(y) \, dy.
		\]
		Da cui otteniamo la forma esplicita dell'operatore:
		\[
		(Kf)(x) = -\int_{x_0}^x e^{x-y}f(y) \, dy = \int_0^1 A(x,y) f(y) \, dy.
		\]
		Il kernel integrale $A(x,y)$ è definito come:
		\[
		A(x,y) = \begin{cases} -e^{x-y} & \text{se } y \text{ è compreso tra } x_0 \text{ e } x \\ 0 & \text{altrove}. \end{cases}
		\]
		
		\subsection*{3. Limitatezza e Compattezza}
		Un operatore integrale $K$ su $L^2(I)$ è \textbf{compatto} (e quindi limitato) se è di classe \textit{Hilbert-Schmidt}, condizione equivalente all'appartenenza del kernel a $L^2(I \times I)$.
		Mostriamo che la definizione spettrale di operatore Hilbert-Schmidt coincide con la condizione $L^2$ sul nucleo integrale.
		Sia $\{e_n\}_{n \in \mathbb{N}}$ una base ortonormale di $L^2(I)$. La definizione è:
		\[
		\norm{K}_{HS}^2 = \sum_{n=1}^\infty \norm{K e_n}^2.
		\]
		L'azione dell'operatore integrale su un elemento della base è data da:
		\[
		(K e_n)(x) = \int_I A(x,y) e_n(y) \, dy = \langle \overline{A(x,\cdot)}, e_n \rangle_{L^2_y},
		\]
		dove l'integrale è interpretato come il prodotto scalare rispetto alla variabile $y$ tra la funzione coniugata del kernel e il vettore di base.
		Sostituendo nella definizione di norma e scambiando la serie con l'integrale in $dx$ (giustificato dalla non-negatività dei termini o dal teorema di Beppo Levi):
		\[
		\norm{K}_{HS}^2 = \int_I dx \left( \sum_{n=1}^\infty \left| \langle \overline{A(x,\cdot)}, e_n \rangle \right|^2 \right).
		\]
		Riconosciamo nella parentesi l'\textbf{Identità di Parseval}, la quale afferma che la somma dei moduli quadri dei coefficienti di Fourier eguaglia la norma quadra della funzione:
		\[
		\sum_{n=1}^\infty \left| \langle \overline{A(x,\cdot)}, e_n \rangle \right|^2 = \norm{\overline{A(x,\cdot)}}_{L^2_y}^2 = \int_I |A(x,y)|^2 \, dy.
		\]
		Sostituendo questo risultato nell'integrale esterno, otteniamo l'equivalenza cercata:
		\[
		\norm{K}_{HS}^2 = \int_I dx \int_I dy \, |A(x,y)|^2 = \norm{A}_{L^2(I \times I)}^2.
		\]
		Dobbiamo verificare che:
		\[
		\norm{K}_{HS}^2 = \int_0^1 \int_0^1 |A(x,y)|^2 \, dy \, dx < \infty.
		\]
		Sostituendo l'espressione del kernel:
		\[
		\norm{K}_{HS}^2 = \int_0^1 dx \left| \int_{x_0}^x e^{2(x-y)} \, dy \right|.
		\]
		Calcoliamo l'integrale interno:
		\[
		\int_{x_0}^x e^{2x}e^{-2y} \, dy = e^{2x} \left[ -\frac{e^{-2y}}{2} \right]_{x_0}^x = \frac{e^{2x}}{2} (e^{-2x_0} - e^{-2x}) = \frac{1}{2} (e^{2(x-x_0)} - 1).
		\]
		Poiché stiamo valutando il modulo (o considerando l'orientamento degli estremi di integrazione), la funzione integranda in $x$ è limitata e continua sull'intervallo compatto $[0,1]$.
		Non è necessario calcolare il valore numerico esatto: essendo l'integrale di una funzione continua su un dominio limitato, il risultato è certamente finito.
		\[
		\norm{K}_{HS}^2 < \infty \implies K \text{ è compatto (e limitato)}.
		\]
		
		\subsection*{4. Positività}
		Verifichiamo se $K$ è un operatore positivo, ovvero se $\inner{f}{Kf} \ge 0$ per ogni $f$.
		Scegliamo la funzione test $f(x) = 1$ e consideriamo il caso standard $x_0 = 0$ (o un $x_0$ sufficientemente piccolo).
		La soluzione è:
		\[
		(Kf)(x) = -\int_{0}^x e^{x-y} \, dy = 1 - e^x.
		\]
		Il prodotto scalare risulta:
		\[
		\inner{1}{K1} = \int_0^1 (1 - e^x) \, dx = \left[ x - e^x \right]_0^1 = (1-e) - (-1) = 2 - e.
		\]
		Poiché $e > 2$, si ha $\inner{1}{K1} < 0$.
		L'operatore \textbf{non è positivo}.
		
	\subsection*{5. Esempio di Positività}
	L'operatore $K$ può risultare positivo modificando la condizione al contorno. Consideriamo il problema con condizione al bordo destro:
	\[
	\begin{cases}
		-\psi' + \psi = f \\
		\psi(1) = 0
	\end{cases}
	\]
	Valutiamo la forma quadratica $\langle f, Kf \rangle$. Poiché $Kf = \psi$ e $f = -\psi' + \psi$, abbiamo:
	\[
	\langle f, Kf \rangle = \langle -\psi' + \psi, \psi \rangle = \int_0^1 (-\psi'(x) + \psi(x)) \overline{\psi(x)} \, dx.
	\]
	Assumendo funzioni reali per semplicità (il risultato si estende al caso complesso):
	\[
	\int_0^1 (-\psi'\psi + \psi^2) \, dx = -\int_0^1 \psi \psi' \, dx + \norm{\psi}^2.
	\]
	Integriamo per parti il primo termine:
	\[
	-\int_0^1 \psi \psi' \, dx = -\frac{1}{2} \int_0^1 \frac{d}{dx}(\psi^2) \, dx = -\frac{1}{2} \left[ \psi(x)^2 \right]_0^1 = -\frac{1}{2} (\psi(1)^2 - \psi(0)^2).
	\]
	Imponendo la condizione al contorno scelta $\psi(1)=0$, otteniamo:
	\[
	-\frac{1}{2} (0 - \psi(0)^2) = \frac{1}{2} \psi(0)^2 \ge 0.
	\]
	Ricostruendo l'espressione completa:
	\[
	\langle f, Kf \rangle = \frac{1}{2} \psi(0)^2 + \norm{\psi}_{L^2}^2.
	\]
	Essendo somma di termini non negativi, risulta $\langle f, Kf \rangle \ge 0$ per ogni $f$. In questo caso specifico, $K$ è un \textbf{operatore positivo}.
		
		\qed
	\end{sol}
	
	\section*{Traccia dell'Esercizio 5 Hard}
	Sia $I = [0,1]$. Si consideri l'equazione differenziale:
	\[
	-\frac{d\psi}{dx} + \psi = f, \quad \text{con } f \in L^2(I).
	\]
	Si definisca la mappa $K: L^2(I) \to L^2(I)$ tale che $\psi = K(f)$ è soluzione dell'equazione. Si discuta se $K$ è limitato, compatto e/o positivo.
	
	\hrule
	\vspace{0.5cm}
	
	\begin{sol}
	L'operatore K descritto nel testo del problema non è ben definito, in quanto l'operatore 
	$$D : H^1(I) \to L^2(I) \quad \quad \quad \psi \mapsto -\frac{d\psi}{dx} + \psi$$
	ha kernel non banale, composto da tutte le funzioni del tipo $A e^x$. Per questo motivo, se l'operatore K dovesse definire la soluzione dell'equazione, $K(0) = Ae^x$ ma questo non lo renderebbe lineare se non per $A=0$. Per rendere tutto il più generale possibile, quindi dobbiamo scegliere un qualsiasi \textbf{funzionale lineare ovunque definito} $C: L^2(I) \to \C$ e, data una $f \in L^2(I)$ con cui risolvere il problema, definire un insieme e degli operatori derivata con condizione iniziale
	\[
		S_C = \{ \psi \in  H^1(I)  | \psi(0) = C(f)\} \quad \quad D_C: S_C \to L^2(I)
	\]
	a questo punto possiamo finalmente scrivere la forma degli opreatori $K_C$ dipendenti dalla condizione iniziale e la cui forma analitica è data dall'integrale di Volterra
	\[
		K_C: L^2(I) \to L^2(I) \quad \quad \quad K_C(f) := e^x\left(C(f) - \int_0^x f(y) e^{-y} dy\right)
	\]
	$K_C$ ora è un buon operatore perchè è lineare, ovunque definito e vale che $K_C(D_C(\psi)) = \psi$ con $\psi \in S_C$
	
	\begin{align*}
		K_C(-\psi' + \psi) &= e^x\left( C(f) + \int_0^x \frac{d \psi}{dy} e^{-y} dy -  \int_0^x \psi e^{-y} dy \right) \\&= e^x C(f) + e^x (\psi(y)e^{-y} |_0^x)+ e^x\int_0^x \psi e^{-y} dy -  e^x\int_0^x \psi e^{-y} dy \\ &= e^x C(f) + \psi(x)e^{-x}e^x - \psi(0)e^x \\&= e^x(C(f)-\psi(0)) + \psi(x) \\&= \psi(x)
	\end{align*}
	dove nell'ultimo passaggio abbiamo usato che $\psi \in S_C$. Perfetto, ora possiamo procdere.
	L'operatore descritto è somma di due componenti 
	\[
		K_C(f) = e^x C(f) - \int_0^x f(y) e^{x-y} dy = e^x C(f) - \int_0^1 f(y) \Theta(x-y) e^{x-y} dy = e^x C(f) - f \star (\Theta(x)e^x)
	\]
	possiamo chiamare
	\[
		K_C(f)  = R_C(f) - T(f)
	\]
	\subsection*{Analisi della convoluzione in $L^2([0,1])$ come in 2}
	
	Definiamo il sistema ortonormale in $L^2([0,1])$:
	\[
	u_n(x) \doteq e^{2\pi i n x}, \quad n \in \mathbb{Z}.
	\]
	Il sistema $\{u_n\}_{n \in \mathbb{Z}}$ è chiaramente ortonormale rispetto al prodotto scalare standard: $\langle u_n, u_m \rangle = \delta_{nm}$.
	
	Per dimostrarne la completezza, mostriamo che il complemento ortogonale del sottospazio generato da $\{u_n\}$ contiene solo il vettore nullo.
	Sia $f \in L^2([0,1])$ tale che $\langle f, u_n \rangle = 0$ per ogni $n \in \mathbb{Z}$.
	\[
	\langle f, u_n \rangle = \int_0^{1} f(x) \overline{u_n(x)} \, dx = \int_0^{1} f(x) e^{-2\pi i n x} \, dx = 0.
	\]
	Gli integrali sopra corrispondono esattamente ai coefficienti di Fourier di $f$, denotati con $\hat{f}_n$.
	L'ipotesi $\langle f, u_n \rangle = 0, \forall n$ implica $\hat{f}_n = 0, \forall n$.
	Per il \textbf{Teorema di Unicità} della serie di Fourier (conseguenza della densità dei polinomi trigonometrici e della completezza di $L^2$), una funzione $L^2$ con tutti i coefficienti di Fourier nulli è nulla quasi ovunque.
	\[
	f = 0 \quad \text{q.o.}
	\]
	Pertanto, $\{u_n\}_{n \in \mathbb{Z}}$ è una base ortonormale completa (Base di Hilbert) per $\mathcal{H}$.
	\newline
	Per dimostrare che l'operatore $T$ è \textbf{compatto} ($T \in \mathcal{B}_\infty(\mathcal{H})$), verifichiamo la condizione più forte che $T$ sia un operatore di Hilbert-Schmidt.
	Un operatore $T$ è di Hilbert-Schmidt se, data una base ortonormale $\{e_k\}$, la quantità $\|T\|_{HS}^2 \doteq \sum_k \|T e_k\|^2$ è finita.
	
	Scegliamo come base proprio gli autovettori $\{u_n\}_{n \in \mathbb{Z}}$.
	Poiché $T u_n = \lambda_n u_n$, abbiamo:
	\[
	\|T u_n\|^2 = \|\lambda_n u_n\|^2 = |\lambda_n|^2 \|u_n\|^2 = |\lambda_n|^2.
	\]
	La norma Hilbert-Schmidt è dunque data dalla serie degli autovalori al quadrato:
	\[
	\|T\|_{HS}^2 = \sum_{n \in \mathbb{Z}} |\lambda_n|^2.
	\]
	Analizziamo i termini $\lambda_n$. Dalla definizione dell'operatore di convoluzione su $[0,1]$, gli autovalori sono dati dai coefficienti di Fourier del nucleo $g$:
	\[
	\lambda_n = \int_0^{1} g(z) e^{-2\pi i n z} \, dz = \hat{g}_n.
	\]
	l'autovalore coincide esattamente con il coefficiente di Fourier $\hat{g}_n$.
	
	Sostituendo nella somma:
	\[
	\sum_{n \in \mathbb{Z}} |\lambda_n|^2 = \sum_{n \in \mathbb{Z}} |\hat{g}_n|^2.
	\]
	Poiché per ipotesi $g \in L^2([0, 1])$, l'\textbf{Identità di Parseval} garantisce che la somma dei quadrati dei suoi coefficienti di Fourier converga al quadrato della norma della funzione:
	\[
	\sum_{n \in \mathbb{Z}} |\hat{g}_n|^2 = \|g\|_{L^2}^2 < \infty.
	\]
	Di conseguenza:
	\[
	\|T\|_{HS}^2 = \|g\|_{L^2}^2 < \infty.
	\]
	L'operatore $T$ è dunque di Hilbert-Schmidt. Poiché la classe degli operatori di Hilbert-Schmidt è contenuta in quella degli operatori compatti ($\mathcal{B}_{HS} \subset \mathcal{B}_\infty$), concludiamo che $T$ è un operatore compatto.
	
	\subsection*{Analisi dell'operatore risolvente $K$}
	
	\subsubsection*{1. Limitatezza}
	L'operatore $K_C$ è limitato se e solo se il funzionale $C$ è limitato.
	\begin{itemize}
		\item[$\Rightarrow$] Se $C$ è limitato, allora per la disuguaglianza triangolare:
		\[
		\|K_Cf\|_{L^2} \le \|T f\|_{L^2} + |C(f)| \|e^x\|_{L^2}.
		\]
		Essendo $T$ limitato e $C$ limitato per ipotesi, $K$ è limitato.
		\item[$\Leftarrow$] Se $K$ è limitato, allora $R_C = K - T$ è differenza di operatori limitati, dunque è limitato. Poiché $R_C(f) = C(f)e^x$, la limitatezza di $R$ implica necessariamente la limitatezza del funzionale $C$.
	\end{itemize}
	
	\subsubsection*{2. Compattezza}
	Assumendo $C$ limitato (condizione necessaria per la limitatezza di $K$), l'operatore $K$ risulta sempre compatto.
	\begin{itemize}
		\item L'operatore $T$ è di Hilbert-Schmidt, pertanto, $T$ è compatto.
		\item L'operatore $R_C(f) = C(f)e^x$ ha immagine unidimensionale, generata dal vettore $e^x$. Essendo un operatore di rango finito limitato, $R_C$ è compatto. (Da esercitazione di prof. Costeri)
	\end{itemize}
	Poiché lo spazio degli operatori compatti è un sottospazio vettoriale, $K_C = T + R_C$ è compatto.
	Per il Teorema di Rappresentazione di Riesz, possiamo esplicitare la struttura di $R_C$: esiste un'unica $g \in L^2(I)$ tale che $C(f) = \langle g, f \rangle$, da cui $R_C(f) = R_g(f) = e^x\langle g, f \rangle$.
	
	\subsubsection*{3. Positività}
	L'operatore $K_C$ non è, in generale, positivo. Consideriamo il caso $\psi(0)=0$ (che implica $C \equiv 0$). Valutiamo la parte reale della forma quadratica associata nel campo complesso.
	
	Consideriamo il prodotto scalare standard in $L^2$:
	\[
	\langle K_0f, f \rangle = \int_0^1 (-\psi'(x)+\psi(x)) \overline{\psi(x)} \, dx.
	\]
	Sfruttiamo l'identità per la derivata del modulo quadro: $\frac{d}{dx}|\psi|^2 = \psi' \bar{\psi} + \psi \bar{\psi}' = 2 \Re(\psi' \bar{\psi})$.
	Quindi, integrando per parti o usando la suddetta identità:
	\[
	\Re \int_0^1 -\psi'(x)\overline{\psi(x)} \, dx = -\frac{1}{2} \int_0^1 \frac{d}{dx}|\psi(x)|^2 \, dx = -\frac{1}{2} \left( |\psi(1)|^2 - |\psi(0)|^2 \right).
	\]
	Sostituendo nell'espressione della forma quadratica otteniamo:
	\begin{equation}
		\Re \langle K_0f, f \rangle = \|\psi\|_{L^2}^2 - \frac{1}{2}|\psi(1)|^2 + \frac{1}{2}|\psi(0)|^2.
	\end{equation}
	Essendo nel caso $\psi(0)=0$, l'espressione si riduce a:
	\[
	\Re \langle K_0f, f \rangle = \|\psi\|_{L^2}^2 - \frac{1}{2}|\psi(1)|^2.
	\]
	
	Per mostrare che l'operatore non è positivo, scegliamo la funzione test (reale, valida anche in ambito complesso) $\psi(x) = x$, che implica $f(x)=x-1$. Si ottiene:
	\[
	\Re \langle K_0f, f \rangle = \int_0^1 |x|^2 \, dx - \frac{1}{2}|1|^2 = \frac{1}{3} - \frac{1}{2} = -\frac{1}{6} < 0.
	\]
	Esistono quindi vettori per cui la forma quadratica assume valori negativi, violando la condizione di positività.
	\newline
	Per costruire un operatore $K_C$ positivo, osserviamo l'espressione generale ricavata sopra:
	\begin{equation}
		\Re \langle K_Cf, f \rangle = \|\psi\|_{L^2}^2 + \frac{1}{2}|\psi(0)|^2 - \frac{1}{2}|\psi(1)|^2.
	\end{equation}
	Per garantire la positività è sufficiente eliminare il termine negativo imponendo la condizione al bordo $\psi(1) = 0$.
	Restringiamoci dunque alla classe dei funzionali lineari che impongono tale condizione, ad esempio:
	\[
	C(f) = \int_0^1 e^{-y}f(y) \, dy.
	\]
	In questo caso, la forma quadratica diventa $\Re \langle K_Cf, f \rangle = \|\psi\|^2 + \frac{1}{2}|\psi(0)|^2 \ge 0$ per ogni $f \neq 0$.
	
	L'analisi completa per condizioni necessarie e sufficienti più generali risulta complessa e non banale al di fuori di queste costruzioni dirette.
	\qed
	\end{sol}
	
	\section*{Traccia dell'Esercizio 6}
	Su $L^2(\R)$ si consideri l'operatore differenziale:
	\[
	T = -\frac{d^2}{dx^2} + x^2 + i\frac{d}{dx}.
	\]
	Si discuta se $T$ ammette estensioni autoaggiunte calcolando eventualmente gli indici di difetto.
	
	\hrule
	\vspace{0.5cm}
	
	\begin{sol}
		Per analizzare le proprietà di autoaggiunzione di $T$, cerchiamo di ricondurlo a una forma canonica nota tramite una trasformazione unitaria. L'operatore ricorda molto l'Hamiltoniana dell'Oscillatore Armonico unidimensionale ($H_{HO} = -\frac{d^2}{dx^2} + x^2$), con l'aggiunta di un termine del primo ordine.
		
		\subsection*{1. Completamento del Quadrato}
		Riscriviamo l'operatore utilizzando l'operatore momento $p = -i \frac{d}{dx}$ (in unità con $\hbar=1$). Notiamo che $i\frac{d}{dx} = -p$.
		\[
		T = p^2 + x^2 - p.
		\]
		L'idea è di "completare il quadrato" per la parte dipendente dal momento, trattando $p$ come una variabile algebrica (lecito poiché stiamo cercando una trasformazione unitaria che agisce come una traslazione nello spazio dei momenti).
		Osserviamo che:
		\[
		\left(p - \frac{1}{2}\right)^2 = p^2 - p + \frac{1}{4}.
		\]
		Pertanto possiamo scrivere:
		\[
		T = \left(p - \frac{1}{2}\right)^2 + x^2 - \frac{1}{4}.
		\]
		
		\subsection*{2. Equivalenza Unitaria}
		Vogliamo eliminare lo shift $-1/2$ nell'operatore momento. Sappiamo dalla meccanica quantistica che l'operatore di posizione $x$ è il generatore delle traslazioni nello spazio dei momenti.
		Consideriamo l'operatore unitario $U: L^2(\R) \to L^2(\R)$ definito dalla moltiplicazione per una fase (trasformazione di gauge):
		\[
		(U\psi)(x) = e^{i \frac{1}{2} x} \psi(x).
		\]
		L'aggiunto è $(U^\dagger \psi)(x) = e^{-i \frac{1}{2} x} \psi(x)$.
		Calcoliamo come trasforma l'operatore momento $p$:
		\begin{align*}
			(U^\dagger p U \psi)(x) &= e^{-ix/2} \left( -i \frac{d}{dx} \right) \left( e^{ix/2} \psi(x) \right) \\
			&= e^{-ix/2} \left[ -i \left( \frac{i}{2} e^{ix/2} \psi(x) + e^{ix/2} \psi'(x) \right) \right] \\
			&= e^{-ix/2} e^{ix/2} \left( \frac{1}{2} \psi(x) - i \psi'(x) \right) \\
			&= \left( \frac{1}{2} + p \right) \psi(x).
		\end{align*}
		Quindi $U^\dagger p U = p + \frac{1}{2}$, oppure equivalentemente $U \left(p + \frac{1}{2}\right) U^\dagger = p$.
		Applichiamo questa trasformazione all'operatore $T$:
		\begin{align*}
			\tilde{T} \doteq U T U^\dagger &= U \left[ \left(p - \frac{1}{2}\right)^2 + x^2 - \frac{1}{4} \right] U^\dagger \\
			&= \left[ U \left(p - \frac{1}{2}\right) U^\dagger \right]^2 + U x^2 U^\dagger - \frac{1}{4}.
		\end{align*}
		Poiché $U$ è funzione solo di $x$, commuta con $x^2$. Inoltre, dall'inverso della relazione trovata sopra ($p \to p-1/2$ sotto l'azione di $U^\dagger \cdot U$), il termine al quadrato diventa semplicemente $p^2$.
		Formalmente:
		\[
		U \left( -i\frac{d}{dx} - \frac{1}{2} \right) U^\dagger = -i\frac{d}{dx}.
		\]
		Dunque l'operatore trasformato è:
		\[
		\tilde{T} = p^2 + x^2 - \frac{1}{4} = -\frac{d^2}{dx^2} + x^2 - \frac{1}{4}.
		\]
		
		\subsection*{3. Analisi dell'Operatore Trasformato}
		L'operatore $\tilde{T}$ è (a meno della costante additiva $-1/4$, che non influenza le proprietà di dominio o autoaggiunzione per la nota qui sotto) l'Hamiltoniana dell'Oscillatore Armonico Quantistico:
		\[
		H_{HO} = -\frac{d^2}{dx^2} + x^2.
		\]
		È un risultato classico e fondamentale (dimostrabile ad esempio notando che ha uno spettro discreto completo di autofunzioni in $L^2(\R)$, le funzioni di Hermite) che l'Oscillatore Armonico definito su $C_c^\infty(\R)$ (o sullo spazio di Schwartz $\mathcal{S}(\R)$) è essenzialmente autoaggiunto.
		Ciò significa che la sua chiusura $\overline{H}_{HO}$ è autoaggiunta e unica.
		
		Gli indici di difetto di un operatore essenzialmente autoaggiunto sono $(0, 0)$.
		
		\paragraph{Nota: Autoaggiunzione di $S = T - I$.}
		Verifichiamo esplicitamente che $S = T - I$ è autoaggiunto usando la definizione.
		Per ogni $\phi, \psi \in \HH$:
		\[
		\inner{\phi}{S\psi} = \inner{\phi}{(T-I)\psi} = \inner{\phi}{T\psi} - \inner{\phi}{\psi}.
		\]
		Poiché $T=T^*$ per ipotesi, $\inner{\phi}{T\psi} = \inner{T\phi}{\psi}$. Inoltre, banalmente $\inner{\phi}{\psi} = \inner{I\phi}{\psi}$.
		Quindi:
		\[
		\inner{\phi}{S\psi} = \inner{T\phi}{\psi} - \inner{I\phi}{\psi} = \inner{(T-I)\phi}{\psi} = \inner{S\phi}{\psi}.
		\]
		Ciò prova che $S^* = S$.
		
		\subsection*{Conclusione}
		Poiché $T$ è unitariamente equivalente a un operatore essenzialmente autoaggiunto ($\tilde{T}$), anche $T$ è essenzialmente autoaggiunto sul dominio iniziale delle funzioni test. Il motivo è l'esercitazione 2 della prof.Costeri nella quale dice che 
		\[
			\overline{T'} = U \overline{T} U^\star \quad \quad \text{e} \quad \quad T'^\star = U^\star T^\star U
		\]
		il motivo è perchè U è unitario quindi limitato.
		\begin{itemize}
			\item $T$ ammette un'unica estensione autoaggiunta (la sua chiusura $\bar{T}$).
			\item Gli indici di difetto sono $n_+ = n_- = 0$.
		\end{itemize}
		
		\begin{oss}[Nota sul Metodo della Coniugazione]
			Si poteva anche osservare che, pur non essendo $T$ reale ($T \ne \bar{T}$), esso commuta con l'operatore anti-unitario $\mathcal{J} = \mathcal{P}\mathcal{C}$, dove $\mathcal{P}$ è la parità ($x \to -x$) e $\mathcal{C}$ la coniugazione complessa. Infatti:
			\[
			\mathcal{P}\mathcal{C} \left( -\partial_x^2 + x^2 + i\partial_x \right) (\mathcal{P}\mathcal{C})^{-1} = -\partial_x^2 + (-x)^2 - i(-\partial_x) = T.
			\]
			La commutazione con una coniugazione antiunitaria garantisce $n_+ = n_-$, assicurando l'esistenza di estensioni autoaggiunte, ma non la loro unicità (essenziale autoaggiunzione). Il metodo della trasformazione unitaria è quindi più forte in questo contesto.
		\end{oss}
		
		\qed
	\end{sol}
	
	\section*{Traccia dell'Esercizio 7}
	Sia $\HH$ uno spazio di Hilbert separabile e si consideri l'equazione:
	\[
	T\psi = \lambda \psi + f,
	\]
	con $T = T^* \in \BB_\infty(\HH)$ (operatore compatto autoaggiunto), $\lambda \in \C \setminus \{0\}$ e $f \in \HH$.
	Si provi che, se $\lambda$ è autovalore di $T$, allora l'equazione ha infinite soluzioni (sottointendendo: qualora sia risolubile).
	
	\hrule
	\vspace{0.5cm}
	
	\begin{sol}
		Riscriviamo l'equazione nella forma operatoriale omogenea:
		\[
		(T - \lambda I)\psi = f.
		\]
		Definiamo l'operatore $S_\lambda \doteq T - \lambda I$.
		
		\subsection*{1. Proprietà Spettrali Preliminari}
		Poiché $T$ è autoaggiunto, i suoi autovalori sono reali. Essendo $\lambda$ un autovalore per ipotesi, segue che $\lambda \in \R$.
		Inoltre, essendo $T$ e $I$ operatori limitati e autoaggiunti, anche $S_\lambda$ è limitato e autoaggiunto:
		\[
		S_\lambda^* = (T - \lambda I)^* = T^* - \bar{\lambda} I^* = T - \lambda I = S_\lambda.
		\]
		Dato che $T$ è compatto e $\lambda \neq 0$, l'operatore $S_\lambda$ è un \textbf{operatore di Fredholm} esattamente come nell'esercizio 1. Valgono le seguenti proprietà, dimostrate nell'esercizio 1:
		\begin{enumerate}
			\item Il nucleo $\Ker(S_\lambda)$ ha dimensione finita.
			\item Il rango $\Ran(S_\lambda)$ è chiuso in $\HH$.
		\end{enumerate}
		
		\subsection*{2. Analisi dell'Esistenza e Molteplicità}
		L'ipotesi che $\lambda$ sia un autovalore di $T$ implica che il nucleo di $S_\lambda$ non è banale:
		\[
		\Ker(S_\lambda) \neq \{0\}.
		\]
		Sia $d = \dim(\Ker(S_\lambda)) \ge 1$.
		
		\paragraph{Condizione di Risolubilità.}
		Affinché l'equazione $S_\lambda \psi = f$ ammetta soluzioni, il termine noto $f$ deve appartenere all'immagine dell'operatore. Poiché il rango è chiuso e l'operatore è autoaggiunto, vale la decomposizione ortogonale:
		\[
		\Ran(S_\lambda) = (\Ker(S_\lambda^*))^\perp = (\Ker(S_\lambda))^\perp.
		\]
		Quindi, l'equazione ammette soluzioni se e solo se $f \perp \Ker(S_\lambda)$.
		\textit{Nota: Se $f$ non soddisfa questa condizione, l'insieme delle soluzioni è vuoto. Assumeremo nel seguito che $f$ sia compatibile o che l'esercizio richieda di discutere la cardinalità dell'insieme delle soluzioni nel caso in cui questo non sia vuoto.}
		
		\paragraph{Struttura dello Spazio delle Soluzioni.}
		Supponiamo che esista almeno una soluzione particolare $\psi_p$ tale che $S_\lambda \psi_p = f$.
		La soluzione generale dell'equazione lineare non omogenea è data dalla somma della soluzione particolare e della soluzione generale dell'equazione omogenea associata ($S_\lambda \phi = 0$).
		L'insieme delle soluzioni $\Sigma$ è quindi lo spazio affine:
		\[
		\Sigma = \psi_p + \Ker(S_\lambda) = \{ \psi_p + \phi \mid \phi \in \Ker(S_\lambda) \}.
		\]
		Poiché $\lambda$ è un autovalore, esiste almeno un autovettore $u \neq 0$ tale che $u \in \Ker(S_\lambda)$.
		Consideriamo la famiglia di vettori:
		\[
		\psi_\alpha = \psi_p + \alpha u, \quad \forall \alpha \in \C.
		\]
		Verifichiamo che sono soluzioni:
		\[
		S_\lambda(\psi_\alpha) = S_\lambda \psi_p + \alpha S_\lambda u = f + \alpha \cdot 0 = f.
		\]
		Essendo $\alpha$ un parametro continuo in $\C$, la famiglia $\{\psi_\alpha\}$ contiene infiniti elementi distinti.
		Pertanto, se l'equazione è risolubile, essa ammette infinite soluzioni.
		
		\qed
	\end{sol}
	
	\section*{Traccia dell'Esercizio 7 (Metodo Spettrale)}
	Sia $\HH$ uno spazio di Hilbert separabile e si consideri l'equazione:
	\[
	T\psi = \lambda \psi + f,
	\]
	con $T = T^* \in \BB_\infty(\HH)$, $\lambda \in \C \setminus \{0\}$ e $f \in \HH$.
	Si provi che, se $\lambda$ è autovalore di $T$, allora l'equazione ha infinite soluzioni (assumendo la compatibilità del termine noto).
	
	\hrule
	\vspace{0.5cm}
	
	\begin{sol}
		Utilizziamo il \textbf{Teorema Spettrale di Hilbert-Schmidt}.
		Poiché $T$ è un operatore compatto e autoaggiunto su uno spazio di Hilbert separabile, esiste una base ortonormale completa $\{e_n\}_{n=1}^\infty$ costituita da autovettori di $T$, con autovalori reali $\{\mu_n\}_{n=1}^\infty$ tali che $\mu_n \to 0$ per $n \to \infty$.
		
		\subsection*{1. Diagonalizzazione del Problema}
		Scomponiamo l'incognita $\psi$ e il termine noto $f$ lungo la base spettrale $\{e_n\}$:
		\[
		\psi = \sum_{n=1}^\infty c_n e_n, \quad f = \sum_{n=1}^\infty f_n e_n,
		\]
		dove $c_n = \inner{e_n}{\psi}$ sono le incognite e $f_n = \inner{e_n}{f}$ sono dati.
		
		L'equazione $T\psi - \lambda \psi = f$ diventa:
		\[
		T\left( \sum_{n=1}^\infty c_n e_n \right) - \lambda \sum_{n=1}^\infty c_n e_n = \sum_{n=1}^\infty f_n e_n.
		\]
		Sfruttando la linearità e la proprietà degli autovettori ($T e_n = \mu_n e_n$):
		\[
		\sum_{n=1}^\infty (\mu_n - \lambda) c_n e_n = \sum_{n=1}^\infty f_n e_n.
		\]
		Per l'unicità dello sviluppo in serie di Fourier, uguagliamo i coefficienti componente per componente:
		\[
		(\mu_n - \lambda) c_n = f_n, \quad \forall n \in \N.
		\]
		
		\subsection*{2. Analisi delle Soluzioni}
		Dobbiamo risolvere per i coefficienti $c_n$. Distinguiamo gli indici in base al valore dell'autovalore $\mu_n$:
		
		\paragraph{Caso A: Indici $n$ tali che $\mu_n \neq \lambda$.}
		In questo caso il fattore $(\mu_n - \lambda)$ è non nullo e possiamo dividere:
		\[
		c_n = \frac{f_n}{\mu_n - \lambda}.
		\]
		Questi coefficienti sono determinati univocamente dal termine noto $f$.
		
		\paragraph{Caso B: Indici $n$ tali che $\mu_n = \lambda$.}
		Poiché $\lambda$ è un autovalore per ipotesi, l'insieme di indici $J_\lambda = \{ n \in \N \mid \mu_n = \lambda \}$ non è vuoto. Per tali indici l'equazione algebrica diventa:
		\[
		(\lambda - \lambda) c_n = f_n \implies 0 \cdot c_n = f_n.
		\]
		Da qui deduciamo due conseguenze fondamentali:
		\begin{enumerate}
			\item \textbf{Condizione di esistenza:} Affinché l'uguaglianza sia possibile, deve essere necessariamente $f_n = 0$ per ogni $n \in J_\lambda$. Geometricamente, questo significa che $f$ deve essere ortogonale all'autospazio relativo a $\lambda$ (ossia $f \in \Ker(T-\lambda I)^\perp$).
			
			\item \textbf{Molteplicità delle soluzioni:} Se la condizione di esistenza è soddisfatta ($f_n=0$), l'equazione $0 \cdot c_n = 0$ è verificata per \textbf{qualsiasi valore complesso} di $c_n$.
		\end{enumerate}
		
		\subsection*{Conclusione}
		Essendo $\lambda$ un autovalore, esiste almeno un indice $k \in J_\lambda$. Il coefficiente $c_k$ associato è totalmente arbitrario.
		La soluzione generale si scrive come:
		\[
		\psi = \underbrace{\sum_{n \notin J_\lambda} \frac{f_n}{\mu_n - \lambda} e_n}_{\psi_{\text{particolare}}} + \underbrace{\sum_{n \in J_\lambda} \alpha_n e_n}_{\psi_{\text{omogenea}}}, \quad \text{con } \alpha_n \in \C \text{ arbitrari}.
		\]
		La presenza di almeno un parametro libero $\alpha_k$ implica che, se il problema ammette soluzione, allora ne ammette infinite.
		
		\qed
	\end{sol}
	
	\section*{Traccia dell'Esercizio 8}
	Si consideri una particella di massa $m > 0$ in $\mathbb{R}^3$, soggetta a un potenziale centrale $V(r)$. Sia dato l'operatore:
	\[
	T = \vecOp{r} \cdot \vecOp{p} + \vecOp{p} \cdot \vecOp{r},
	\]
	dove $\vecOp{r}$ e $\vecOp{p}$ sono gli operatori posizione e impulso.
	Si mostri che, per ogni $\psi \in L^2(\mathbb{R}^3)$ con $\norm{\psi}=1$ (e nel dominio dell'operatore), vale l'equazione di evoluzione:
	\[
	\frac{d}{dt}\expval{T} = \frac{2}{m}\expval{\vecOp{p}^2} - 2\expval{\vecOp{r} \cdot \nabla V}.
	\]
	
	\hrule
	\vspace{0.5cm}
	
	\begin{sol}
		La risoluzione dell'esercizio si basa sull'applicazione del \textbf{Teorema di Ehrenfest}, che lega la derivata temporale del valore di aspettazione di un osservabile al commutatore dell'osservabile con l'Hamiltoniana.
		
		\subsection*{Ipotesi minimali per il Teorema di Ehrenfest}
		
		Sia $H$ l'operatore Hamiltoniano autoaggiunto su un dominio denso $D(H) \subset \mathcal{H}$ e sia $A$ un'osservabile autoaggiunta (con $\partial_t A = 0$) definita su $D(A)$.
		L'evoluzione temporale è data dal gruppo unitario fortemente continuo $U(t) = e^{-itH/\hbar}$.
		
		\subsubsection*{1. Ruolo del Teorema di Stone e Domini}
		Per poter derivare rispetto al tempo il valore di aspettazione $\langle A \rangle_{\psi_t}$, dobbiamo garantire che lo stato $\psi_t$ sia derivabile.
		Il \textbf{Teorema di Stone} stabilisce una corrispondenza biunivoca tra generatori autoaggiunti e gruppi unitari, affermando che la relazione:
		\[
		\frac{d}{dt} U(t)\psi = -\frac{i}{\hbar} H U(t)\psi
		\]
		è valida (nella topologia forte) \textbf{se e solo se} $\psi \in D(H)$.
		Pertanto, l'ipotesi $\psi \in L^2(\mathbb{R})$ \textbf{non è sufficiente}. Se $\psi \in L^2 \setminus D(H)$, la funzione $t \mapsto \psi_t$ è continua ma non derivabile, rendendo privo di senso il membro sinistro dell'equazione di Ehrenfest.
		
		\textbf{Ipotesi minimali:}
		Affinché la relazione di Ehrenfest valga come uguaglianza tra forme quadratiche all'istante $t$, richiediamo:
		\begin{enumerate}
			\item $\psi_t \in D(H)$ (per l'esistenza della derivata temporale);
			\item $\psi_t \in D(A)$ (per l'esistenza del valore di aspettazione).
		\end{enumerate}
		
		\subsubsection*{2. Derivazione come Forma Quadratica}
		Sotto le ipotesi sopra citate, non è necessario che $\psi_t \in D(HA)$ o $D(AH)$ (cioè che il commutatore esista come operatore). Interpretiamo il commutatore come una forma quadratica derivando il prodotto scalare:
		\begin{align*}
			\frac{d}{dt} \langle \psi_t, A \psi_t \rangle &= \langle \dot{\psi}_t, A \psi_t \rangle + \langle \psi_t, A \dot{\psi}_t \rangle \\
			&= \left\langle -\frac{i}{\hbar} H \psi_t, A \psi_t \right\rangle + \left\langle \psi_t, A \left( -\frac{i}{\hbar} H \psi_t \right) \right\rangle \\
			&= \frac{1}{i\hbar} \left( \langle A \psi_t, H \psi_t \rangle - \langle H \psi_t, A \psi_t \rangle \right).
		\end{align*}
		Definendo la media del commutatore in senso debole:
		\[
		\langle [H, A] \rangle_{\psi}^{weak} := \langle H\psi, A\psi \rangle - \langle A\psi, H\psi \rangle,
		\]
		otteniamo la relazione $\frac{d}{dt} \langle A \rangle = \frac{i}{\hbar} \langle [H, A] \rangle^{weak}$.
		
		
		
		\subsection*{Calcolo del Commutatore}
		L'Hamiltoniana per una particella in un potenziale $V(r)$ è data da:
		\[
		H = \frac{\vecOp{p}^2}{2m} + V(\vecOp{r}).
		\]
		Dobbiamo calcolare $[T, H]$. Per linearità:
		\[
		[T, H] = \left[ T, \frac{\vecOp{p}^2}{2m} \right] + [T, V(\vecOp{r})].
		\]
		
		\subsubsection*{1. Commutatore con l'Energia Cinetica}
		Calcoliamo $\left[ T, \frac{\vecOp{p}^2}{2m} \right] = \frac{1}{2m} [ \vecOp{r} \cdot \vecOp{p} + \vecOp{p} \cdot \vecOp{r}, \vecOp{p}^2 ]$.
		Poiché l'operatore impulso commuta con se stesso ($[\vecOp{p}, \vecOp{p}^2] = 0$), il termine $\vecOp{p} \cdot \vecOp{r}$ si semplifica notevolmente usando la regola $[AB, C] = A[B,C] + [A,C]B$:
		\[
		[\vecOp{p} \cdot \vecOp{r}, \vecOp{p}^2] = \vecOp{p} \cdot [\vecOp{r}, \vecOp{p}^2] + \underbrace{[\vecOp{p}, \vecOp{p}^2]}_{0} \cdot \vecOp{r} = \vecOp{p} \cdot [\vecOp{r}, \vecOp{p}^2].
		\]
		Analogamente per il primo termine:
		\[
		[\vecOp{r} \cdot \vecOp{p}, \vecOp{p}^2] = [\vecOp{r}, \vecOp{p}^2] \cdot \vecOp{p} + \vecOp{r} \cdot \underbrace{[\vecOp{p}, \vecOp{p}^2]}_{0} = [\vecOp{r}, \vecOp{p}^2] \cdot \vecOp{p}.
		\]
		Utilizziamo l'identità fondamentale $[x_j, \vecOp{p}^2] = 2i\hbar p_j$, che in notazione vettoriale si scrive $[\vecOp{r}, \vecOp{p}^2] = 2i\hbar \vecOp{p}$.
		Sostituendo:
		\begin{itemize}
			\item Primo termine: $[\vecOp{r} \cdot \vecOp{p}, \vecOp{p}^2] = (2i\hbar \vecOp{p}) \cdot \vecOp{p} = 2i\hbar \vecOp{p}^2$.
			\item Secondo termine: $[\vecOp{p} \cdot \vecOp{r}, \vecOp{p}^2] = \vecOp{p} \cdot (2i\hbar \vecOp{p}) = 2i\hbar \vecOp{p}^2$.
		\end{itemize}
		Sommando i contributi:
		\[
		\left[ T, \frac{\vecOp{p}^2}{2m} \right] = \frac{1}{2m} (2i\hbar \vecOp{p}^2 + 2i\hbar \vecOp{p}^2) = \frac{4i\hbar \vecOp{p}^2}{2m} = \frac{2i\hbar}{m} \vecOp{p}^2.
		\]
		
		\subsubsection*{2. Commutatore con l'Energia Potenziale}
		Calcoliamo $[T, V(\vecOp{r})]$.
		Poiché $V(\vecOp{r})$ è funzione solo delle coordinate, commuta con $\vecOp{r}$. Quindi:
		\[
		[\vecOp{r} \cdot \vecOp{p}, V] = \vecOp{r} \cdot [\vecOp{p}, V] \quad \text{e} \quad [\vecOp{p} \cdot \vecOp{r}, V] = [\vecOp{p}, V] \cdot \vecOp{r}.
		\]
		Utilizziamo la relazione fondamentale $[\vecOp{p}, V] = -i\hbar \nabla V$.
		\begin{itemize}
			\item Primo termine: $\vecOp{r} \cdot [\vecOp{p}, V] = \vecOp{r} \cdot (-i\hbar \nabla V) = -i\hbar (\vecOp{r} \cdot \nabla V)$.
			\item Secondo termine: $[\vecOp{p}, V] \cdot \vecOp{r} = (-i\hbar \nabla V) \cdot \vecOp{r} = -i\hbar (\vecOp{r} \cdot \nabla V)$ (dato che $V$ è scalare).
		\end{itemize}
		Sommando i contributi:
		\[
		[T, V] = -2i\hbar (\vecOp{r} \cdot \nabla V).
		\]
		
		\subsection*{Conclusione}
		Unendo i risultati parziali, il commutatore totale è:
		\[
		[T, H] = \frac{2i\hbar}{m} \vecOp{p}^2 - 2i\hbar (\vecOp{r} \cdot \nabla V).
		\]
		Inserendo questo risultato nell'equazione di Ehrenfest derivata all'inizio:
		\[
		\frac{d}{dt}\expval{T} = \frac{1}{i\hbar} \expval{ \frac{2i\hbar}{m} \vecOp{p}^2 - 2i\hbar (\vecOp{r} \cdot \nabla V) }.
		\]
		Semplificando il fattore $i\hbar$, otteniamo la tesi:
		\[
		\frac{d}{dt}\expval{T} = \frac{2}{m}\expval{\vecOp{p}^2} - 2\expval{\vecOp{r} \cdot \nabla V}.
		\]
		
		\qed
	\end{sol}
	\section*{Traccia dell'Esercizio 9 Versione Hard}
	Siano $T: \Dom(T) \to \HH$ e $V: \Dom(V) \to \HH$ due operatori su uno spazio di Hilbert $\HH$ tali che:
	\begin{enumerate}
		\item[(a)] $T$ è autoaggiunto ($T = T^*$);
		\item[(b)] $V$ è simmetrico;
		\item[(c)] $V$ è $T$-limitato con limite relativo $a < 1$. Ovvero, $\Dom(T) \subset \Dom(V)$ ed esistono costanti $a \in [0, 1)$ e $b \ge 0$ tali che:
		\[
		\norm{V\psi} \le a\norm{T\psi} + b\norm{\psi}, \quad \forall \psi \in \Dom(T).
		\]
	\end{enumerate}
	Si mostri che l'operatore somma $S \doteq T + V$, definito su $\Dom(S) = \Dom(T)$, è autoaggiunto.
	
	\hrule
	\vspace{0.5cm}
	\begin{sol}
		La dimostrazione si basa sul criterio di suriettività dei ranghi per operatori simmetrici.
		L'obiettivo è dimostrare che esiste un valore $\nu > 0$ sufficientemente grande tale che:
		\[
		\Ran(T + V \pm i\nu I) = \HH.
		\]
		Se ciò è vero, poiché $T+V$ è simmetrico, esso è necessariamente autoaggiunto.
		
		\subsection*{1. Stime sul Risolvente dell'Operatore Non Perturbato}
		Consideriamo l'operatore non perturbato $T$. Poiché $T$ è autoaggiunto, per ogni $\mu > 0$ e per ogni $\phi \in \Dom(T)$ vale l'identità pitagorica:
		\[
		\norm{(T \pm i\mu I)\phi}^2 = \inner{(T \pm i\mu I)\phi}{(T \pm i\mu I)\phi} = \norm{T\phi}^2 + \mu^2\norm{\phi}^2.
		\]
		(I termini misti si cancellano per simmetria di $T$).
		Da questa uguaglianza seguono immediatamente due disuguaglianze fondamentali. Ponendo $\psi = (T + i\mu I)\phi$ (e quindi $\phi = (T + i\mu I)^{-1}\psi$, dato che quell'operatore è invertibile essendo T autoaggiunto), abbiamo:
		\begin{enumerate}
			\item $\mu^2 \norm{\phi}^2 \le \norm{\psi}^2 \implies \norm{(T + i\mu I)^{-1}} \le \frac{1}{\mu}$.
			\item $\norm{T\phi}^2 \le \norm{\psi}^2 \implies \norm{T(T + i\mu I)^{-1}} \le 1$.
		\end{enumerate}
		
		\subsection*{2. Stima dell'Operatore Perturbativo}
		Vogliamo stimare "quanto disturba" $V$ rispetto al risolvente di $T$. Consideriamo il vettore $$\phi = (T + i\mu I)^{-1}\psi$$ per un generico $\psi \in \HH$.
		Applichiamo la condizione di limitatezza relativa di $V$ (ipotesi c):
		\[
		\norm{V\phi} \le a\norm{T\phi} + b\norm{\phi}.
		\]
		Sostituendo $\phi$ con l'espressione in termini di $\psi$ e utilizzando le stime del punto 1:
		\begin{align*}
			\norm{V(T + i\mu I)^{-1}\psi} &\le a \norm{T(T + i\mu I)^{-1}\psi} + b \norm{(T + i\mu I)^{-1}\psi} \\
			&\le a \cdot 1 \cdot \norm{\psi} + b \cdot \frac{1}{\mu} \cdot \norm{\psi} \\
			&= \left( a + \frac{b}{\mu} \right) \norm{\psi}.
		\end{align*}
		
		\subsection*{3. Costruzione dell'Inversa tramite Serie di Neumann}
		Definiamo l'operatore $U_\mu \doteq V(T + i\mu I)^{-1}$. Abbiamo appena dimostrato che la sua norma operatoriale è limitata da:
		\[
		\norm{U_\mu} \le a + \frac{b}{\mu}.
		\]
		Poiché per ipotesi $a < 1$, possiamo scegliere un valore $\mu = \nu$ sufficientemente grande tale che:
		\[
		a + \frac{b}{\nu} < 1 \implies \norm{U_\nu} < 1.
		\]
		Intanto otteniamo che, avendo la norma limitata è un operatore limitato definito su tutto $\HH$. Se la norma di un operatore $U_\nu$ è strettamente minore di 1, allora $-1$ non appartiene al suo spettro ($\sigma(U_\nu)$), e l'operatore $(I + U_\nu)$ è invertibile con inverso limitato). In particolare, il rango di $(I + U_\nu)$ è tutto lo spazio $\HH$:
		\[
		\Ran(I + U_\nu) = \HH.
		\]
		
		\subsection*{4. Fattorizzazione e Conclusione}
		Consideriamo ora l'operatore perturbato $(T + V + i\nu I)$. Possiamo fattorizzarlo come segue per ogni $\phi \in \Dom(T)$:
		\begin{align*}
			(T + V + i\nu I)\phi &= (T + i\nu I)\phi + V\phi \\
			&= \left[ I + V(T + i\nu I)^{-1} \right] (T + i\nu I)\phi \\
			&= (I + U_\nu)(T + i\nu I)\phi.
		\end{align*}
		Analizziamo i ranghi di questa composizione:
		\begin{itemize}
			\item L'operatore $(T + i\nu I)$ mappa suriettivamente $\Dom(T)$ su $\HH$ (poiché $T$ è autoaggiunto).
			\item L'operatore $(I + U_\nu)$ mappa suriettivamente $\HH$ su $\HH$ (poiché $\norm{U_\nu} < 1$).
		\end{itemize}
		La composizione di due mappe suriettive è suriettiva. Dunque:
		\[
		\Ran(T + V + i\nu I) = \HH.
		\]
		Un ragionamento perfettamente analogo vale per il segno opposto $-i\nu$, dimostrando che $$\Ran(T + V - i\nu I) = \HH$$.
		Avendo dimostrato che gli operatori di difetto sono suriettivi (o equivalentemente che gli indici di difetto sono $(0,0)$), concludiamo che $T+V$ è autoaggiunto sul dominio $\Dom(T)$.
		
		\qed
	\end{sol}
	
		\section*{Traccia dell'Esercizio 9 Versione Soft - Confermata}
	Siano $T: \Dom(T) \to \HH$ e $V: \Dom(V) \to \HH$ due operatori su uno spazio di Hilbert $\HH$ tali che:
	\begin{enumerate}
		\item[(a)] $T$ è autoaggiunto ($T = T^*$);
		\item[(b)] $V$ è simmetrico;
		\item[(c)] $V$ è $T$-limitato con limite relativo $a < 1$. Ovvero, $T \subset V$ ed esistono costanti $a \in [0, 1)$ e $b \ge 0$ tali che:
		\[
		\norm{V\psi} \le a\norm{T\psi} + b\norm{\psi}, \quad \forall \psi \in \Dom(T).
		\]
	\end{enumerate}
	Si mostri che l'operatore somma $S \doteq T + V$, definito su $\Dom(S) = \Dom(T)$, è autoaggiunto.
	
	\hrule
	\vspace{0.5cm}
	\begin{sol}
		$S = T + V$ sul dominio di di T è $2T$ che è autoaggiunto.
	\end{sol}
	
	\section*{Traccia dell'Esercizio 10 Metodo I (non corretto)}
	Sia $\HH$ uno spazio di Hilbert con norma $\norm{\cdot}$. Sia $\norm{\cdot}'$ una seconda norma su $\HH$ tale che $\exists C > 0$:
	\[ \norm{\psi} \le C\normprime{\psi}, \quad \forall \psi \in \HH. \]
	Sia $T: D(T) \to \HH$ un operatore simmetrico tale che $\exists \kappa > 0$:
	\[ \normprime{T\psi} \le \kappa\normprime{\psi}, \quad \forall \psi \in \HH. \]
	Mostrare che $T \in \mathcal{B}(\HH)$.
	\begin{quote}
		\textit{[Hint: Un operatore lineare chiuso tra spazi di Banach e con dominio denso (inteso come tutto lo spazio) è limitato]}
	\end{quote}
	
	\hrule
	\vspace{0.5cm}
	
	\begin{sol}
		Per dimostrare che $T \in \mathcal{B}(\HH)$ (ovvero che $T$ è limitato rispetto alla norma standard $\norm{\cdot}$), utilizzeremo il suggerimento fornito, che è un richiamo al Teorema del Grafico Chiuso.
		La strategia si divide in due passi logici:
		\begin{enumerate}
			\item Identificazione del dominio e dimostrazione che $T$ è un operatore chiuso su $(\HH, \norm{\cdot})$.
			\item Applicazione del Teorema del Grafico Chiuso.
		\end{enumerate}
		
		\subsection*{1. Analisi del Dominio e Chiusura}
		La seconda disuguaglianza fornita nel testo afferma che:
		\[ \normprime{T\psi} \le \kappa\normprime{\psi}, \quad \forall \psi \in \HH. \]
		L'uso del quantificatore $\forall \psi \in \HH$ implica necessariamente che il dominio di $T$ coincide con l'intero spazio di Hilbert:
		\[ D(T) = \HH. \]
		Inoltre, per ipotesi, $T$ è un operatore \textbf{simmetrico}. Un operatore simmetrico $T$ definito su tutto lo spazio di Hilbert è sempre un operatore \textbf{chiuso}.
		Dimostriamolo formalmente per completezza (senza dare per scontato il teorema di Hellinger-Toeplitz).
		
		Sia $\{ \psi_n \}_{n \in \mathbb{N}} \subset \HH$ una successione tale che:
		\[ \psi_n \xrightarrow{\norm{\cdot}} \psi \quad \text{e} \quad T\psi_n \xrightarrow{\norm{\cdot}} \phi. \]
		Dobbiamo mostrare che $\phi = T\psi$.
		Poiché $T$ è simmetrico, per ogni $\eta \in \HH$ vale l'uguaglianza:
		\[ \inner{T\psi_n}{\eta} = \inner{\psi_n}{T\eta}. \]
		Passando al limite per $n \to \infty$ e sfruttando la continuità del prodotto scalare:
		\[ \lim_{n \to \infty} \inner{T\psi_n}{\eta} = \inner{\phi}{\eta}, \]
		\[ \lim_{n \to \infty} \inner{\psi_n}{T\eta} = \inner{\psi}{T\eta}. \]
		Quindi:
		\[ \inner{\phi}{\eta} = \inner{\psi}{T\eta}. \]
		Sfruttando nuovamente la simmetria di $T$ sul membro di destra:
		\[ \inner{\phi}{\eta} = \inner{T\psi}{\eta}, \quad \forall \eta \in \HH. \]
		Poiché l'uguaglianza vale per ogni $\eta$, segue che $\phi = T\psi$.
		Abbiamo così dimostrato che il grafico di $T$ è chiuso rispetto alla topologia indotta dalla norma $\norm{\cdot}$.
		
		\subsection*{2. Applicazione del Teorema del Grafico Chiuso}
		Siamo ora nelle seguenti condizioni:
		\begin{itemize}
			\item $T: \HH \to \HH$ è un operatore lineare.
			\item $\HH$ è uno spazio di Hilbert (e quindi di Banach) rispetto alla norma $\norm{\cdot}$.
			\item $T$ è un operatore \textbf{chiuso} rispetto a $\norm{\cdot}$.
			\item $D(T) = \HH$ (il dominio è l'intero spazio).
		\end{itemize}
		
		Il \textbf{Teorema del Grafico Chiuso} afferma che un operatore chiuso definito su tutto uno spazio di Banach a valori in uno spazio di Banach è necessariamente continuo (limitato).
		Pertanto:
		\[ T \in \mathcal{B}(\HH). \]
		
		\begin{oss}[Sulle norme ausiliarie]
			Le condizioni sulla norma ausiliaria $\norm{\cdot}'$ e la limitatezza di $T$ rispetto ad essa (ossia $T \in \mathcal{B}(\HH, \norm{\cdot}')$) sono condizioni sufficienti a garantire la buona definizione dell'operatore, ma la dimostrazione della limitatezza in norma standard $\norm{\cdot}$ segue direttamente dalla struttura simmetrica dell'operatore e dal fatto che sia definito ovunque (Teorema di Hellinger-Toeplitz). L'esercizio è quindi risolubile in modo rigoroso basandosi primariamente sulle proprietà spettrali (Simmetria + Dominio totale $\implies$ Chiusura $\implies$ Limitatezza).
		\end{oss}
		
		\qed
	\end{sol}
	
	\section*{Traccia dell'Esercizio 10 Metodo II (confermato)}
	Sia $\HH$ uno spazio di Hilbert con norma $\norm{\cdot}$. Sia $\norm{\cdot}'$ una seconda norma su $\HH$ tale che $\exists C > 0$:
	\[ \norm{\psi} \le C\normprime{\psi}, \quad \forall \psi \in \HH. \]
	Sia $T: D(T) \to \HH$ un operatore simmetrico tale che $\exists \kappa > 0$:
	\[ \normprime{T\psi} \le \kappa\normprime{\psi}, \quad \forall \psi \in \HH'. \]
	dove per $\HH'=(\HH, ||\cdot||')$ si intende lo spazio di Hilbert definito dagli elementi di $(\HH, ||\cdot||)$ che sono finiti rispetto alla norma $||\cdot||'$ decorato della stessa.
	Mostrare che $T$ è limitato rispetto alla norma $\norm{\cdot}$ (ovvero $T \in \mathcal{B}(\HH)$ nel senso dell'estensione).
	
	\begin{quote}
		\textit{[Hint: Un operatore lineare chiuso tra spazi di Banach e con dominio denso (inteso come tutto lo spazio) è limitato]}
	\end{quote}
	
	\hrule
	\vspace{0.5cm}
	
	\begin{sol}
		L'obiettivo è dimostrare che l'operatore $T$ è limitato rispetto alla norma naturale dello spazio di Hilbert $\norm{\cdot}$, ossia che esiste $M > 0$ tale che $\norm{T\psi} \le M\norm{\psi}$ per ogni $\psi \in D(T)$.
		
		\subsection*{1. Equivalenza delle Norme}
		Consideriamo i due spazi di Banach (ridenominiamoli se no si incrociano gli occhi):
		\[ X \doteq (\HH, \norm{\cdot}) \quad \text{e} \quad Y \doteq (\HH, \norm{\cdot}'). \]
		Definiamo l'operatore \textbf{Identità} che mappa dallo spazio con norma standard allo spazio con norma "prima":
		\[
		J: X \to Y, \quad J\psi = \psi.
		\]
		Questo operatore è lineare ed è definito su tutto lo spazio $\HH$ (dominio denso e completo). Per poter applicare l'Hint e concludere che $J$ è limitato, dobbiamo verificare che $J$ sia un \textbf{operatore chiuso}.
		
		\textbf{Verifica della chiusura di $J$:}
		Consideriamo una successione $\{\psi_n\}_{n \in \mathbb{N}} \subset \HH$ tale che:
		\begin{enumerate}
			\item $\psi_n \to \psi$ nello spazio $X$ (convergenza rispetto a $\norm{\cdot}$);
			\item $J\psi_n \to \phi$ nello spazio $Y$ (convergenza rispetto a $\norm{\cdot}'$).
		\end{enumerate}
		Dobbiamo mostrare che $J\psi = \phi$, ovvero che $\psi = \phi$.
		
		Sfruttiamo l'ipotesi data dalla traccia: $\norm{u} \le C \normprime{u}$. Questa disuguaglianza implica che la convergenza nella norma $\norm{\cdot}'$ è più forte della convergenza nella norma $\norm{\cdot}$.
		Poiché $\psi_n \to \phi$ nella norma $\norm{\cdot}'$ (punto 2), allora $\psi_n \to \phi$ anche nella norma $\norm{\cdot}$.
		Tuttavia, per il punto 1, sappiamo che $\psi_n \to \psi$ nella norma $\norm{\cdot}$.
		Per l'unicità del limite, deve essere:
		\[ \psi = \phi. \]
		Il grafico di $J$ è dunque chiuso.
		
		\textbf{Applicazione dell'Hint:}
		L'operatore $J$ soddisfa tutte le condizioni dell'Hint: è lineare, chiuso, e definito su tutto lo spazio di Banach $X$. Pertanto, $J$ è \textbf{limitato}.
		Esiste quindi una costante $K > 0$ tale che $\norm{J\psi}_Y \le K \norm{\psi}_X$, che si traduce in:
		\begin{equation} \label{eq:equiv}
			\normprime{\psi} \le K \norm{\psi}, \quad \forall \psi \in \HH.
		\end{equation}
		
		\subsection*{2. Dimostrazione della Limitatezza di $T$}
		Per ogni $\psi \in D(T)$:
		\begin{align*}
			\norm{T\psi} &\le C \normprime{T\psi} & \text{(Dominazione della norma)} \\
			&\le C \cdot \kappa \normprime{\psi} & \text{(Limitatezza in } \HH') \\
			&\le C \cdot \kappa \cdot c \norm{\psi} & \text{(Equivalenza inversa \ref{eq:equiv})}
		\end{align*}
		Posto $M = C \kappa c$, si ha $\norm{T\psi} \le M \norm{\psi}$.
		
		\subsection*{3. Estensione a tutto $\HH$}
		Poiché $T$ è limitato su un dominio denso, ammette un'unica estensione continua (BLT Theorem) su tutto $\HH$, che coincide con la sua chiusura (essendo $T$ simmetrico). Dunque $T \in \mathcal{B}(\HH)$.
	\end{sol}
	
\end{document}